\subsection{Type casting "selvaggio"}
\label{sec:5-15-wild-type-casting}

L'ultima particolarità della traduzione riguarda una stranezza occasionalmente necessaria quando
le espressioni parte di un'applicazione richiedono un ulteriore \textit{cast}: questi \textit{cast} non sempre sono davvero necessari
ma prevederne l'esigenza a priori è difficile, ed è il motivo per cui sono stati battezzati "selvaggi" (\textit{wild cast}).


L'implementazione dei \textit{wild cast} è data da una \textit{flag} booleana abilitata nel nodo \texttt{Application}
qualora le espressioni coinvolte siano oggetti \texttt{Lambda} o \texttt{Let}, mentre il \textit{cast} è applicato
nei metodi \texttt{visit} interessati.

\vspace{4mm}
\begin{lstlisting}[caption={Metodo \texttt{visit} per le applicazioni di funzione}, style=javaCode, label={lst:5-15-visit-application-java}]
@Override
public Void visitApplication(Expression.Application application) {
    // left and right expressions necessitate a wild cast
    // if they are lambda or let expressions
    if (application.left instanceof Expression.Lambda
            || application.left instanceof Expression.Let)
        wildCast = true;
    visit(application.left);

    builder.append(".apply(");
    if (application.right instanceof Expression.Lambda
            || application.right instanceof Expression.Let)
        wildCast = true;
    visit(application.right);
    builder.append(")");
    return null;
}
\end{lstlisting}
\vspace{4mm}
\begin{lstlisting}[caption={\textit{Wild cast} in espressioni \texttt{lambda} e \texttt{let}}, style=javaCode, label={lst:5-15-wild-casts-java}]
@Override
public Void visitLambda(Expression.Lambda lambda) {
    // ...
    if (wildCast) {
        // wild cast is needed for lambdas in applications
        builder.append("(").append(typeStringOf(lambda.type())).append(") ");
        wildCast = false;
    }
    // ...
    return null;
}

@Override
public Void visitLet(Expression.Let let) {
    // ...
    if (wildCast) {
        // wild cast is needed for lets in applications
        builder.append("(").append(typeStringOf(let.type())).append(") ");
        wildCast = false;
    }
    // ...
    return null;
}
\end{lstlisting}

\newpage

Nei Codici~\ref{lst:5-15-wild-funx}~e~\ref{lst:5-15-wild-java} si nota come questo approccio renda la traduzione ancora meno
leggibile, ma si può facilmente verificare che la compilazione fallisce se alcuni \textit{cast} sono rimossi
(\textit{cast} di destra in \texttt{reverseApply} e \textit{cast} di sinistra in \texttt{anonymousIds}).

\vspace{4mm}
\begin{lstlisting}[caption={Applicazione tra funzioni, espressioni \texttt{let} e \texttt{lambda}}, style=funxCode, label={lst:5-15-wild-funx}]
reverseApply = flip (let
        apply1 f x = f x
    in apply1)

anonymousIds = (\x -> x) (\x -> x)
\end{lstlisting}
\vspace{4mm}
\begin{lstlisting}[caption={Traduzione in \texttt{Java} con \textit{cast} "selvaggi"}, style=javaCode, label={lst:5-15-wild-java}]
public class Chapter5Wild {
    private Chapter5Wild() {
        // private constructor to prevent instantiation
    }
    
    public static @<j, k@> Function@<j, Function@<Function@<j, k@>, k@>> reverseApply() {
        return FunxPrelude.@<Function@<j, k@>, j, k@>flip()
            .apply(
                (Function@<Function@<j, k@>, Function@<j, k@>>) // right wild cast
                    (new Let@<@>() {
                        private @<h, i@> Function@<Function@<h, i@>, Function@<h, i@>> apply1() {
                            return (f @-> (x @-> f.apply(x)));
                        }
    
                        @Override
                        public Function@<Function@<j, k@>, Function@<j, k@>> _eval() {
                            return this.@<j, k@>apply1();
                        }
                    })._eval());
    }
    
    public static @<n@> Function@<n, n@> anonymousIds() {
        return ((Function@<Function@<n, n@>, Function@<n, n@>>) x @-> x) // left wild cast
            .apply(((Function@<n, n@>) x @-> x)); // right wild cast
    }
}
\end{lstlisting}