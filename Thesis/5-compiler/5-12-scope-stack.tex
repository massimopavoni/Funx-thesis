\subsection{Stack dei contesti}
\label{sec:5-12-scope-stack}

Durante l'esplorazione dell'\textbf{AST} è necessario tenere traccia dello \textit{scope}, il contesto, del quale le dichiarazioni
o i parametri lambda fanno parte: la soluzione adottata prevede l'uso di uno \textit{stack} contenente i contesti attivi,
rappresentati a loro volta da mappe con le variabili dichiarate come chiavi e le relative informazioni.

\noindent Lo \textit{stack} dei contesti è in realtà implementato attraverso una lista per via della necessità di iterare
su tutti i contesti e trovare una variabile richiesta (\textit{lookup}).

\vspace{4mm}
\begin{lstlisting}[caption={Struttura e metodi per lo \textit{scope}}, style=javaCode, label={lst:5-12-scope-stack-methods-java}]
private final List@<Map@<String, Utils.Tuple@<Scheme, String@>@>@>
    scopes = new ArrayList@<@>();

private void addToScope(List@<Declaration@> declarations, String scope) {
    scopes.addFirst(declarations.stream()
        .collect(ImmutableMap.toImmutableMap(
            decl @-> decl.id,
            decl @-> new Utils.Tuple@<@>(decl.scheme(), scope))));
}

private void addToScope(String id, Scheme scheme, String scope) {
    scopes.addFirst(Collections.singletonMap(id, new Utils.Tuple@<@>(scheme, scope)));
}
\end{lstlisting}
\vspace{4mm}
\begin{lstlisting}[caption={Aggiunta di contesto per modulo, lambda astrazioni e \texttt{let}}, style=javaCode, label={lst:5-12-lambda-let-scope-java}]
@Override
public Void visitModule(ASTNode.Module module) {
    // add Prelude functions to the scope
    scopes.addFirst(Arrays.stream(PreludeFunction.values())
        .collect(ImmutableMap.toImmutableMap(
            pf @-> pf.id,
            pf @-> new Utils.Tuple@<@>(pf.scheme, pf.nativeJava
                @? JavaPrelude.class.getSimpleName()
                @: FunxPrelude.class.getSimpleName()))));
    // add module functions to the scope
    addToScope(module.let.localDeclarations.declarationList, module.name);
    // ...
    scopes.removeFirst();
    return null;
}

@Override
public Void visitLambda(Expression.Lambda lambda) {
    // add lambda parameter to the scope
    addToScope(lambda.paramId, new Scheme(Collections.emptySet(),
        ((Type.FunctionApplication) lambda.type()).arguments.getFirst()), null);
    // ...
    scopes.removeFirst();
    return null;
}

@Override
public Void visitLet(Expression.Let let) {
    currentLevel++;
    // add let local declarations to the scope
    addToScope(let.localDeclarations.declarationList, "this");
    // ...
    // restore previous scope state and level
    scopes.removeFirst();
    currentLevel--;
    return null;
}
\end{lstlisting}

\newpage

\noindent Nei Codici~\ref{lst:5-12-scope-stack-methods-java}~e~\ref{lst:5-12-lambda-let-scope-java} sono mostrate:
\begin{itemize}
    \item la lista dei contesti e le funzioni per la loro gestione: le tuple assegnate ai nomi delle dichiarazioni di uno \textit{scope}
          conservano lo schema di tipo e il nome della classe statica in cui la variabile è definita
          (\texttt{JavaPrelude}, \texttt{FunxPrelude}, nome del modulo creato,
          \texttt{null} per parametri lambda e \texttt{this} per i \texttt{let});
    \item l'aggiunta dei contesti per \texttt{Module}, \texttt{Lambda} e \texttt{Let}, rispettivamente:
          funzioni di libreria e dichiarazioni del modulo stesso, parametro lambda
          e dichiarazioni locali con l'incremento del livello di annidamento.
\end{itemize}