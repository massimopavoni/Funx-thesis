\subsection{Analisi sintattica}
\label{sec:5-3-syntactic-analysis}

Il file \texttt{FunxParser.g4} contiene le regole concrete della grammatica di \textbf{Funx}:
nonostante la somiglianza con le grammatiche delle Figure~\ref{fig:2-3-funx-syntax}~e~\ref{fig:3-3-system-hm},
è evidente che queste non collimino esattamente a causa di zucchero sintattico e requisiti di \texttt{ANTLR}.

\noindent Lo strumento utilizzato, infatti, è un generatore di \textit{parser} di tipo \textit{top-down}
per grammatiche \textit{LL}, le quali in generale non supportano regole ricorsive a sinistra.

\noindent Essendo tali regole spesso comuni nella definizione di qualsiasi linguaggio di programmazione, \textbf{Funx} incluso,
\texttt{ANTLRv4} offre un diverso tipo di parsing, detto \textit{Adaptive LL(*)} \cite{Parr-2011-FoundationANTLR,Parr-2014-AdaptiveLL}:
quest'ultimo è in grado di riscrivere automaticamente le grammatiche, eliminando la ricorsione a sinistra diretta (e.g. linee 36 e 38-43),
così da non incorrere in regole ambigue che potrebbero causare \textit{backtracking} e conseguente \textit{overhead}.

\noindent Il Codice~\ref{lst:5-3-parser-antlr} riporta integralmente le regole concrete della sintassi di \textbf{Funx}, tra cui:
\begin{itemize}
    \item \textit{module}: nome del modulo, funzione \texttt{main} opzionale e dichiarazioni globali;
    \item \textit{main}: funzione \texttt{main}, diversa dalle dichiarazioni classiche per l'assenza di schema di tipo e parametri lambda;
    \item \textit{declaration}: funzione con nome, tipo e parametri (e opzionalmente \texttt{with} per funzioni locali);
    \item \textit{typeElems}: tipo di una funzione, definito ricorsivamente secondo la grammatica del sistema di tipo di \textbf{Funx};
    \item \textit{statement}: per evitare ricorsione a sinistra indiretta, la separazione tra \textit{statement} e \textit{expression}
          forza l'uso di parentesi nei casi in cui lambda astrazioni, let e if siano usati all'interno di un'espressione;
    \item \textit{expression}: racchiude l'applicazione funzionale, tutte le regole relative agli operatori simbolici,
          specificandone la priorità implicita (Tabella~\ref{tab:2-4-sugar}), e le espressioni primarie
          (costanti, variabili e parentesi per controllare la precedenza);
    \item \textit{lambda, let, if}: corrispondenti alle produzioni per astrazione, let e if della grammatica formale.
\end{itemize}

\newpage

\begin{lstlisting}[caption={Grammatica per il \textit{parser}}, style=antlrCode, label={lst:5-3-parser-antlr}]
parser grammar FunxParser;
options { tokenVocab = FunxLexer; }

// Module
module: (MODULE MODULEID (Dot MODULEID)* NEWLINE+)?
    (main NEWLINE+)? declarations EOF;

declarations: declaration (NEWLINE declaration?)*;

main: id = MAIN Equals statement with?;

// Declaration
declaration: (declarationScheme NEWLINE)?
    id = VARID lambdaParams? Equals statement with?;

declarationScheme: id = VARID Colon typeElems;

with: NEWLINE WITH localDeclarations OUT;

localDeclarations: NEWLINE declarations NEWLINE;

// Type
typeElems: OpenParen typeElems CloseParen %# parenType%
    | VARID %# typeVar%
    | TYPE %# namedType%
    | <assoc = right> typeElems Arrow typeElems %# arrowType%;

// Statement
statement: expression %# expressionStatement%
    | lambda %# lambdaStatement%
    | let %# letStatement%
    | ifS %# ifStatement%;

// Expression
expression: primary %# primExpression%
    | expression expression %# appExpression%
    | <assoc = right> expression bop = Dot expression %# composeExpression%
    | expression bop = (Divide | Modulo | Multiply) expression %# divModMultExpression%
    | expression bop = (Add | Subtract) expression %# addSubExpression%
    | expression
        bop = (GreaterThan | GreaterThanEquals | LessThan | LessThanEquals)
        expression %# compExpression%
    | expression bop = (EqualsEquals | NotEquals) expression %# eqExpression%
    | uop = Not expression %# notExpression%
    | <assoc = right> expression bop = And expression %# andExpression%
    | <assoc = right> expression bop = Or expression %# orExpression%
    | <assoc = right> expression bop = Dollar expression %# rightAppExpression%;

primary: OpenParen statement CloseParen %# parenPrimary%
    | constant %# constPrimary% | VARID %# varPrimary%;

// Lambda
lambda: Backslash lambdaParams? Arrow statement;

lambdaParams: VARID+;

// Let
let: LET localDeclarations IN statement;

// If
ifS: IF statement THEN statement ELSE statement FI;

// Constant
constant: BOOL | numConstant;

numConstant: INT;    
\end{lstlisting}