\section{Motore inferenziale}
\label{sec:5-7-inference-engine}

I sistemi di tipo e l'inferenza descritti nel Capitolo~\ref{chap:3-inference} costituiscono una parte imprescindibile del software,
data l'incapacità di simulare il polimorfismo parametrico del \textit{sistema HM} lasciando al compilatore \texttt{Java}
il compito di inferire e controllare i tipi dei metodi (o classi) generici.
In tal caso, infatti, la compilazione fallirebbe a causa di tipi "troppo generici" e \textit{type casting} non permesso
in maniera implicita (e.g. non si è in grado di conciliare ad esempio un tipo generico con una lambda espressione).

\begin{figure}
    \vspace{4mm}
    \includegraphics[width=\textwidth]{5-7-impossible-java.png}
    \caption{Esempio di errore in \texttt{Java} dovuto a tipi generici}
    \label{fig:5-7-impossible-java}
    \vspace{4mm}
\end{figure}

\noindent La linea di codice citata in Figura~\ref{fig:5-7-impossible-java} non è intrinsecamente errata,
ma non può essere compilata in mancanza di informazioni riguardanti il parametro di tipo generico della funzione \texttt{id}
(con tipo di ritorno \texttt{Function<T, T>}): il problema da risolvere è quindi conoscere il tipo dell'espressione
in input alla prima chiamata al metodo \texttt{apply}.


Il motore inferenziale è la parte del compilatore che si occupa di stabilire i tipi delle espressioni seguendo le regole
di inferenza e i passi dell'\textit{algoritmo $\mathcal{W}$}%
\footnote{\citetitle{Grabmuller-2006-AlgorithmW} \cite{Grabmuller-2006-AlgorithmW}}
descritti nella sezione~\ref{sec:3-4-hm-type-inference}; la fase di inferenza avviene subito dopo il parsing
e la costruzione dell'\textbf{AST} e precede la generazione del codice \texttt{Java}.

\subsection{Sistema HM}
\label{sec:5-8-system-hm}


\newpage

\subsection{Inferenza su espressioni}
\label{sec:5-9-expression-inference}
