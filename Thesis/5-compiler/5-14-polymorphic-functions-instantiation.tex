\subsection{Istanziazione di funzioni polimorfe}
\label{sec:5-14-polymorphic-functions-instantiation}

Dal momento che il contesto può cambiare con l'introduzione di parametri lambda oltre che con espressioni \texttt{let},
il livello di annidamento di quest'ultime è segnalato da una variabile secondaria utilizzata durante il \textit{lookup}
delle variabili per verificare se è possibile istanziare una funzione polimorfa in modo naturale e idiomatico per \texttt{Java}.


Tale variabile (\texttt{currentLevel}) è stata introdotta nei precedenti estratti di codice del \texttt{JavaTranspiler},
ma se ne fa maggior uso nel caso di funzioni polimorfe per specificare i parametri di tipo quando possibile.

\noindent Visitando un nodo \texttt{Variable} dell'albero sintattico astratto si incontrano tre casi:
\begin{enumerate}
    \item funzione monomorfa o parametro lambda (tutte le variabili introdotte da una lambda astrazione
          possiedono un monotipo per via del polimorfismo di rango 1);
    \item funzione polimorfa proveniente da uno \textit{scope} compreso tra quello globale e il livello di annidamento corrente;
          opzione rappresentante uno dei casi limite già menzionati, causato dall'impossibilità di fare riferimento
          esplicito ad un membro di una classe anonima esterna; la soluzione è utilizzare il metodo definito in \texttt{JavaPrelude}
          per istanziare la funzione parametrica tramite un \textit{cast};
    \item funzione polimorfa proveniente dal livello di annidamento attivo, livello globale
          (\textit{scope} con stesso nome del modulo) o libreria standard; in questa alternativa è possibile unificare un'istanza
          dello schema di tipo con il tipo della variabile nell'espressione considerata e quindi parametrizzare la chiamata.
\end{enumerate}

\newpage

\begin{lstlisting}[caption={Metodo \texttt{visit} per le variabili}, style=javaCode, label={lst:5-14-visit-variable-java}]
@Override
public Void visitVariable(Expression.Variable variable) {
    // firstly, find the variable in the scopes
    int i;
    Utils.Tuple@@<Scheme, String@@> variableScheme = null;
    for (i = 0; i @< scopes.size(); i++)
        if ((variableScheme = scopes.get(i).get(variable.id)) != null)
            break;

    // cannot have a null tuple, since type inference would have failed before this
    if (Objects.requireNonNull(variableScheme).fst().variables.isEmpty())
        // 1 -> lambda param or monomorphic declaration
        builder.append(variable.id);

    else if (variableScheme.snd().equals("this") && i @> 0 && i @< currentLevel)
        // 2 -> polymorphic declaration from an intermediate let scope 
        // needs the worst: an unchecked cast
        builder.append(JavaPrelude.class.getSimpleName()).append(".@<")
            .append(typeStringOf(variable.type())).append("@>_instantiationCast(")
            .append(variable.id)
            .append("())");

    else {
        // 3 -> polymorphic declaration from Prelude,
        // top level or same let scope can properly use generics
        builder.append(variableScheme.snd()).append(".@<");
        try {
            // to do so we need to instantiate the scheme and find the substitution
            Utils.Tuple@@<Substitution, Type@@> instantiation =
                variableScheme.fst().instantiate();

            Substitution subst = instantiation.fst()
                .applySubstitution(instantiation.snd().unify(variable.type()));

            // to then apply to the sorted variables
            builder.append(variableScheme.fst().sortedVariables.stream()
                    .map(ov @@-@> subst.variables().contains(ov)
                        @? typeStringOf(subst.substituteOf(ov))
                        @: Type.Variable.toString(ov))
                    .collect(Collectors.joining(", ")))
                .append("@>").append(variable.id).append("()");
        } catch (TypeException e) {
            // should never happen
            throw new InferenceException(e.getMessage());
        }
    }
    return null;
}
\end{lstlisting}
\vspace{4mm}
\begin{lstlisting}[caption={Metodo di istanziazione con \textit{cast}}, style=javaCode, label={lst:5-14-instantiation-cast-java}]
// Cast method for polymorphic functions instantiation
@SuppressWarnings("rawtypes, unchecked")
public static @@<T extends Function@@> T _instantiationCast(Function f) {
    return (T) f;
}
\end{lstlisting}

\newpage

\noindent Nei Codici~\ref{lst:5-14-nested-funx}~e~\ref{lst:5-14-nested-java} si può osservare come il procedimento
descritto produca variabili usate semplicemente, istanziate tramite \textit{cast} o con parametri di tipo.

\vspace{4mm}
\begin{lstlisting}[caption={\texttt{Let} annidati e differente uso di variabili polimorfe}, style=funxCode, label={lst:5-14-nested-funx}]
multipleIds x = let
        id1 = id
    in let
            id2 = id1
        in id2 x
\end{lstlisting}
\vspace{4mm}
\begin{lstlisting}[caption={Corrispondente traduzione in \texttt{Java}}, style=javaCode, label={lst:5-14-nested-java}]
public class Chapter5Nested {
    private Chapter5Nested() {
        // private constructor to prevent instantiation
    }
    
    public static @<h@> Function@<h, h@> multipleIds() {
        return (x @->
            (new Let@<@>() {
                private @<d@> Function@<d, d@> id1() {
                    return FunxPrelude.@<d@>id(); // 3
                }
    
                @Override
                public h _eval() {
                    return (new Let@<@>() {
                        private @<f@> Function@<f, f@> id2() {
                            return JavaPrelude
                                .@<Function@<f, f@>@>_instantiationCast(id1()); // 2
                        }
    
                        @Override
                        public h _eval() {
                            return this.@<h@>id2().apply(x); // 3 and 1
                        }
                    })._eval();
                }
            })._eval());
    }
}
\end{lstlisting}