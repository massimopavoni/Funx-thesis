\section{ANTLR}
\label{sec:5-1-antlr}

Al fine di semplificare lo sviluppo di \textit{lexer} e \textit{parser} per il linguaggio funzionale ideato
è stato scelto il generatore di \textit{parser} chiamato \texttt{ANTLR}%
\footnote{ANother Tool for Language Recognition (\url{https://www.antlr.org})}
\cite{Parr-1995-ANTLRGenerator,Parr-2013-DefinitiveANTLR}.

\noindent Grazie a tale strumento il processo iterativo di creazione della grammatica di \textbf{Funx}
è stato notevolmente semplificato e accelerato, in quanto \texttt{ANTLR} mette a disposizione del programmatore
un linguaggio per definire uno o più file di specifica per lessico e sintassi
(directory \texttt{Funx-jt/src/main/antlr} nel repository): questi vengono poi processati
per generare il codice sorgente del \textit{lexer} e del \textit{parser}.

\subsection{Analisi lessicale}
\label{sec:5-2-lexical-analysis}

Data la probabile complessità delle regole della grammatica di \textbf{Funx}, fin dall'inizio la definizione dei \textit{token} (lessemi)
del linguaggio è stata separata dalla specifica del \textit{parser}.

\noindent Il file \texttt{FunxLexer.g4} descrive i lessemi dividendoli nelle seguenti cateogorie:
\begin{enumerate}
    \item \textit{whitespace}: caratteri di spaziatura e tabulazione;
    \item \textit{comments}: commenti di linea e blocco;
    \item \textit{keywords}: parole chiave del linguaggio;
    \item \textit{Java keywords}: parole chiave del linguaggio Java, da evitare;
    \item \textit{types}: tipi di dato (funzioni di tipo con arità 0);
    \item \textit{literals}: costanti booleane e numeriche;
    \item \textit{variables}: identificatori per variabili di tipo o nomi di funzioni;
    \item \textit{module}: identificatori il modulo;
    \item vari operatori simbolici per:
          \begin{itemize}
              \item \textit{bool}: valori booleani;
              \item \textit{comparison}: confronti tra numeri;
              \item \textit{arithmetic}: operazioni aritmetiche;
              \item \textit{other symbols}: simboli della sintassi (come \texttt{->}) e varie funzioni di libreria;
              \item \textit{delimiters}: parentesi tonde, quadre e graffe.
          \end{itemize}
\end{enumerate}

\noindent Le categorie 1 e 2 contengono token da scartare, tranne \texttt{NEWLINE}, mentre la categoria 4 è utile
qualora eventualmente si permetta allo sviluppatore di utilizzare tali parole chiave riservate,
effettuando una rinomina automatica; le categorie 7 e 8 devono necessariamente apparire dopo le categorie 3 e 4,
poiché tra \textit{keyword} e identificatori di ogni genere le prime devono avere la precedenza
(la posizione della categoria 5 tiene conto di una possibile futura estensione per consentire la creazione di nuovi tipi).


Oltre alle categorie illustrate, in testa al file sono presenti dei cosiddetti \textit{fragment} (frammenti)
che semplificano le espressioni regolari dei \textit{token} e complessivamente aumentano la leggibilità della specifica.

\newpage

\begin{lstlisting}[caption={Alcune \textit{token} del \textit{lexer}}, style=antlrCode, label={lst:5-2-lexer-antlr}]
lexer grammar FunxLexer;

// Fragments
fragment LALPHA: [a-z];
fragment UALPHA: [A-Z];
fragment ALPHA: LALPHA | UALPHA;
fragment ALPHA_: ALPHA | UnderScore;

fragment DIGIT: [0-9];
fragment DECIMAL: DIGIT+;

// Whitespace
NEWLINE: '\r'? '\n' | '\r';

TAB: [\t]+ @-> skip;
WS: [\u0020\u00a0\u1680\u2000\u200a\u202f\u205f\u3000]+ @-> skip;

// Comments
CloseMultiComment: '\./';
OpenMultiComment: '/\.';
SingleComment: '//';

COMMENT: SingleComment ~[\r\n]* @-> skip;
MULTICOMMENT: OpenMultiComment .*? CloseMultiComment @-> skip;

// Keywords
ELSE: 'else';
FI: 'fi';
IF: 'if';
IN: 'in';
LET: 'let';

// Java keywords
RESERVED_JAVA_KEYWORD: 'abstract' | 'assert' | 'boolean' | 'break' | 'byte' | [\.\.\.];

// Types
TYPE: BOOLTYPE | INTTYPE;
BOOLTYPE: 'Bool';

// Literals
INT: DECIMAL | OpenParen '\-' DECIMAL CloseParen;

// Variables
VARID: LALPHA (ALPHA_ | DIGIT)*;

// Module
MODULEID: UALPHA (ALPHA_ | DIGIT)*;

// Bool
And: '&&';
Not: '!!';

// Comparison
EqualsEquals: '\=\=';
NotEquals: '!\=';

// Arithmetic
Add: '\+';

// Other symbols
UnderScore: '_';
Arrow: '\-\>';

// Delimiters
OpenParen: '\(';
CloseParen: '\)';
\end{lstlisting}

\subsection{Analisi sintattica}
\label{sec:5-3-syntactic-analysis}

Il file \texttt{FunxParser.g4} contiene le regole concrete della grammatica di \textbf{Funx}:
nonostante la somiglianza con le grammatiche delle Figure~\ref{fig:2-funx-syntax} e \ref{fig:3-system-hm},
è evidente che queste non collimino esattamente a causa di zucchero sintattico e requisiti di \texttt{ANTLR}.

\noindent Lo strumento utilizzato, infatti, è un generatore di \textit{parser} di tipo \textit{top-down}
per grammatiche \textit{LL}, le quali in generale non supportano regole ricorsive a sinistra.

\noindent Essendo tali regole spesso comuni nella definizione di qualsiasi linguaggio di programmazione,
\textbf{Funx} incluso, \texttt{ANTLRv4} offre un diverso tipo di parsing, detto \textit{Adaptive LL(*)}%
\footnote{\citetitle{Parr-2011-FoundationANTLR}, \cite{Parr-2011-FoundationANTLR}
    and \citetitle{Parr-2014-AdaptiveLL}, \cite{Parr-2014-AdaptiveLL}}:
quest'ultimo è in grado di riscrivere automaticamente le grammatiche, eliminando la ricorsione a sinistra diretta (e.g. linee 36 e 38-43),
così da non incorrere in regole ambigue che potrebbero causare \textit{backtracking} e conseguente \textit{overhead}.

\noindent Il Codice~\ref{lst:5-parser} riporta integralmente le regole concrete della sintassi di \textbf{Funx}, tra cui:
\begin{itemize}
    \item \textit{module}: nome del modulo, funzione \texttt{main} opzionale e dichiarazioni globali;
    \item \textit{main}: funzione \texttt{main}, diversa dalle dichiarazioni classiche per l'assenza di schema di tipo e parametri lambda;
    \item \textit{declaration}: funzione con nome, tipo e parametri (e opzionalmente \texttt{with} per funzioni locali);
    \item \textit{typeElems}: tipo di una funzione, definito ricorsivamente secondo la grammatica del sistema di tipo di \textbf{Funx};
    \item \textit{statement}: per evitare ricorsione a sinistra indiretta, la separazione tra \textit{statement} e \textit{expression}
          forza l'uso di parentesi nei casi in cui lambda astrazioni, let e if siano usati all'interno di un'espressione;
    \item \textit{expression}: racchiude l'applicazione funzionale, tutte le regole relative agli operatori simbolici,
          specificandone la priorità implicite (Tabella~\ref{tab:2-sugar}), e le espressioni primarie
          (costanti, variabili e parentesi per controllare la precedenza);
    \item \textit{lambda, let, it}: corrispondenti alle produzioni per astrazione, let e if della grammatica formale.
\end{itemize}

\newpage

\begin{lstlisting}[caption={Grammatica per il \textit{parser}}, style=antlrCode, label={lst:5-parser}]
parser grammar FunxParser;
options { tokenVocab = FunxLexer; }

// Module
module: (MODULE MODULEID (Dot MODULEID)* NEWLINE+)?
    (main NEWLINE+)? declarations EOF;

declarations: declaration (NEWLINE declaration?)*;

main: id = MAIN Equals statement with?;

// Declaration
declaration: (declarationScheme NEWLINE)?
    id = VARID lambdaParams? Equals statement with?;

declarationScheme: id = VARID Colon typeElems;

with: NEWLINE WITH localDeclarations OUT;

localDeclarations: NEWLINE declarations NEWLINE;

// Type
typeElems: OpenParen typeElems CloseParen %# parenType%
    | VARID %# typeVar%
    | TYPE %# namedType%
    | <assoc = right> typeElems Arrow typeElems %# arrowType%;

// Statement
statement: expression %# expressionStatement%
    | lambda %# lambdaStatement%
    | let %# letStatement%
    | ifS %# ifStatement%;

// Expression
expression: primary %# primExpression%
    | expression expression %# appExpression%
    | <assoc = right> expression bop = Dot expression %# composeExpression%
    | expression bop = (Divide | Modulo | Multiply) expression %# divModMultExpression%
    | expression bop = (Add | Subtract) expression %# addSubExpression%
    | expression
        bop = (GreaterThan | GreaterThanEquals | LessThan | LessThanEquals)
        expression %# compExpression%
    | expression bop = (EqualsEquals | NotEquals) expression %# eqExpression%
    | uop = Not expression %# notExpression%
    | <assoc = right> expression bop = And expression %# andExpression%
    | <assoc = right> expression bop = Or expression %# orExpression%
    | <assoc = right> expression bop = Dollar expression %# rightAppExpression%;

primary: OpenParen statement CloseParen %# parenPrimary%
    | constant %# constPrimary% | VARID %# varPrimary%;

// Lambda
lambda: Backslash lambdaParams? Arrow statement;

lambdaParams: VARID+;

// Let
let: LET localDeclarations IN statement;

// If
ifS: IF statement THEN statement ELSE statement FI;

// Constant
constant: BOOL | numConstant;

numConstant: INT;    
\end{lstlisting}