\subsection{AST builder}
\label{sec:5-6-ast-builder}

\texttt{ANTLR} offre due diversi approcci per analizzare l'albero di parsing:
\begin{itemize}
    \item \textit{listener}: i metodi implementati per l'analisi dei nodi sono chiamati seguendo la ricerca
          in profondità (\textit{DFS, depth-first search});
    \item \textit{visitor}: i nodi possono essere visitati arbitrariamente tramite il metodo \texttt{visit},
          senza seguire un ordine specifico; non tutti i metodi devono essere implementati
          ed è possibile ignorare dei nodi o ri-visitarli.
\end{itemize}

\noindent La seconda opzione è indubbiamente più versatile, e può essere per esempio anche utilizzata
qualora si desideri evitare completamente la costruzione di un \textbf{AST} e utilizzare i metodi di visita
per eseguire il codice (e.g. linguaggi interpretati).

\noindent Nel caso di \textbf{Funx-jt} la scelta ricade appunto sul \textit{visitor} in virtù
della probabilità di dover "saltare" alcuni nodi e ottenerne direttamente le informazioni interne.

\noindent Durante la compilazione del progetto \texttt{Java}, \texttt{Gradle} è configurato per generare
le seguenti classi nel package \texttt{com.github.massimopavoni.funx.jt.parser}:
\begin{itemize}
    \item \texttt{FunxLexer}: \textit{lexer};
    \item \texttt{FunxParser}: \textit{parser};
    \item \texttt{FunxParserVisitor}: interfaccia generica per differenti implementazioni del \textit{visitor}
          (estende l'interfaccia \texttt{ParseTreeVisitor} di \texttt{ANTLR});
    \item \texttt{FunxParserBaseVisitor}: classe predefinita che percorre l'intero albero di parsing
          (estende la classe astratta \texttt{AbstractParseTreeVisitor} per ereditare alcuni metodi standard come \texttt{visitChildren}).
\end{itemize}

\noindent La classe \texttt{ASTBuilder} estende a sua volta \texttt{FunxParserBaseVisitor} e sovrascrive quasi tutti i metodi ereditati;
in particolare, \texttt{ASTBuilder} effettua la composizione degli schemi di tipo definiti dall'utente (sezione~\ref{sec:5-8-system-hm})
e l'eliminazione dello zucchero sintattico mediante l'uso di alcune proprietà e metodi ausiliari.


Inoltre, all'interno del package \texttt{com.github.massimopavoni.funx.jt.ast} viene definita una classe enumeratore,
\texttt{PreludeFunction} contenente le funzioni di libreria standard, con i relativi simboli e gli schemi di tipo corrispondenti.

\vspace{4mm}
\begin{lstlisting}[caption={Parte del codice di \texttt{PreludeFunction}}, style=javaCode, label={lst:5-preludefunction-java}]
public enum PreludeFunction {
    COMPOSE(".", "compose",
        new Scheme(Set.of(0L, 1L, 2L),
            arrowOf(arrowOf(ZERO, ONE), arrowOf(TWO, ZERO), TWO, ONE)), false);

    public final String symbol;
    public final String id;
    public final Scheme scheme;
    public final boolean nativeJava;

    PreludeFunction(
        String symbol, String id, Scheme scheme, boolean nativeJava) { ... }

    public static PreludeFunction fromSymbol(String symbol) {
        return Utils.enumFromField(PreludeFunction.class,
            f @-> f.symbol.equals(symbol));
    }
}
\end{lstlisting}

\newpage

\begin{lstlisting}[caption={Metodi per astrazioni annidate e operatori simbolici binari}, style=javaCode, label={lst:5-auxiliary-methods-java}]
private ASTNode createLambdaChain(
        InputPosition position, Deque@<String@> params, ASTNode expression) {
    if (params.size() == 1)
        return new Expression.Lambda(
                position,
                params.getFirst(),
                expression);
    return new Expression.Lambda(
            position,
            params.pop(),
            createLambdaChain(position, params, expression));
}

private ASTNode binarySymbolApplication(
        InputPosition position, String symbol, ASTNode left, ASTNode right) {
    return new Expression.Application(
            position,
            new Expression.Application(
                    position,
                    new Expression.Variable(
                            position,
                            PreludeFunction.fromSymbol(symbol).id),
                    left),
            right);
}
\end{lstlisting}
\vspace{4mm}
\begin{lstlisting}[caption={Alcuni metodi \textit{visit} di \texttt{ASTBuilder}}, style=javaCode, label={lst:5-astbuilder-java}]
public class ASTBuilder extends FunxParserBaseVisitor@<ASTNode@> {
    @Override
    public ASTNode visitAppExpression(FunxParser.AppExpressionContext ctx) {
        return new Expression.Application(getInputPosition(ctx),
            visit(ctx.expression(0)), visit(ctx.expression(1)));
    }

    @Override
    public ASTNode visitComposeExpression(FunxParser.ComposeExpressionContext ctx) {
        return binarySymbolApplication(getInputPosition(ctx),
            Utils.fromLexerToken(ctx.bop.getType()),
            visit(ctx.expression(0)), visit(ctx.expression(1)));
    }

    @Override
    public ASTNode visitAndExpression(FunxParser.AndExpressionContext ctx) {
        // transform logical conjunction into if statement for lazy behavior
        return new Expression.If(getInputPosition(ctx),
                visit(ctx.expression(0)), visit(ctx.expression(1)),
                new Expression.Constant(InputPosition.UNKNOWN, false));
    }

    @Override
    public ASTNode visitLambda(FunxParser.LambdaContext ctx) {
        return createLambdaChain(getInputPosition(ctx),
            // lambda params are syntactic sugar for a lambda chain
            ctx.lambdaParams().VARID().stream().map(ParseTree::getText)
                .collect(Collectors.toCollection(ArrayDeque::new)),
                visit(ctx.statement()));
    }
}    
\end{lstlisting}