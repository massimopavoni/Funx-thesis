\section{Traduzione in Java}
\label{sec:5-10-java-translation}

La sottoclasse più importante di \texttt{ASTVisitor} è \texttt{JavaTranspiler}, il \textit{visitor} che si occupa
dell'ultimo stadio di compilazione, la traduzione dell'\textbf{AST} in codice \texttt{Java}:
come per \texttt{GraphvizBuilder}, la classe compone una stringa
che rappresenta il programma \texttt{Java} corrispondente al codice \textbf{Funx} sorgente.


Avendo già illustrato alcuni esempi di traduzione nel Capitolo~\ref{chap:4-java}, in questa sezione si
discuteranno le scelte e i compromessi nel processo di traduzione, e il modo in cui le limitazioni
di \texttt{Java} possano essere talvolta aggirate "piegando" le regole.

\subsection{Membri statici}
\label{sec:5-11-static-members}

Il paradigma dichiarativo dei linguaggi funzionali è ben diverso dalla programmazione ad oggetti di molti altri linguaggi rinomati,
motivo per cui la scelta di tradurre ogni programma \textbf{Funx} in un'unica classe statica è vista come semplice soluzione
per evitare complicanze e \textit{overhead} per la creazione di oggetti in aggiunta alle \texttt{Function}.


Ogni funzione definita diviene perciò una proprietà statica della classe in caso di monotipi (sezione~\ref{sec:5-13-monomorphic-declarations})
o un metodo statico con parametri di tipo in caso di politipi (sezione~\ref{sec:5-14-polymorphic-functions-instantiation}).
Fanno eccezione le funzioni appartenenti a espressioni \texttt{let} annidate: essendo classi anonime, queste creano
un unico oggetto con proprietà e metodi privati (accessibili solamente al metodo pubblico \texttt{eval} della classe \texttt{Let}).

\noindent La traduzione inizia con \textit{import} statici per la libreria standard e un costruttore privato.

\vspace{4mm}
\begin{lstlisting}[caption={Prime aggiunte alla stringa \texttt{Java}}, style=javaCode, label={lst:5-11-first-append-java}]
// append package, imports, class declaration and constructor
builder.append(module.packageName.isEmpty()
        @? ""
        @: String.format("package %s;%n", module.packageName.toLowerCase()))
    .append("\n\nimport ").append(Function.class.getName())
    .append(";\n\nimport ").append(JavaPrelude.class.getName())
    .append(";\n\nimport ").append(FunxPrelude.class.getName())
    .append(";\n\nimport static ").append(JavaPrelude.class.getName())
    .append(".*;\nimport static ").append(FunxPrelude.class.getName())
    .append(".*;\n\npublic class ").append(module.name).append(" {\n")
    .append("private ").append(module.name)
    .append("() {\n// private constructor to prevent instantiation\n}\n\n");
\end{lstlisting}
\vspace{4mm}
\begin{lstlisting}[caption={Corrispondente codice \texttt{Java} generato}, style=javaCode, label={lst:5-11-class-start-java}]
import java.util.function.Function;

import com.github.massimopavoni.funx.lib.JavaPrelude;

import com.github.massimopavoni.funx.lib.FunxPrelude;

import static com.github.massimopavoni.funx.lib.JavaPrelude.*;
import static com.github.massimopavoni.funx.lib.FunxPrelude.*;

public class Chapter5Header {
    private Chapter5Header() {
        // private constructor to prevent instantiation
    }

    // ...
}
\end{lstlisting}

\newpage

\subsection{Lista dei contesti}
\label{sec:5-12-scope-stack}

\subsection{Dichiarazioni monomorfe}
\label{sec:5-13-monomorphic-declarations}

La possibilità in \textbf{Funx} di utilizzare funzioni ricorsive (e/o mutuamente ricorsive)
e la volontà di evitare la traduzione in metodi quando possibile sono in conflitto
a causa di \textit{Illegal Self Reference} e \textit{Illegal Forward Reference}:
tali errori si presentano durante la compilazione del codice \texttt{Java}
qualora i campi statici che identificano funzioni monomorfe vengano dichiarati e inizializzati
nello stesso \textit{statement} (stessa linea).


La dichiarazione delle proprietà deve avvenire prima dell'inizializzazione di altre variabili che ne fanno uso;
si potrebbe effettuare un'analisi iniziale dell'\textbf{AST} per identificare le dipendenze tra le funzioni
(approccio di ordinamento topologico estremamente utile anche per l'inferenza), ma la soluzione adottata
è di più semplice implementazione.


Come si può notare nel Codice~\ref{lst:5-13-monomorphic-java} e in alcuni esempi già presentati in precedenza,
si effettua la dichiarazione di ogni campo, pubblico e statico per le dichiarazioni globali, privato per quelle locali,
e solo successivamente si inizializzano rispettivamente con blocco statico e metodo \texttt{eval}.

\vspace{4mm}
\begin{lstlisting}[caption={Esempio di traduzione per funzioni monomorfe}, style=javaCode, label={lst:5-13-monomorphic-java}]
public class Chapter5Monomorphic {
    private Chapter5Monomorphic() {
        // private constructor to prevent instantiation
    }
    
    public static void main(String[] args) {
        System.out.println(add.apply(add.apply(fun1).apply(fun2)).apply(letFun));
    }
    
    public static Long fun1;    
    public static Long fun2;    
    public static Long letFun;
    
    static {
        fun1 = 1L;    
        fun2 = 2L;    
        letFun =
            (new Let@<@>() {
                private Long a;    
                private Long b;
    
                @Override
                public Long _eval() {
                    a = 3L;
                    b = 4L;
                    return add.apply(a).apply(b);
                }
            })._eval();
    }
}
\end{lstlisting}

\newpage

\noindent Poiché potrebbero essere presenti diversi \texttt{let} annidati, è necessario tenere traccia delle espressioni
corpo delle dichiarazioni monomorfe in modo da poterle inizializzare al momento corretto, dopo aver tradotto ulteriori classi interne.

\noindent La procedura di traduzione di dichiarazioni monomorfe si compone delle seguenti fasi:
\begin{itemize}
    \item definizione di uno \textit{stack} contenente mappe tra nomi delle dichiarazioni e nodi espressione corrispondenti;
    \item inserimento di una nuova mappa per il livello corrente di annidamento (modulo o espressione \texttt{let});
    \item dichiarazione delle variabili e aggiunta delle espressioni monomorfe
          alla mappa corrente (potrebbero essere aggiunti nuovi livelli prima di poterne "riempire" uno);
    \item creazione del blocco statico (o metodo \texttt{eval} per le espressioni \texttt{let}) con l'inizializzazione
          delle funzioni monomorfe: in questa fase finale torna utile la versatilità del \textit{visitor pattern} per posticipare
          la traduzione delle espressioni.
\end{itemize}

\vspace{4mm}
\begin{lstlisting}[caption={Traduzione di funzioni monomorfe in \texttt{let}}, style=javaCode, label={lst:5-13-monomorphic-translation-java}]
private final Deque<Map<String, Expression>>
    monomorphicDeclarationsQueue = new ArrayDeque<>();

@Override
public Void visitLet(Expression.Let let) {
    currentLevel++;
    // ...
    // use a new anonymous class for the let expression
    // and push a new monomorphic let declarations map
    builder.append("(new ")
            .append(JavaPrelude.Let.class.getSimpleName()).append("<>() {\n");
    monomorphicDeclarationsQueue.push(new LinkedHashMap<>());
    visit(let.localDeclarations);
    builder.append("""
                    @Override
                    public\s""")
            .append(typeStringOf(let.expression.type()))
            .append("\n_eval() {\n");
    // if there are any monomorphic declarations, initialize them in the _eval method,
    // then pop the map either way
    if (!monomorphicDeclarationsQueue.getFirst().isEmpty())
        monomorphicDeclarationsQueue.getFirst().forEach((id, expression) @-> {
            builder.append(id).append(" = ");
            visit(expression); // deferred expression visit
            appendSemiColon();
            appendNewline();
        });
    monomorphicDeclarationsQueue.pop();
    // ...
    currentLevel--;
    return null;
}
\end{lstlisting}

\newpage

\begin{lstlisting}[caption={Metodo \texttt{visit} per le dichiarazioni}, style=javaCode, label={lst:5-13-visit-declaration-java}]
@Override
public Void visitDeclaration(Declaration declaration) {
    // top level declarations should be static and public,
    // while let local declarations should be private to the anonymous class
    builder.append(currentLevel == 0 @? "public static " @: "private ");
    String scheme = schemeStringOf(declaration.scheme());
    if (declaration.scheme().variables.isEmpty()) {
        // defer initialization of monomorphic declarations
        builder.append(scheme).append(" ").append(declaration.id);
        appendSemiColon();
        monomorphicDeclarationsQueue
            .getFirst().put(declaration.id, declaration.expression);
    } else {
        // initialize polymorphic declarations immediately (as methods with generics)
        builder.append(scheme)
                .append(" ").append(declaration.id).append("() {\nreturn ");
        visit(declaration.expression);
        appendSemiColon();
        appendCloseBrace();
    }
    appendNewline();
    return null;
}
\end{lstlisting}

\subsection{Instanziazione di funzioni polimorfe}
\label{sec:5-14-polymorphic-functions-instantiation}

\subsection{Type casting "selvaggio"}
\label{sec:5-15-wild-type-casting}
