\section{Traduzione in Java}
\label{sec:5-10-java-translation}

La sottoclasse più importante di \texttt{ASTVisitor} è \texttt{JavaTranspiler}, il \textit{visitor} che si occupa
dell'ultimo stadio di compilazione, la traduzione dell'\textbf{AST} in codice \texttt{Java}:
come per \texttt{GraphvizBuilder}, la classe compone una stringa
che rappresenta il programma \texttt{Java} corrispondente al codice \textbf{Funx} sorgente.


Avendo già illustrato alcuni esempi di traduzione nel Capitolo~\ref{chap:4-java}, in questa sezione si
discuteranno le scelte e i compromessi nel processo di traduzione, e il modo in cui le limitazioni
di \texttt{Java} possano essere talvolta aggirate "piegando" le regole.

\subsection{Membri statici}
\label{sec:5-11-static-members}

Il paradigma dichiarativo dei linguaggi funzionali è ben diverso dalla programmazione ad oggetti di molti altri linguaggi rinomati,
motivo per cui la scelta di tradurre ogni programma \textbf{Funx} in un'unica classe statica è vista come semplice soluzione
per evitare complicanze e \textit{overhead} per la creazione di oggetti in aggiunta alle \texttt{Function}.


Ogni funzione definita diviene perciò una proprietà statica della classe in caso di monotipi (sezione~\ref{sec:5-13-monomorphic-declarations})
o un metodo statico con parametri di tipo in caso di politipi (sezione~\ref{sec:5-14-polymorphic-functions-instantiation}),
mentre scrittura della classe stessa inizia con \textit{import} statici delle funzioni di libreria standard e un costruttore privato.

\vspace{4mm}
\begin{lstlisting}[caption={Prime aggiunte alla stringa \texttt{Java}}, style=javaCode, label={lst:5-11-first-append-java}]
// append package, imports, class declaration and constructor

builder.append(module.packageName.isEmpty()
        @? ""
        @: String.format("package %s;%n", module.packageName.toLowerCase()))
    .append("\n\nimport ").append(Function.class.getName())
    .append(";\n\nimport ").append(JavaPrelude.class.getName())
    .append(";\n\nimport ").append(FunxPrelude.class.getName())
    .append(";\n\nimport static ").append(JavaPrelude.class.getName())
    .append(".*;\nimport static ").append(FunxPrelude.class.getName())
    .append(".*;\n\npublic class ").append(module.name).append(" {\n")
    .append("private ").append(module.name)
    .append("() {\n// private constructor to prevent instantiation\n}\n\n");
\end{lstlisting}
\vspace{4mm}
\begin{lstlisting}[caption={Corrispondente codice \texttt{Java} generato}, style=javaCode, label={lst:5-11-class-start-java}]
import java.util.function.Function;

import com.github.massimopavoni.funx.lib.JavaPrelude;

import com.github.massimopavoni.funx.lib.FunxPrelude;

import static com.github.massimopavoni.funx.lib.JavaPrelude.*;
import static com.github.massimopavoni.funx.lib.FunxPrelude.*;

public class Test {
    private Test() {
        // private constructor to prevent instantiation
    }    
\end{lstlisting}

\newpage

\subsection{Stack dei contesti}
\label{sec:5-12-scope-stack}

Durante l'esplorazione dell'\textbf{AST} è necessario tenere traccia dello \textit{scope}, il contesto, del quale le dichiarazioni
o i parametri lambda fanno parte: la soluzione adottata prevede l'uso di uno \textit{stack} contenente i contesti attivi,
rappresentati a loro volta da mappe (le variabili dichiarate sono le chiavi, le relative informazioni sono i valori).

\noindent Lo \textit{stack} dei contesti è in realtà implementato attraverso una lista, per via della necessità di iterare
su tutti i contesti e trovare una variabile richiesta (\textit{lookup}).

\vspace{4mm}
\begin{lstlisting}[caption={Struttura e metodi per lo \textit{scope}}, style=javaCode, label={lst:5-12-scope-stack-methods-java}]
private final List@<Map@<String, Utils.Tuple@<Scheme, String@>@>@>
    scopes = new ArrayList@<@>();

private void addToScope(List@<Declaration@> declarations, String scope) {
    scopes.addFirst(declarations.stream()
        .collect(ImmutableMap.toImmutableMap(
            decl @-> decl.id,
            decl @-> new Utils.Tuple@<@>(decl.scheme(), scope))));
}

private void addToScope(String id, Scheme scheme, String scope) {
    scopes.addFirst(Collections.singletonMap(id, new Utils.Tuple@<@>(scheme, scope)));
}
\end{lstlisting}
\vspace{4mm}
\begin{lstlisting}[caption={Aggiunta di contesto per modulo, lambda astrazioni e \texttt{let}}, style=javaCode, label={lst:5-12-lambda-let-scope-java}]
@Override
public Void visitModule(ASTNode.Module module) {
    // add Prelude functions to the scope
    scopes.addFirst(Arrays.stream(PreludeFunction.values())
        .collect(ImmutableMap.toImmutableMap(
            pf @-> pf.id,
            pf @-> new Utils.Tuple@<@>(pf.scheme, pf.nativeJava
                @? JavaPrelude.class.getSimpleName()
                @: FunxPrelude.class.getSimpleName()))));
    // add module functions to the scope
    addToScope(module.let.localDeclarations.declarationList, module.name);
    // ...
    scopes.removeFirst();
    return null;
}

@Override
public Void visitLambda(Expression.Lambda lambda) {
    // add lambda parameter to the scope
    addToScope(lambda.paramId, new Scheme(Collections.emptySet(),
        ((Type.FunctionApplication) lambda.type()).arguments.getFirst()), null);
    // ...
    scopes.removeFirst();
    return null;
}

@Override
public Void visitLet(Expression.Let let) {
    currentLevel++;
    // add let local declarations to the scope
    addToScope(let.localDeclarations.declarationList, "this");
    // ...
    // restore previous scope state and level
    scopes.removeFirst();
    currentLevel--;
    return null;
}
\end{lstlisting}

\newpage

\noindent Nei Codici~\ref{lst:5-12-scope-stack-methods-java}~e~\ref{lst:5-12-lambda-let-scope-java} sono mostrate:
\begin{itemize}
    \item la lista dei contesti e le funzioni per la loro gestione: le tuple assegnate ai nomi delle dichiarazioni di uno \textit{scope}
          conservano lo schema di tipo e il nome della classe in cui la variabile è definita
          (\texttt{JavaPrelude}, \texttt{FunxPrelude}, nome del modulo creato,
          \texttt{null} per parametri lambda e \texttt{this} per i \texttt{let});
    \item l'aggiunta dei contesti per \texttt{Module}, \texttt{Lambda} e \texttt{Let}, rispettivamente:
          funzioni di libreria e dichiarazioni del modulo stesso, parametro lambda,
          dichiarazioni locali con l'incremento del livello di annidamento.
\end{itemize}

\subsection{Dichiarazioni monomorfe}
\label{sec:5-13-monomorphic-declarations}

La possibilità in \textbf{Funx} di utilizzare funzioni ricorsive e mutuamente ricorsive
e la volontà di evitare la traduzione di ogni dichiarazione in un metodo sono in conflitto
a causa di \textit{Illegal Self Reference} e \textit{Illegal Forward Reference}:
tali errori si presentano durante la compilazione del codice \texttt{Java}
qualora i campi statici che identificano funzioni monomorfe vengano dichiarati e inizializzati
nello stesso \textit{statement} (stessa linea).


La dichiarazione delle proprietà deve avvenire prima di poter apparire nell'inizializzazione di altre variabili;
si potrebbe effettuare un'analisi iniziale dell'\textbf{AST} per identificare le dipendenze tra le funzioni
(approccio di ordinamento topologico estremamente utile anche per l'inferenza), ma la soluzione adottata
è di più semplice implementazione.


Come si può notare nel Codice~\ref{lst:5-13-monomorphic-java} e in alcuni esempi già presentati in precedenza,
si effettua la dichiarazione di ogni campo, pubblico e statico per le dichiarazioni globali, privato per quelle locali,
e solo successivamente si inizializzano rispettivamente con blocco statico e metodo \texttt{eval}.

\vspace{4mm}
\begin{lstlisting}[caption={Traduzione di funzioni monomorfe}, style=javaCode, label={lst:5-13-monomorphic-java}]
public class Chapter5Monomorphic {
    private Chapter5Monomorphic() {
        // private constructor to prevent instantiation
    }
    
    public static void main(String[] args) {
        System.out.println(add.apply(add.apply(fun1).apply(fun2)).apply(letFun));
    }
    
    public static Long fun1;    
    public static Long fun2;    
    public static Long letFun;
    
    static {
        fun1 = 1L;    
        fun2 = 2L;    
        letFun =
            (new Let@<@>() {
                private Long a;    
                private Long b;
    
                @Override
                public Long _eval() {
                    a = 3L;
                    b = 4L;
                    return add.apply(a).apply(b);
                }
            })._eval();
    }
}
\end{lstlisting}

\newpage

\noindent Poiché potrebbero essere presenti diversi \texttt{let} annidati, è necessario tenere traccia delle espressioni
corpo delle dichiarazioni monomorfe in modo da poterle inizializzare al momento corretto, dopo aver tradotto ulteriori classi interne.

\noindent La procedura di traduzione di dichiarazioni monomorfe si compone delle seguenti fasi:
\begin{itemize}
    \item definizione di uno \textit{stack} contenente mappe tra nomi delle dichiarazioni e nodi espressione corrispondenti;
    \item inserimento di una nuova mappa per il livello corrente di annidamento (modulo o espressione \texttt{let});
    \item inferenza delle funzioni globali o locali con dichiarazione delle variabili e aggiunta delle espressioni monomorfe
          alla mappa corrente (potrebbero essere aggiunti nuovi livelli prima di poterne "riempire" uno);
    \item creazione del blocco statico (o metodo \texttt{eval} per le espressioni \texttt{let}) con le inizializzazioni
          delle funzioni con monotipo: in questa fase finale torna utile la versatilità del \textit{visitor pattern} per posticipare
          la traduzione delle espressioni.
\end{itemize}

\vspace{4mm}
\begin{lstlisting}[caption={Traduzione di funzioni monomorfe in \texttt{let}}, style=javaCode, label={lst:5-13-monomorphic-translation-java}]
private final Deque<Map<String, Expression>>
    monomorphicDeclarationsQueue = new ArrayDeque<>();

@Override
public Void visitLet(Expression.Let let) {
    currentLevel++;
    // ...
    // use a new anonymous class for the let expression
    // and push a new monomorphic let declarations map
    builder.append("(new ")
            .append(JavaPrelude.Let.class.getSimpleName()).append("<>() {\n");
    monomorphicDeclarationsQueue.push(new LinkedHashMap<>());
    visit(let.localDeclarations);
    builder.append("""
                    @Override
                    public\s""")
            .append(typeStringOf(let.expression.type()))
            .append("\n_eval() {\n");
    // if there are any monomorphic declarations, initialize them in the _eval method,
    // then pop the map either way
    if (!monomorphicDeclarationsQueue.getFirst().isEmpty())
        monomorphicDeclarationsQueue.getFirst().forEach((id, expression) @-> {
            builder.append(id).append(" = ");
            visit(expression); // deferred expression visit
            appendSemiColon();
            appendNewline();
        });
    monomorphicDeclarationsQueue.pop();
    // ...
    currentLevel--;
    return null;
}
\end{lstlisting}

\newpage

\begin{lstlisting}[caption={Metodo \texttt{visit} per le dichiarazioni}, style=javaCode, label={lst:5-13-visit-declaration-java}]
@Override
public Void visitDeclaration(Declaration declaration) {
    // top level declarations should be static and public,
    // while let local declarations should be private to the anonymous class
    builder.append(currentLevel == 0 @? "public static " @: "private ");
    String scheme = schemeStringOf(declaration.scheme());
    if (declaration.scheme().variables.isEmpty()) {
        // defer initialization of monomorphic declarations
        builder.append(scheme).append(" ").append(declaration.id);
        appendSemiColon();
        monomorphicDeclarationsQueue
            .getFirst().put(declaration.id, declaration.expression);
    } else {
        // initialize polymorphic declarations immediately (as methods with generics)
        builder.append(scheme)
                .append(" ").append(declaration.id).append("() {\nreturn ");
        visit(declaration.expression);
        appendSemiColon();
        appendCloseBrace();
    }
    appendNewline();
    return null;
}
\end{lstlisting}

\subsection{Istanziazione di funzioni polimorfe}
\label{sec:5-14-polymorphic-functions-instantiation}

Dal momento che il contesto può cambiare con l'introduzione di parametri lambda oltre che con espressioni \texttt{let},
il livello di annidamento di quest'ultime è segnalato da una variabile secondaria.
Quest'ultima è utilizzata durante il \textit{lookup}
delle variabili per verificare se è possibile istanziare una funzione polimorfa in modo idiomatico per \texttt{Java}.


Tale variabile (\texttt{currentLevel}) è stata introdotta nei precedenti estratti di codice del \texttt{JavaTranspiler},
ma se ne fa maggior uso nel caso di funzioni polimorfe per specificare i parametri di tipo quando possibile.

\noindent Visitando un nodo \texttt{Variable} dell'albero sintattico astratto si incontrano tre casi:
\begin{enumerate}
    \item funzione monomorfa o parametro lambda (tutte le variabili introdotte da una lambda astrazione
          possiedono un monotipo per via del polimorfismo di rango 1);
    \item funzione polimorfa proveniente da uno \textit{scope} compreso tra quello globale e il livello di annidamento corrente;
          opzione rappresentante uno dei casi limite già menzionati, causato dall'impossibilità di fare riferimento
          esplicito a un membro di una classe anonima esterna; la soluzione è utilizzare il metodo definito in \texttt{JavaPrelude}
          per istanziare la funzione parametrica tramite un \textit{cast};
    \item funzione polimorfa proveniente dal livello di annidamento attivo, livello globale
          (\textit{scope} con stesso nome del modulo) o libreria standard; in questa alternativa è possibile unificare un'istanza
          dello schema di tipo con il tipo della variabile nell'espressione considerata e quindi parametrizzare la chiamata.
\end{enumerate}

\newpage

\begin{lstlisting}[caption={Metodo \texttt{visit} per le variabili}, style=javaCode, label={lst:5-14-visit-variable-java}]
@Override
public Void visitVariable(Expression.Variable variable) {
    // firstly, find the variable in the scopes
    int i;
    Utils.Tuple@@<Scheme, String@@> variableScheme = null;
    for (i = 0; i @< scopes.size(); i++)
        if ((variableScheme = scopes.get(i).get(variable.id)) != null)
            break;

    // cannot have a null tuple, since type inference would have failed before this
    if (Objects.requireNonNull(variableScheme).fst().variables.isEmpty())
        // 1 -> lambda param or monomorphic declaration
        builder.append(variable.id);

    else if (variableScheme.snd().equals("this") && i @> 0 && i @< currentLevel)
        // 2 -> polymorphic declaration from an intermediate let scope 
        // needs the worst: an unchecked cast
        builder.append(JavaPrelude.class.getSimpleName()).append(".@<")
            .append(typeStringOf(variable.type())).append("@>_instantiationCast(")
            .append(variable.id)
            .append("())");

    else {
        // 3 -> polymorphic declaration from Prelude,
        // top level or same let scope can properly use generics
        builder.append(variableScheme.snd()).append(".@<");
        try {
            // to do so we need to instantiate the scheme and find the substitution
            Utils.Tuple@@<Substitution, Type@@> instantiation =
                variableScheme.fst().instantiate();

            Substitution subst = instantiation.fst()
                .applySubstitution(instantiation.snd().unify(variable.type()));

            // to then apply to the sorted variables
            builder.append(variableScheme.fst().sortedVariables.stream()
                    .map(ov @@-@> subst.variables().contains(ov)
                        @? typeStringOf(subst.substituteOf(ov))
                        @: Type.Variable.toString(ov))
                    .collect(Collectors.joining(", ")))
                .append("@>").append(variable.id).append("()");
        } catch (TypeException e) {
            // should never happen
            throw new InferenceException(e.getMessage());
        }
    }
    return null;
}
\end{lstlisting}
\vspace{4mm}
\begin{lstlisting}[caption={Metodo di istanziazione con \textit{cast}}, style=javaCode, label={lst:5-14-instantiation-cast-java}]
// Cast method for polymorphic functions instantiation
@SuppressWarnings("rawtypes, unchecked")
public static @@<T extends Function@@> T _instantiationCast(Function f) {
    return (T) f;
}
\end{lstlisting}

\newpage

\noindent Nei Codici~\ref{lst:5-14-nested-funx}~e~\ref{lst:5-14-nested-java} si può osservare come il procedimento
descritto produca variabili usate semplicemente, istanziate tramite \textit{cast} o con parametri di tipo.

\vspace{4mm}
\begin{lstlisting}[caption={\texttt{Let} annidati e differente uso di variabili polimorfe}, style=funxCode, label={lst:5-14-nested-funx}]
multipleIds x = let
        id1 = id
    in let
            id2 = id1
        in id2 x
\end{lstlisting}
\vspace{4mm}
\begin{lstlisting}[caption={Corrispondente traduzione in \texttt{Java}}, style=javaCode, label={lst:5-14-nested-java}]
public class Chapter5Nested {
    private Chapter5Nested() {
        // private constructor to prevent instantiation
    }
    
    public static @<h@> Function@<h, h@> multipleIds() {
        return (x @->
            (new Let@<@>() {
                private @<d@> Function@<d, d@> id1() {
                    return FunxPrelude.@<d@>id(); // 3
                }
    
                @Override
                public h _eval() {
                    return (new Let@<@>() {
                        private @<f@> Function@<f, f@> id2() {
                            return JavaPrelude
                                .@<Function@<f, f@>@>_instantiationCast(id1()); // 2
                        }
    
                        @Override
                        public h _eval() {
                            return this.@<h@>id2().apply(x); // 3 and 1
                        }
                    })._eval();
                }
            })._eval());
    }
}
\end{lstlisting}

\newpage

\subsection{Type casting "selvaggio"}
\label{sec:5-15-wild-type-casting}

L'ultima particolarità della traduzione riguarda una stranezza occasionalmente necessaria quando l'applicazione
di una funzione ad un'espressione più complessa richiede un ulteriore \textit{cast}: questi cast sono non sempre necessari
ma prevederne l'esigenza a priori è difficile ed è il motivo per cui sono stati battezzati "selvaggi" (\textit{wild cast}).


L'implementazione dei \textit{wild cast} è data da una \textit{flag} booleana abilitata in un nodo \texttt{Application}
qualora le espressioni coinvolte siano oggetti \texttt{Lambda} o \texttt{Let}, mentre il \textit{cast} è applicato
nei metodi \texttt{visit} interessati.

\vspace{4mm}
\begin{lstlisting}[caption={Metodo \texttt{visit} per le applicazioni di funzione}, style=javaCode, label={lst:5-15-visit-application-java}]
@Override
public Void visitApplication(Expression.Application application) {
    // left and right expressions necessitate a wild cast
    // if they are lambda or let expressions
    if (application.left instanceof Expression.Lambda
            || application.left instanceof Expression.Let)
        wildCast = true;
    visit(application.left);

    builder.append(".apply(");
    if (application.right instanceof Expression.Lambda
            || application.right instanceof Expression.Let)
        wildCast = true;
    visit(application.right);
    builder.append(")");
    return null;
}
\end{lstlisting}
\vspace{4mm}
\begin{lstlisting}[caption={\textit{Wild cast} in espressioni \texttt{lambda} e \texttt{let}}, style=javaCode, label={lst:5-15-wild-casts-java}]
@Override
public Void visitLambda(Expression.Lambda lambda) {
    // ...
    if (wildCast) {
        // wild cast is needed for lambdas in applications
        builder.append("(").append(typeStringOf(lambda.type())).append(") ");
        wildCast = false;
    }
    // ...
    return null;
}

@Override
public Void visitLet(Expression.Let let) {
    // ...
    if (wildCast) {
        // wild cast is needed for lets in applications
        builder.append("(").append(typeStringOf(let.type())).append(") ");
        wildCast = false;
    }
    // ...
    return null;
}
\end{lstlisting}

\newpage

Nei Codici~\ref{lst:5-15-wild-funx}~e~\ref{lst:5-15-wild-java} si nota come i \textit{cast} rendano la traduzione ancora meno
leggibile, ma si può facilmente verificare che la compilazione fallisce se alcuni \textit{cast} sono rimossi
(\textit{cast} di destra in \texttt{reverseApply} e \textit{cast} di sinistra in \texttt{anonymousIds}).

\vspace{4mm}
\begin{lstlisting}[caption={Applicazione tra funzioni, espressioni \texttt{let} e \texttt{lambda}}, style=funxCode, label={lst:5-15-wild-funx}]
reverseApply = flip (let
        apply1 f x = f x
    in apply1)

anonymousIds = (\x -> x) (\x -> x)
\end{lstlisting}
\vspace{4mm}
\begin{lstlisting}[caption={Traduzione in \texttt{Java} con \textit{cast} "selvaggi"}, style=javaCode, label={lst:5-15-wild-java}]
public class Chapter5Wild {
    private Chapter5Wild() {
        // private constructor to prevent instantiation
    }
    
    public static @<j, k@> Function@<j, Function@<Function@<j, k@>, k@>> reverseApply() {
        return FunxPrelude.@<Function@<j, k@>, j, k@>flip()
            .apply(
                (Function@<Function@<j, k@>, Function@<j, k@>>) // right wild cast
                    (new Let@<@>() {
                        private @<h, i@> Function@<Function@<h, i@>, Function@<h, i@>> apply1() {
                            return (f @-> (x @-> f.apply(x)));
                        }
    
                        @Override
                        public Function@<Function@<j, k@>, Function@<j, k@>> _eval() {
                            return this.@<j, k@>apply1();
                        }
                    })._eval());
    }
    
    public static @<n@> Function@<n, n@> anonymousIds() {
        return ((Function@<Function@<n, n@>, Function@<n, n@>>) x @-> x) // left wild cast
            .apply(((Function@<n, n@>) x @-> x)); // right wild cast
    }
}
\end{lstlisting}
