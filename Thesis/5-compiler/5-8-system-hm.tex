\subsection{Sistema HM}
\label{sec:5-8-system-hm}

Seguendo la grammatica del sistema di tipo di \textbf{Funx} e similmente alla gerarchia per l'albero sintattico astratto,
l'implementazione del \textit{sistema HM} presenta le seguenti classi e sottoclassi:
\begin{itemize}
    \item \texttt{Scheme}: schemi di tipo;
    \item \texttt{Type}: classe astratta per i monotipi;
    \item \texttt{Error}: tipo errore per la continuazione dell'inferenza;
    \item \texttt{Boring}: tipo vuoto per costanti con valore \texttt{null};
    \item \texttt{Variable}: variabili di tipo;
    \item \texttt{FunctionApplication}: applicazione di funzione di tipo.
\end{itemize}

\newpage

\begin{figure}
    \begin{tikzpicture}
        % classes
        \umlsimpleclass[type=interface]{Types}
        \umlsimpleclass[x=-4,type=interface]{Inferable}
        \umlsimpleclass[x=4,type=enum]{TypeFunction}
        \umlsimpleclass[x=-3,y=-1.15]{Context}
        \umlsimpleclass[x=3,y=-1.15]{Substitution}
        \umlsimpleclass[x=-3,y=-2.3]{Scheme}
        \umlsimpleclass[x=3,y=-2.3,type=abstract]{Type}
        \umlsimpleclass[x=-1,y=-3.45]{Error}
        \umlsimpleclass[x=-1,y=-4.6]{Boring}
        \umlsimpleclass[x=-1,y=-5.75]{Variable}
        \umlsimpleclass[x=-1,y=-6.9]{FunctionApplication}
        % relationships
        \umlinherit[geometry=-|,anchor2=-145]{Context}{Types}
        \umlinherit[geometry=-|,anchor2=-35]{Substitution}{Types}
        \umlinherit[geometry=-|,anchor2=-117]{Scheme}{Types}
        \umlinherit[geometry=-|,anchor2=-63]{Type}{Types}
        \umlinherit[geometry=-|,anchor2=-145]{Error}{Type}
        \umlinherit[geometry=-|,anchor2=-116]{Boring}{Type}
        \umlinherit[geometry=-|,anchor2=-64]{Variable}{Type}
        \umlinherit[geometry=-|,anchor2=-35]{FunctionApplication}{Type}
    \end{tikzpicture}
    \caption{Diagramma semplificato delle classi del \textit{sistema HM}}
    \label{fig:5-hm-classes}
    \vspace{4mm}
\end{figure}

\begin{lstlisting}[caption={Esempio di sottoclasse di \texttt{Type}}, style=javaCode, label={lst:5-hm-class-java}]
public static final class Variable extends Type {
    public final long id;
    
    public Variable(long id) { this.id = id; }

    public static String toString(long id) {
        return id < 26 ? Character.toString((char) ('a' + id)) : "t" + id;
    }

    public static String toFancyString(long id) {
        return id < 24 ? Character.toString((char) (945 + id)) : "t" + id;
    }

    @Override
    public Set@<Long@> freeVariables() { ... }

    @Override
    public Type applySubstitution(Substitution substitution) { ... }
}
\end{lstlisting}
\vspace{4mm}

\noindent Altre classi e interfacce di supporto sono fondamentali per la definizione dei comportamenti precedentemente descritti:
\begin{itemize}
    \item \texttt{Types}: interfaccia per operazioni comuni negli oggetti che agiscono a stretto contatto con i tipi;
          specifica i metodi per il calcolo delle variabili libere e l'applicazione di una sostituzione;
    \item \texttt{Inferable}: interfaccia già visibile in Figura~\ref{fig:5-ast-classes} e che identifica i nodi
          dell'\textbf{AST} che implementano un metodo di inferenza (i.e. le sottoclassi di \texttt{Expression});
    \item \texttt{TypeFunction}: enumerazione delle funzioni di tipo disponibili;
    \item \texttt{Substitution}: lista di sostituzioni da variabili di tipo a monotipi;
    \item \texttt{Context}: contesto di inferenza con vincoli tra variabili e schemi di tipo.
\end{itemize}

\newpage

\begin{lstlisting}[caption={Interfacce utili nel \textit{sistema HM}}, style=javaCode, label={lst:5-hm-interfaces-java}]
package com.github.massimopavoni.funx.jt.ast.node;

public sealed interface Inferable permits Expression {
    Utils.Tuple@<Substitution, Type@> infer(Context ctx);
}

// ----------------

package com.github.massimopavoni.funx.jt.ast.typesystem;

sealed interface Types@<T extends Types@<T@>@>
        permits Type, Scheme, Substitution, Context {
    Set@<Long@> freeVariables();

    T applySubstitution(Substitution substitution);
}
\end{lstlisting}
\vspace{4mm}
\begin{lstlisting}[caption={Altre classi del \textit{sistema HM}}, style=javaCode, label={lst:5-hm-more-classes-java}]
public final class Context implements Types@<Context@> {
    private final Map@<String, Scheme@> environment;

    public Scheme bindingOf(String variable) {
        return environment.get(variable);
    }

    @Override
    public Set@<Long@> freeVariables() {
        return environment.values().stream().flatMap(s @-> s.freeVariables().stream())
            .collect(ImmutableSet.toImmutableSet());
    }

    @Override
    public Context applySubstitution(Substitution substitution) {
        Context newCtx = new Context(this);
        newCtx.environment.replaceAll((v, s) @-> s.applySubstitution(substitution));
        return newCtx;
    }
}

public final class Substitution implements Types@<Substitution@> {
    public static final Substitution EMPTY = new Substitution();
    private final Map@<Long, Type@> variableTypes;

    public Type substituteOf(Long variable) {
        return variableTypes.get(variable);
    }

    @Override
    public Set@<Long@> freeVariables() {
        // similar to Context
    }

    @Override
    public Substitution applySubstitution(Substitution substitution) {
        return new Substitution(variableTypes.entrySet().stream()
            .collect(Collectors.toMap(Map.Entry::getKey,
                e @-> e.getValue().applySubstitution(substitution))));
    }
}
\end{lstlisting}