\subsection{Gerarchia delle classi}
\label{sec:5-5-class-hierarchy}

Il primo passo per la costruzione di un \textbf{AST} per la sintassi di \textbf{Funx} è la definizione
di una gerarchia di classi \texttt{Java} che rappresentano i nodi dell'albero
(package \texttt{com.github.massimopavoni.funx.jt.ast.node}).


La classe astratta \texttt{ASTNode} è la radice dell'ordinamento poiché sarà usata per gli oggetti creati
a partire dal \textit{CST}: contiene la proprietà \texttt{inputPosition} per facilitare la segnalazione di errori
e vincola le classi figlie all'implementazione del metodo astratto \texttt{accept} per visitare i nodi.

\noindent Un'altra classe astratta, derivata dalla precedente, è \texttt{Expression}, la quale identifica
un'espressione ed è dotata di alcuni campi e metodi utili all'inferenza di tipo.


Ogni altra sottoclasse (ad eccezione di \texttt{Declarations}, utilizzata per dichiarazioni globali e locali)
è una trascrizione in \texttt{Java} delle produzioni della grammatica formale:
\begin{itemize}
    \item \texttt{Module}: modulo del programma;
    \item \texttt{Declaration}: dichiarazione di funzione;
    \item \texttt{Constant}: termini costanti;
    \item \texttt{Variable}: simboli per variabili;
    \item \texttt{Application}: applicazione di funzione;
    \item \texttt{Lambda}: astrazione per le funzioni anonime;
    \item \texttt{Let}: contenitore di dichiarazioni locali;
    \item \texttt{If}: costrutto condizionale.
\end{itemize}

\newpage

\begin{figure}
    \begin{tikzpicture}
        \umlsimpleclass[type=abstract]{ASTNode}
        \umlsimpleclass[x=4,type=interface]{Inferable}
        \umlsimpleclass[x=-4,y=-2]{Module}
        \umlsimpleclass[y=-2]{Declarations}
        \umlsimpleclass[x=4,y=-2,type=abstract]{Expression}
        \umlsimpleclass[x=-2,y=-3.5]{Constant}
        \umlsimpleclass[x=-2,y=-5]{Variable}
        \umlsimpleclass[x=-2,y=-6.5]{Application}
        \umlsimpleclass[x=-2,y=-8]{Lambda}
        \umlsimpleclass[x=-2,y=-9.5]{Let}
        \umlsimpleclass[x=-2,y=-11]{If}
        \umlinherit[geometry=|-]{Module}{ASTNode}
        \umlinherit{Declarations}{ASTNode}
        \umlinherit[geometry=-|-]{Expression}{ASTNode}
        \umlimpl{Expression}{Inferable}
        \umlinherit[geometry=-|,anchor2=-160]{Constant}{Expression}
        \umlinherit[geometry=-|,anchor2=-149]{Variable}{Expression}
        \umlinherit[geometry=-|,anchor2=-117]{Application}{Expression}
        \umlinherit[geometry=-|,anchor2=-63]{Lambda}{Expression}
        \umlinherit[geometry=-|,anchor2=-31]{Let}{Expression}
        \umlinherit[geometry=-|,anchor2=-20]{If}{Expression}
    \end{tikzpicture}
    \caption{Diagramma semplificato delle classi dell'\textbf{AST}}
    \label{fig:5-ast-classes}
    \vspace{4mm}
\end{figure}

\begin{lstlisting}[caption={Esempio di classe della gerarchia}, style=javaCode, label={lst:5-class-example-java}]
public static final class Lambda extends Expression {
    public final String paramId;

    public final Expression expression;

    public Lambda(InputPosition inputPosition, String paramId, ASTNode expression) {
        super(inputPosition); // ASTNode constructor
        this.paramId = paramId;
        this.expression = (Expression) expression;
    }

    @Override // from Inferable interface
    public Utils.Tuple<Substitution, Type> infer(Context ctx) { ... }

    @Override // from Expression abstract class
    protected void propagateSubstitution(Substitution substitution) { ... }

    @Override // from ASTNode abstract class
    public @<T@> T accept(ASTVisitor<? extends T> visitor) {
        return visitor.visitLambda(this);
    }
}
\end{lstlisting}

