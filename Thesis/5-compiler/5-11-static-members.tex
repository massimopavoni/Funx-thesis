\subsection{Membri statici}
\label{sec:5-11-static-members}

Il paradigma dichiarativo dei linguaggi funzionali è ben diverso dalla programmazione ad oggetti di molti altri linguaggi rinomati,
motivo per cui la scelta di tradurre ogni programma \textbf{Funx} in un'unica classe statica è vista come semplice soluzione
per evitare complicanze e \textit{overhead} per la creazione di oggetti in aggiunta alle \texttt{Function}.


Ogni funzione definita diviene perciò una proprietà statica della classe in caso di monotipi (sezione~\ref{sec:5-13-monomorphic-declarations})
o un metodo statico con parametri di tipo in caso di politipi (sezione~\ref{sec:5-14-polymorphic-functions-instantiation}).
Fanno eccezione le funzioni appartenenti a espressioni \texttt{let} annidate: essendo classi anonime, queste creano
un unico oggetto con proprietà e metodi privati (accessibili solamente al metodo pubblico \texttt{eval} della classe \texttt{Let}).

\noindent La traduzione inizia con \textit{import} statici per la libreria standard e un costruttore privato.

\vspace{4mm}
\begin{lstlisting}[caption={Prime aggiunte alla stringa \texttt{Java}}, style=javaCode, label={lst:5-11-first-append-java}]
// append package, imports, class declaration and constructor
builder.append(module.packageName.isEmpty()
        @? ""
        @: String.format("package %s;%n", module.packageName.toLowerCase()))
    .append("\n\nimport ").append(Function.class.getName())
    .append(";\n\nimport ").append(JavaPrelude.class.getName())
    .append(";\n\nimport ").append(FunxPrelude.class.getName())
    .append(";\n\nimport static ").append(JavaPrelude.class.getName())
    .append(".*;\nimport static ").append(FunxPrelude.class.getName())
    .append(".*;\n\npublic class ").append(module.name).append(" {\n")
    .append("private ").append(module.name)
    .append("() {\n// private constructor to prevent instantiation\n}\n\n");
\end{lstlisting}
\vspace{4mm}
\begin{lstlisting}[caption={Corrispondente codice \texttt{Java} generato}, style=javaCode, label={lst:5-11-class-start-java}]
import java.util.function.Function;

import com.github.massimopavoni.funx.lib.JavaPrelude;

import com.github.massimopavoni.funx.lib.FunxPrelude;

import static com.github.massimopavoni.funx.lib.JavaPrelude.*;
import static com.github.massimopavoni.funx.lib.FunxPrelude.*;

public class Chapter5Header {
    private Chapter5Header() {
        // private constructor to prevent instantiation
    }

    // ...
}
\end{lstlisting}