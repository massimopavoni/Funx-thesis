\subsection{Analisi lessicale}
\label{sec:5-2-lexical-analysis}

Data la probabile complessità delle regole della grammatica di \textbf{Funx}, fin dall'inizio la definizione dei \textit{token} (lessemi)
del linguaggio è stata separata dalla specifica del \textit{parser}.

\noindent Il file \texttt{FunxLexer.g4} descrive i lessemi dividendoli nelle seguenti categorie:
\begin{enumerate}
    \item \textit{whitespace}: caratteri di spaziatura e tabulazione;
    \item \textit{comments}: commenti di linea e blocco;
    \item \textit{keywords}: parole chiave del linguaggio;
    \item \textit{Java keywords}: parole chiave del linguaggio Java, da evitare;
    \item \textit{types}: tipi di dato (funzioni di tipo con arità 0);
    \item \textit{literals}: costanti booleane e numeriche;
    \item \textit{variables}: identificatori per variabili di tipo o nomi di funzioni;
    \item \textit{module}: identificatori per il modulo;
    \item vari operatori simbolici per:
          \begin{itemize}
              \item \textit{bool}: valori booleani;
              \item \textit{comparison}: confronti tra numeri;
              \item \textit{arithmetic}: operazioni aritmetiche;
              \item \textit{other symbols}: simboli della sintassi (come \texttt{->}) e varie funzioni di libreria;
          \end{itemize}
    \item \textit{delimiters}: parentesi tonde, quadre e graffe.
\end{enumerate}

\noindent Le categorie 1 e 2 contengono \textit{token} da scartare, tranne \texttt{NEWLINE}, mentre la categoria 4 è utile
qualora eventualmente si permetta allo sviluppatore di utilizzare tali parole chiave riservate,
effettuando una rinomina automatica; le categorie 7 e 8 devono necessariamente apparire dopo le categorie 3 e 4,
poiché tra \textit{keyword} e identificatori di ogni genere le prime devono avere la precedenza
(la posizione della categoria 5 tiene conto di una possibile futura estensione per consentire la creazione di nuovi tipi).


Oltre alle categorie illustrate, in testa al file sono presenti dei cosiddetti \textit{fragment} (frammenti)
che semplificano le espressioni regolari dei \textit{token} e complessivamente aumentano la leggibilità della specifica.

\newpage

\begin{lstlisting}[caption={Alcuni \textit{token} del \textit{lexer}}, style=antlrCode, label={lst:5-2-lexer-antlr}]
lexer grammar FunxLexer;

// Fragments
fragment LALPHA: [a-z];
fragment UALPHA: [A-Z];
fragment ALPHA: LALPHA | UALPHA;
fragment ALPHA_: ALPHA | UnderScore;

fragment DIGIT: [0-9];
fragment DECIMAL: DIGIT+;

// Whitespace
NEWLINE: '\r'? '\n' | '\r';

TAB: [\t]+ @-> skip;
WS: [\u0020\u00a0\u1680\u2000\u200a\u202f\u205f\u3000]+ @-> skip;

// Comments
CloseMultiComment: '\./';
OpenMultiComment: '/\.';
SingleComment: '//';

COMMENT: SingleComment ~[\r\n]* @-> skip;
MULTICOMMENT: OpenMultiComment .*? CloseMultiComment @-> skip;

// Keywords
ELSE: 'else';
FI: 'fi';
IF: 'if';
IN: 'in';
LET: 'let';

// Java keywords
RESERVED_JAVA_KEYWORD: 'abstract' | 'assert' | 'boolean' | 'break' | 'byte' | [\.\.\.];

// Types
TYPE: BOOLTYPE | INTTYPE;
BOOLTYPE: 'Bool';

// Literals
INT: DECIMAL | OpenParen '\-' DECIMAL CloseParen;

// Variables
VARID: LALPHA (ALPHA_ | DIGIT)*;

// Module
MODULEID: UALPHA (ALPHA_ | DIGIT)*;

// Bool
And: '&&';
Not: '!!';

// Comparison
EqualsEquals: '\=\=';
NotEquals: '!\=';

// Arithmetic
Add: '\+';

// Other symbols
UnderScore: '_';
Arrow: '\-\>';

// Delimiters
OpenParen: '\(';
CloseParen: '\)';
\end{lstlisting}