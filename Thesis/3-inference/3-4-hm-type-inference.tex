\section{Inferenza secondo Hindley–Milner}
\label{sec:3-4-hm-type-inference}

Come già accennato, il sistema \textit{Hindley–Milner (HM)} \cite{Hindley-1969-SchemeObject,Milner-1978-TheoryPolymorphism},
o \textit{Damas-Hindley-Milner} \cite{Damas-1982-PrincipalSchemes},
è un sistema di tipo per il lambda calcolo con polimorfismo parametrico largamente utilizzato
in molti moderni linguaggi di programmazione ad alto livello. Il maggiore punto di forza del sistema
è il relativo metodo d'inferenza, in grado di dedurre automaticamente il tipo di un termine
senza annotazioni esplicite fornite dagli sviluppatori.


Prima di presentare l'algoritmo d'inferenza è necessario complementare le nozioni di lambda calcolo esteso
(Figure~\ref{fig:2-3-lambda-syntax}~e~\ref{fig:2-3-funx-syntax}) e del sistema di \textbf{Funx} (Figura~\ref{fig:3-3-system-hm})
con due concetti fondamentali:
\begin{itemize}
    \item \textbf{contesto}: una insieme di associazioni tra variabili e schemi di tipo,
          che rappresenta lo stato corrente dell'ambiente in cui un termine viene analizzato; l'unione tra la grammatica
          del linguaggio e il sistema di tipi è data da un giudizio di tipo effettuato nel contesto su un termine (espressione);
    \item \textbf{variabili libere}: l'insieme delle variabili libere di un tipo è semplicemente il complemento
          delle variabili vincolate, quantificate universalmente in un politipo.
\end{itemize}

\begin{figure}
    \vspace{4mm}
    \begin{bnf}
        $\Gamma$ : \small{Contesto} ::=
        | $\epsilon$ : \small{contesto vuoto}
        | $\Gamma + x \colon \sigma$ : \small{aggiunta di associazione}
        ;;
        : \small{Giudizio di tipo} ::= $\Gamma \vdash E \colon \sigma$
    \end{bnf}
    \par\vspace{12mm}
    \begin{bnf}
        $free(\forall\ \alpha\ .\ \sigma)$ == $free(\sigma) - \{\alpha\}$ : \small{tipo polimorfo}
        ;;
        $free(\alpha)$ == $\{\alpha\}$ : \small{variabile di tipo}
        ;;
        $free(F\ \tau_1\ldots\tau_n)$ == $\bigcup\limits_{i=1}^{n} free(\tau_i)$ : \small{applicazione di funzione di tipo}
        ;;
        $free(\Gamma)$ == $\bigcup\limits_{x\colon\sigma\in\Gamma} free(\sigma)$ : \small{contesto}
        ;;
        $free(\Gamma\vdash E\ \colon\ \sigma)$ == $free(\sigma) - free(\Gamma)$ : \small{giudizio di tipo}
    \end{bnf}
    \caption{Definizioni di contesto e variabili libere}
    \label{fig:3-4-context-free-variables}
    \vspace{4mm}
\end{figure}

\newpage

\noindent Le regole d'inferenza \cite{Clement-1986-MiniML}
del \textit{sistema HM}, riportate in Figura~\ref{fig:3-4-inference-rules},
informano il comportamento dell'\textit{algoritmo $\mathcal{W}$}, implementazione dell'inferenza di tipo.

\noindent In aggiunta a principi già noti, le peculiarità non immediatamente chiare sono:
\begin{itemize}
    \item $constantType$: funzione che restituisce il tipo di una costante;
    \item $\sigma \sqsubseteq \tau$: indica intuitivamente che $\sigma$ è più generale di $\tau$
          ($\tau$ è un'\textit{istanza} di $\sigma$);
    \item $Clos_\Gamma$: ottiene la \textit{chiusura} di una variabile, ossia il suo tipo più generale,
          quantificando universalmente le variabili libere del tipo iniziale.
\end{itemize}

\begin{figure}
    \vspace{4mm}
    \begin{mathpar}
        \begin{tabularx}{0.9\textwidth}{M P{8em}}
            \inferrule{
                \tau = constantType(c)}
            {
                \Gamma \vdash c : \tau}
             & \inferdesc{[costante]}     \\
            \inferrule{
                x : \sigma \in \Gamma  \qquad  \sigma \sqsubseteq \tau}
            {
                \Gamma \vdash x : \tau}
             & \inferdesc{[variabile]}    \\
            \inferrule{
                \Gamma \vdash E_l : \tau \rightarrow \beta  \qquad  \Gamma \vdash E_r : \tau}
            {
                \Gamma \vdash E_l\ E_r : \beta}
             & \inferdesc{[applicazione]} \\
            \inferrule{
                \Gamma + x : \beta \vdash E : \tau}
            {
                \Gamma \vdash \lambda x\ .\ E : \beta \rightarrow \tau}
             & \inferdesc{[astrazione]}   \\
            \inferrule{
                \inferrule{
                    \overline{\Gamma} = \Gamma + \vec{x} : \vec{\beta}}
                {
                    \overline{\Gamma} \vdash \vec{E} : \vec{\kappa} \qquad \overline{\Gamma} \{Clos_{\overline{\Gamma}}(\vec{\kappa}) / \vec{\beta}\} \vdash E_i : \tau}}
            {
                \overline{\Gamma} \vdash \textbf{let}\ \vec{x} = \vec{E}\ \textbf{in}\ E_i : \tau}
             & \inferdesc{[let]}          \\
            $Clos_\Gamma(\tau) = \forall\ \hat{\alpha}\ .\ \tau \qquad \hat{\alpha} = free(\tau) - free(\Gamma)$
             &                            \\
            \inferrule{
                \Gamma \vdash E_c : Bool  \qquad  \Gamma \vdash E_t : \tau  \qquad  \Gamma \vdash E_e : \tau}
            {
                \Gamma \vdash \textbf{if}\ E_c\ \textbf{then}\ E_t\ \textbf{else}\ E_e : \tau}
             & \inferdesc{[if]}           \\
        \end{tabularx}
    \end{mathpar}
    \caption{Regole di inferenza del \textit{sistema HM} in \textbf{Funx}}
    \label{fig:3-4-inference-rules}
    \vspace{4mm}
\end{figure}

\noindent L'\textit{algoritmo $\mathcal{W}$} \cite{Lee-1998-FolkloreInference}
in Figura~\ref{fig:3-4-algorithm-w} fornisce un'implementazione del procedimento finora descritto,
servendosi delle regole d'inferenza e dettagliando i passi per la deduzione del tipo di ogni espressione disponibile in \textbf{Funx}.

\noindent L'algoritmo appare nel codice del compilatore in una forma molto simile, seppur talvolta più complessa,
come nel caso dell'espressione \texttt{let} (sezione~\ref{sec:5-9-expression-inference}).

\newpage

\begin{figure}
    $\mathcal{W} : Context \times Expression \rightarrow Substitution \times Type$
    \newcommand{\algW}[2]{\mathcal{W}(#1, #2)}
    \newcommand{\algWline}[2]{& \algW{#1}{#2} & & = & &}
    \[
        \begin{aligned}
            \algWline{\Gamma}{c} (id, constantType(c))
            \\
            \algWline{\Gamma}{x}
            \begin{aligned}[t]
                 & \text{let} &  & \forall\ \vec{\alpha}\ .\ \tau = \Gamma(x),\ new\ \vec{\beta} \\
                 & \text{in}  &  & (id, \{\vec{\beta} / \vec{\alpha}\} \tau)
            \end{aligned}
            \\
            \algWline{\Gamma}{E_l\ E_r}
            \begin{aligned}[t]
                 & \text{let} &  & (S_l, \tau_l) = \algW{\Gamma}{E_l}                                   \\
                 &            &  & (S_r, \tau_r) = \algW{S_l\Gamma}{E_r}                                \\
                 &            &  & S_a = \mathcal{U}(S_r \tau_l, \tau_r \rightarrow \beta),\ new\ \beta \\
                 & \text{in}  &  & (S_a S_r S_l, S_a \beta)
            \end{aligned}
            \\
            \algWline{\Gamma}{\lambda x\ .\ E}
            \begin{aligned}[t]
                 & \text{let} &  & (S, \tau) = \algW{\Gamma + x : \beta}{E},\ new\ \beta \\
                 & \text{in}  &  & (S, S \beta \rightarrow \tau)
            \end{aligned}
            \\
            \algWline{\Gamma}{\textbf{let}\ \vec{x} = \vec{E}\ \textbf{in}\ E_i}
            \begin{aligned}[t]
                 & \text{let} &  & \overline{\Gamma} = \Gamma + \vec{x} : \vec{\beta},\ new\ \vec{\beta}                                                  \\
                 &            &  & (\vec{S}, \vec{\kappa}) = \algW{\overline{\Gamma}}{\vec{E}}                                                            \\
                 &            &  & (S_i, \tau) = \algW{\vec{S}\,\overline{\Gamma} \{Clos_{\vec{S}\,\overline{\Gamma}}(\vec{\kappa}) / \vec{\beta}\}}{E_i} \\
                 & \text{in}  &  & (S_i \vec{S}, \tau)
            \end{aligned}
            \\
            \algWline{\Gamma}{\textbf{if}\ E_c\ \textbf{then}\ E_t\ \textbf{else}\ E_e}
            \begin{aligned}[t]
                 & \text{let} &  & (S_c, \tau_c) = \algW{\Gamma}{E_c}         \\
                 &            &  & (S_t, \tau_t) = \algW{S_c \Gamma}{E_t}     \\
                 &            &  & (S_e, \tau_e) = \algW{S_t S_c \Gamma}{E_e} \\
                 &            &  & S_{cb} = \mathcal{U}(\tau_c, Bool)         \\
                 &            &  & S_{te} = \mathcal{U}(S_e \tau_t, \tau_e)   \\
                 & \text{in}  &  & (S_{te} S_{cb} S_e S_t S_c, S_{te} \tau_t)
            \end{aligned}
        \end{aligned}
    \]
    \caption{Algoritmo $\mathcal{W}$}
    \label{fig:3-4-algorithm-w}
    \vspace{4mm}
\end{figure}

\noindent Dato un contesto in cui analizzare un'espressione, $\mathcal{W}$ restituisce un tipo e una sostituzione.
Quest'ultima può essere vuota ($id$), generata manualmente ($\{\tau_2 / \tau_1\}$) o dall'\textit{unificazione} di altri tipi,
e rappresenta una mappatura tra variabili di tipo e altri tipi; può essere applicata a tipi e contesti.

\noindent La funzione $new$ esprime la creazione di nuove variabili di tipo, con il vincolo che non siano state ancora usate dall'algoritmo.

\noindent Infine, l'\textit{unificazione} (funzione $\mathcal{U}$) è un'operazione altrettanto importante per l'inferenza di tipo,
con la quale si cerca di trovare una sostituzione che renda due tipi uguali.

\begin{figure}
    \vspace{4mm}
    $\mathcal{U} : Type \times Type \rightarrow Substitution$
    \newcommand{\funU}[2]{\mathcal{U}(#1, #2)}
    \newcommand{\funUline}[3]{& \funU{#1}{#2} & & = & & {#3}}
    \[
        \begin{aligned}
            \funUline{\alpha}{\alpha}{id}              \\
            \funUline{\alpha}{\tau}{\{\tau / \alpha\}} \\
            \funUline{\tau}{\alpha}{\{\tau / \alpha\}} \\
            \funUline{F\ \vec{\tau}}{F\ \vec{\kappa}}{\funU{\vec{\tau}}{\vec{\kappa}}}
        \end{aligned}
    \]
    \caption{Funzione $\mathcal{U}$}
    \label{fig:3-4-unification}
\end{figure}