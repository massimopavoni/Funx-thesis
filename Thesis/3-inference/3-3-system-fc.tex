\subsection{Sistema FC}
\label{sec:3-3-system-fc}

Tra i vari sistemi di tipo per il lambda calcolo, uno dei più interessanti è \textit{lambda F}
(vertice $\lambda2$ in Figura \ref{fig:3-lambda-cube}) poiché molto utile per la generalizzazione delle funzioni:
un problema molto ricorrente nella programmazione con qualsiasi linguaggio è infatti la duplicazione di codice
per funzioni che svolgono operazioni simili su tipi diversi.


Il \textit{sistema F} risolve tale problema introducendo il \textbf{polimorfismo parametrico}
e di conseguenza la distinzione tra tipi monomorfi (monotipi) e tipi polimorfi (politipi).

\noindent I tipi delle funzioni possono essere caratterizzati tramite quantificatori universali e variabili di tipo
ove sia necessario un tipo generico (spesso vengono usate singole lettere dell'alfabeto greco o latino).


Tuttavia, $\lambda2$ nella sua forma più pura, oltre a non essere un sistema Turing-completo
(è possibile definire solamente la ricorsione primitiva), rende l'inferenza di tipo trattata nella sezione
\ref{sec:3-4-hm-type-inference} un problema non decidibile%
\footnote{\citetitle{Wells-1999-TypabilityUndecidable}, \cite{Wells-1999-TypabilityUndecidable}}.

\noindent Pertanto, il linguaggio \texttt{Haskell} non implementa semplicemente il \textit{sistema F},
ma piuttosto una versione ristretta di $\lambda\omega$ chiamata \textit{sistema FC}%
\footnote{\citetitle{Eisenberg-2015-SystemFC}, \cite{Eisenberg-2015-SystemFC}}.

\noindent Quest'ultima include anche i costruttori di tipo (\textit{funzioni di tipo} in \textbf{Funx}),
frenando però il polimorfismo ai cosiddetti \textit{tipi polimorfici di rango 1} (\textit{polimorfismo predicativo}):
tale limitazione si manifesta nella scrittura di tutti i quantificatori universali all'inizio di un tipo polimorfo
(che prende il nome di \textit{schema di tipo}).

\noindent Le versioni invece più espressive e più vicine a $\lambda\omega$ sono:
\begin{itemize}
    \item \textit{polimorfismo di rango superiore}: supporta quantificatori universali in qualsiasi punto
          nelle definizioni delle funzioni (e.g. \texttt{Bool -> (forall b $\mathord{.}$ b -> b)});
          \texttt{Haskell} lo realizza con l'estensione \texttt{RankNTypes} del compilatore \texttt{GHC},
          mentre offre anche l'estensione \texttt{Rank2Types}, per la quale l'inferenza rimane decidibile;
    \item \textit{polimorfismo impredicativo}: permette di quantificare le variabili di tipo in modo arbitrario,
          anche e soprattutto all'interno dei costruttori di tipo (e.g. \texttt{Maybe (forall a $\mathord{.}$ a -> a) -> Bool},
          possibile in \texttt{Haskell} abilitando l'estensione \texttt{ImpredicativeTypes}).
\end{itemize}

\noindent Il linguaggio \textbf{Funx} ovviamente non è correntemente in grado di supportare queste estensioni
del sistema di tipo, così come non è possibile definire nuovi tipi o fare uso di \textit{classi di tipo}
simili a quelle proprie di \texttt{Haskell}. Affermare che \textbf{Funx} adotti il \textit{sistema FC} potrebbe
lasciare intendere un linguaggio più espressivo di quanto non sia in realtà: è dunque più opportuno realizzare
il \textit{sistema HM}, di cui il \textit{sistema FC} è un ampliamento,
e che comunque ben si presta allo scopo principe di traduzione in \texttt{Java}.


In Tabella \ref{tab:3-polymorphic-functions} si possono osservare i tipi di alcune funzioni polimorfe
di \texttt{Haskell}: la sintassi di \textbf{Funx} è molto simile (identica in ognuno dei casi presentati),
con l'eccezione che la parola chiave \texttt{forall} è completamente assente dal linguaggio, in quanto ogni identificatore
che inizia con una lettera minuscola è considerato una variabile di tipo da quantificare universalmente
(vedi sezione \ref{sec:5-?-type-inference}).

\newpage

\begin{table}[H]
    \vspace{4mm}
    \begin{center}
        \begin{tabularx}{\textwidth}{|P{6em}|X|}
            \hline
            \textbf{Funzione} & \textbf{Schema}                                                    \\
            \hline
            \texttt{id}       & \texttt{forall a $\mathord{.}$ a -> a}                             \\
            \hline
            \texttt{const}    & \texttt{forall a b $\mathord{.}$ a -> b -> a}                      \\
            \hline
            \texttt{(.)}      & \texttt{forall b c a $\mathord{.}$ (b -> c) -> (a -> b) -> a -> c} \\
            \hline
            \texttt{flip}     & \texttt{forall a b c $\mathord{.}$ (a -> b -> c) -> b -> a -> c}   \\
            \hline
            \texttt{(\$)}     & \texttt{forall a b $\mathord{.}$ (a -> b) -> a -> b}               \\
            \hline
            \texttt{(\&)}     & \texttt{forall a b $\mathord{.}$ a -> (a -> b) -> b}               \\
            \hline
        \end{tabularx}
    \end{center}
    \caption{Esempi di funzioni polimorfe}
    \label{tab:3-polymorphic-functions}
    \vspace{4mm}
\end{table}

\noindent In Figura \ref{fig:3-system-hm} è mostrata la grammatica per la definizione dei tipi
implementati nel linguaggio \textbf{Funx}. Si noti come i tipi monomorfi siano solo
variabili di tipo o applicazioni di funzioni ad altri tipi; al momento il linguaggio mette a disposizione
le funzioni di tipo più elementari, la cui arietà è indicata in pedice.

\begin{figure}
    \vspace{4mm}
    \begin{bnf}
        $\sigma$ : \small{Schema di tipo} ::=
        | $\tau$ : \small{monotipo}
        | $\forall\ \alpha\ .\ \sigma$ : \small{politipo}
        ;;
        $\tau$ : \small{Tipo} ::=
        | $\alpha$ : \small{variabile di tipo}
        | $F\ \tau\ldots\tau$ : \small{applicazione di funzione di tipo}
        ;;
        $F$ : \small{Funzione di tipo} ::=
        | $\rightarrow_2$ : \small{funzione}
        | $Bool_0$ : \small{booleano}
        | $Int_0$ : \small{intero}
    \end{bnf}
    \caption{Grammatica del sistema di tipo di \textbf{Funx}}
    \label{fig:3-system-hm}
    \vspace{4mm}
\end{figure}