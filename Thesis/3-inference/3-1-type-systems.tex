\section{Sistemi di tipo}
\label{sec:3-1-type-systems}

Durante la genesi di ogni linguaggio di programmazione, una delle scelte più significative riguarda
l'introduzione di un sistema per gestire i tipi di variabili ed espressioni.

\noindent Tali sistemi di tipo sono di fatto insiemi di regole logiche che permettono
di assegnare una proprietà \textit{"type"} a ciascuno dei termini del linguaggio che ne necessitano.

\noindent Sono principalmente suddivisi in due categorie:
\begin{itemize}
      \item \textbf{tipizzazione statica}: i tipi sono definiti a tempo di compilazione
            e non possono cambiare mentre il programma è in esecuzione;
      \item \textbf{tipizzazione dinamica}: i tipi vengono stabiliti durante l'esecuzione
            e possono cambiare in qualsiasi momento.
\end{itemize}

\noindent Oltre a questa distinzione esistono varie sfumature e approcci differenti,
informalmente classificati in base alla rigidità delle regole di tipizzazione.
Si parla di \textit{tipizzazione debole} quando ad esempio sono consentite conversioni implicite tra tipi diversi,
\textit{tipizzazione forte} se sono impedite, oppure qualora sia o meno disponibile l'aritmetica dei puntatori.

\newpage

\begin{figure}
      \begin{tikzpicture}[scale=0.5, >={Stealth[width=1.5mm,length=2mm]}]
            % nodes instead of proper axis, less of a hassle
            \node (dynamic) at (-11,0) {Dinamico};
            \node (static) at (11,0) {Statico};
            \node (weak) at (0,-11) {Debole};
            \node (strong) at (0,11) {Forte};
            \draw[<->] (dynamic) -- (static);
            \draw[<->] (weak) -- (strong);
            % bleh
            \node at (-8,-5) {\texttt{Visual Basic}};
            \node at (-4,-8) {\texttt{JavaScript}};
            \node at (-8,-2) {\texttt{Perl}};
            \node at (-3,-3) {\texttt{PHP}};
            % meh
            \node at (-7,8) {\texttt{Erlang}};
            \node at (-8,5) {\texttt{Clojure}};
            \node at (-3,7) {\texttt{Groovy}};
            \node at (-6,3) {\texttt{Python}};
            \node at (-2,2) {\texttt{Ruby}};
            % okay
            \node at (3,-4) {\texttt{C}};
            \node at (6,-6) {\texttt{C++}};
            % good
            \node at (3,7) {\texttt{C\#}};
            \node at (4,1) {\texttt{Funx}}; % bleh amongst the good
            \node at (5,4) {\texttt{Java}};
            \node at (2,3) {\texttt{F\#}};
            \node at (7,6) {\texttt{Scala}};
            \node at (8,2) {\texttt{Haskell}}; % hell yeah
            \node at (8,8) {\texttt{Rust}}; % divine
      \end{tikzpicture}
      \caption{Alcuni linguaggi e loro sistemi di tipo}
      \label{fig:3-1-languages-type-systems}
      \vspace{4mm}
\end{figure}

\noindent Grazie ai tipi dinamici, linguaggi quali \texttt{Python} e \texttt{JavaScript} permettono
veloce prototipazione, flessibilità e codice più conciso, a discapito però di una più alta
probabilità d'incontrare errori importanti a runtime, piuttosto che in fase di compilazione.


Al contrario, i tipi statici spesso migliorano naturalmente la mantenibilità di un progetto:
viene limitata la possibilità di scorciatoie nello sviluppo, ma si hanno maggiori garanzie di correttezza,
in quanto il compilatore può implementare ulteriori controlli e segnalare errori semantici più precisi già
prima dell'esecuzione del programma.

\noindent D'altro canto, l'obbligo di specificare i tipi di ogni variabile, oggetto, funzione e
parametro può risultare tedioso e talvolta ridondante; molti linguaggi moderni,
tra cui \texttt{Haskell}, ovviano a quest'inconvenienza tramite l'uso dell'inferenza di tipo.

\noindent Gli algoritmi d'inferenza introducono numerosi benefici, in particolare:
\begin{itemize}
      \item la scrittura del codice è meno onerosa per lo sviluppatore a prescindere dal sistema di tipi utilizzato,
            e diviene quindi estremamente vantaggioso utilizzare tipi statici;
      \item le annotazioni ora opzionali possono essere aggiunte dal programmatore quando vi sono casi difficili
            da disambiguare automaticamente, oppure per migliorare la leggibilità del codice;
      \item gli strumenti di sviluppo per il linguaggio possono sfruttare informazioni fornite dal motore inferenziale
            per suggerire il tipo delle espressioni e arricchire i messaggi di errore e di warning.
\end{itemize}

\newpage

\subsection{\texorpdfstring{\textlambda}{lambda}-cubo}
\label{sec:3-2-lambda-cube}

Al fine di comprendere quale sistema il linguaggio \textbf{Funx} implementi,
prima di discutere l'inferenza si vuol descrivere brevemente il \textit{$\lambda$-cubo}, lambda cubo%
\footnote{\citetitle{Barendregt-1991-GeneralizedSystems}, \cite{Barendregt-1991-GeneralizedSystems}},
un modello introdotto per classificare i sistemi di tipo applicabili al lambda calcolo.


\noindent In Figura \ref{fig:3-lambda-cube} è possibile osservare come la struttura del cubo abbia all'origine
il \textit{lambda calcolo semplicemente tipato} ($\lambda\mkern-9mu\rightarrow$) e come le tre dimensioni
in cui si sviluppa rappresentino ciascuna un'estensione del sistema:
\begin{itemize}
    \item \textbf{tipi dipendenti} ($\rightarrow$): la definizione dei tipi può dipendere dai valori delle variabili
          (implementati da linguaggi funzionali come \texttt{Agda}, \texttt{Coq} e \texttt{Idris});
    \item \textbf{polimorfismo parametrico} ($\uparrow$): i tipi possono essere polimorfi, generalizzati
          tramite variabili di tipo (presenti nei sistemi adottati da \texttt{ML}, \texttt{OCaml} e \texttt{Haskell});
    \item \textbf{costruttori di tipo} ($\nearrow$): capacità di costruire nuovi tipi a partire da tipi esistenti
          (\texttt{Haskell} ne fa grande uso poiché ogni nuovo tipo,
          dichiarato con la keyword \texttt{data}, è un nuovo costruttore di tipo).
\end{itemize}

\begin{figure}
    \vspace{4mm}
    \begin{tikzpicture}[scale=1.6, >={Stealth[width=1.5mm,length=2mm]}]
        % base, weird xyz coords
        \node (lst) at (0,0,2) {$\lambda\mkern-4mu\rightarrow$};
        \node (lwo) at (0,0,0) {$\lambda\underline{\omega}$};
        \node (lP) at (2,0,2) {$\lambda P$};
        \node (lPwo) at (2,0,0) {$\lambda P\underline{\omega}$};
        % cube hat
        \node (l2) at (0,2,2) {$\lambda2$};
        \node (lo) at (0,2,0) {$\lambda\omega$};
        \node (lP2) at (2,2,2) {$\lambda P2$};
        \node (lPw) at (2,2,0) {$\lambda P\omega$};
        \node (lC) at (2.6,2,0) {$=\lambda C$}; % woah
        % connect base
        \draw[->] (lst) -- (lwo);
        \draw[->] (lst) -- (lP);
        \draw[->] (lwo) -- (lPwo);
        \draw[->] (lP) -- (lPwo);
        % connect hat
        \draw[->] (l2) -- (lo);
        \draw[->] (l2) -- (lP2);
        \draw[->] (lo) -- (lPw);
        \draw[->] (lP2) -- (lPw);
        % connect base and hat
        \draw[->] (lst) -- (l2);
        \draw[->] (lwo) -- (lo);
        \draw[->] (lP) -- (lP2);
        \draw[->] (lPwo) -- (lPw);
    \end{tikzpicture}
    \caption{\textlambda-cubo}
    \label{fig:3-lambda-cube}
    \vspace{4mm}
\end{figure}

\noindent Senza entrare troppo nei dettagli, in ordine crescente di potenza espressiva:

\begin{itemize}
    \item $\lambda\mkern-4mu\rightarrow$ (\textit{lambda calcolo semplicemente tipato}): tipi monomorfi;
    \item $\lambda\underline{\omega}$ (\textit{lambda weak omega}): costruttori di tipo;
    \item $\lambda2$ (\textit{lambda due, lambda F, lambda calcolo polimorfico}): polimorfismo parametrico;
    \item $\lambda P$ (\textit{lambda P}): tipi dipendenti;
    \item $\lambda P\underline{\omega}$ (\textit{lambda P weak omega}): costruttori di tipo e tipi dipendenti;
    \item $\lambda\omega$ (\textit{lambda omega}): costruttori di tipo e polimorfismo parametrico;
    \item $\lambda P2$ (\textit{lambda P due}): polimorfismo parametrico e tipi dipendenti;
    \item $\lambda P\omega\!=\!\lambda C$ (\textit{lambda P omega, lambda C, calcolo delle costruzioni}): cstronglyombinazione di tutte le tre estensioni.
\end{itemize}

\subsection{Sistema FC}
\label{sec:3-3-system-fc}

Tra i vari sistemi di tipo per il lambda calcolo, uno dei più interessanti è \textit{lambda F}
(vertice $\lambda2$ in Figura~\ref{fig:3-2-lambda-cube}) poiché molto utile per la generalizzazione delle funzioni:
un problema molto ricorrente nella programmazione con qualsiasi linguaggio è infatti la duplicazione di codice
per funzioni che svolgono operazioni simili su tipi diversi.


Il \textit{sistema F} risolve tale problema introducendo il \textbf{polimorfismo parametrico}
e di conseguenza la distinzione tra tipi monomorfi (monotipi) e tipi polimorfi (politipi).

\noindent I tipi delle funzioni possono essere caratterizzati tramite quantificatori universali e variabili di tipo
ove sia necessario un tipo generico (spesso vengono usate singole lettere dell'alfabeto greco o latino).


Tuttavia, $\lambda2$ nella sua forma più pura, oltre a non essere un sistema Turing-completo
(è possibile definire solamente la ricorsione primitiva), rende l'inferenza di tipo trattata nella sezione
\ref{sec:3-4-hm-type-inference} un problema non decidibile \cite{Wells-1999-TypabilityUndecidable}.

\noindent Pertanto, il linguaggio \texttt{Haskell} non implementa semplicemente il \textit{sistema F},
ma piuttosto una versione ristretta di $\lambda\omega$ chiamata \textit{sistema FC} \cite{Eisenberg-2015-SystemFC}.

\noindent Quest'ultima include anche i costruttori di tipo (\textit{funzioni di tipo} in \textbf{Funx}),
frenando però il polimorfismo ai cosiddetti \textit{tipi polimorfici di rango 1} (\textit{polimorfismo predicativo}):
tale limitazione si manifesta nella scrittura di tutti i quantificatori universali all'inizio di un tipo polimorfo
(che prende il nome di \textit{schema di tipo}).

\noindent Le versioni invece più espressive e più vicine a $\lambda\omega$ sono:
\begin{itemize}
    \item \textit{polimorfismo di rango superiore}: supporta quantificatori universali in qualsiasi punto
          nelle definizioni delle funzioni (e.g. \texttt{Bool -> (forall b $\mathord{.}$ b -> b)});
          \texttt{Haskell} lo realizza con l'estensione \texttt{RankNTypes} del compilatore \texttt{GHC},
          mentre offre anche l'estensione \texttt{Rank2Types}, per la quale l'inferenza rimane decidibile;
    \item \textit{polimorfismo impredicativo}: permette di quantificare le variabili di tipo in modo arbitrario,
          anche e soprattutto all'interno dei costruttori di tipo (e.g. \texttt{Maybe (forall a $\mathord{.}$ a -> a) -> Bool},
          possibile in \texttt{Haskell} abilitando l'estensione \texttt{ImpredicativeTypes}).
\end{itemize}

\noindent Il linguaggio \textbf{Funx} ovviamente non è correntemente in grado di supportare queste estensioni
del sistema di tipo, così come non è possibile definire nuovi tipi o fare uso di \textit{classi di tipo}
simili a quelle proprie di \texttt{Haskell}. Affermare che \textbf{Funx} adotti il \textit{sistema FC} potrebbe
lasciare intendere un linguaggio più espressivo di quanto non sia in realtà: è dunque più opportuno realizzare
il \textit{sistema HM}, di cui il \textit{sistema FC} è un ampliamento,
e che comunque ben si presta allo scopo principe di traduzione in \texttt{Java}.


In Tabella~\ref{tab:3-3-polymorphic-functions} si possono osservare i tipi di alcune funzioni polimorfe
di \texttt{Haskell}: la sintassi di \textbf{Funx} è molto simile (identica in ognuno dei casi presentati),
con l'eccezione che la parola chiave \texttt{forall} è completamente assente dal linguaggio, in quanto ogni identificatore
che inizia con una lettera minuscola è considerato una variabile di tipo da quantificare universalmente
(sezione~\ref{sec:5-2-lexical-analysis}).

\newpage

\begin{table}[H]
    \vspace{4mm}
    \begin{center}
        \begin{tabularx}{\textwidth}{|P{6em}|X|}
            \hline
            \textbf{Funzione} & \textbf{Schema}                                                    \\
            \hline
            \texttt{id}       & \texttt{forall a $\mathord{.}$ a -> a}                             \\
            \hline
            \texttt{const}    & \texttt{forall a b $\mathord{.}$ a -> b -> a}                      \\
            \hline
            \texttt{(.)}      & \texttt{forall b c a $\mathord{.}$ (b -> c) -> (a -> b) -> a -> c} \\
            \hline
            \texttt{flip}     & \texttt{forall a b c $\mathord{.}$ (a -> b -> c) -> b -> a -> c}   \\
            \hline
            \texttt{(\$)}     & \texttt{forall a b $\mathord{.}$ (a -> b) -> a -> b}               \\
            \hline
            \texttt{(\&)}     & \texttt{forall a b $\mathord{.}$ a -> (a -> b) -> b}               \\
            \hline
        \end{tabularx}
    \end{center}
    \caption{Esempi di funzioni polimorfe}
    \label{tab:3-3-polymorphic-functions}
    \vspace{4mm}
\end{table}

\noindent In Figura~\ref{fig:3-3-system-hm} è mostrata la grammatica per la definizione dei tipi
implementati nel linguaggio \textbf{Funx}. Si noti come i tipi monomorfi siano solo
variabili di tipo o applicazioni di funzioni ad altri tipi; al momento il linguaggio mette a disposizione
le funzioni di tipo più elementari, la cui arietà è indicata in pedice.

\begin{figure}
    \vspace{4mm}
    \begin{bnf}
        $\sigma$ : \small{Schema di tipo} ::=
        | $\tau$ : \small{monotipo}
        | $\forall\ \alpha\ .\ \sigma$ : \small{politipo}
        ;;
        $\tau$ : \small{Tipo} ::=
        | $\alpha$ : \small{variabile di tipo}
        | $F\ \tau\ldots\tau$ : \small{applicazione di funzione di tipo}
        ;;
        $F$ : \small{Funzione di tipo} ::=
        | $\rightarrow_2$ : \small{funzione}
        | $Bool_0$ : \small{booleano}
        | $Int_0$ : \small{intero}
    \end{bnf}
    \caption{Grammatica del sistema di tipo di \textbf{Funx}}
    \label{fig:3-3-system-hm}
    \vspace{4mm}
\end{figure}