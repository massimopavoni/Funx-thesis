\section{Estensioni del linguaggio}
\label{sec:6-2-language-extensions}

Nel caso in cui si volesse continuare a sviluppare \textbf{Funx} e il compilatore annesso,
quelle che seguono sono alcune estensioni utili a incrementarne notevolmente l'espressività:
\begin{itemize}
    \item \textbf{tuple}: nuovo tipo analogo alle tuple di \texttt{Haskell}, da ideare anche all'interno di \texttt{Java}
          stesso data la mancanza di supporto;
    \item \textbf{liste}: implementazione della corrispondenza tra le liste di \texttt{Java} e un nuovo tipo in \textbf{Funx},
          con il vantaggio di poter utilizzare alcune funzioni native;
    \item \textbf{stringhe}: la gestione di dati testuali si rivela essere potenzialmente tra le più complicate a seconda
          della traduzione dei singoli caratteri (utilizzare le liste in \texttt{Java} si rivelerebbe inefficiente);
    \item \textbf{switch-case}: aggiunta di una struttura di controllo più versatile rispetto all'attuale \texttt{if-else},
          affine alle \textit{switch expression} di \texttt{Java} e i \textit{case-of} di \texttt{Haskell};
    \item \textbf{tipi custom}: l'introduzione di tipi definiti dall'utente è un'altra espansione difficile da trattare,
          poiché andrebbero probabilmente usate le classi di \texttt{Java}, da generare sempre automaticamente;
    \item \textbf{suite di test}: con l'ampliamento del linguaggio, è indispensabile preparare una collezione di test
          per la verifica della correttezza del compilatore.
\end{itemize}