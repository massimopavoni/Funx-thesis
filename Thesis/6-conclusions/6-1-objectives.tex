\section{Obiettivi}
\label{sec:6-1-objectives}

Relativamente all'obiettivo di sviluppo del compilatore, l'approccio costruttivo della stringa \texttt{Java}
costituisce senz'altro una traduzione molto semplice e non eccessivamente intricata:
si è stati in grado di utilizzare le \textit{feature} menzionate nel Capitolo~\ref{chap:4-java} in modo
appropriato e in accordo con la specifica del piccolo linguaggio funzionale \textbf{Funx}.


Le soluzioni ai problemi di \textit{parsing}, inferenza e generazione del codice si sono rivelate abbastanza
agili e corrette da favorire tempi di compilazione ed esecuzione accettabili, ma ovviamente non sufficienti
per giustificare la preferenza di \textbf{Funx} rispetto ad altri linguaggi maturi;
d'altro canto il progetto non ha mai avuto simili pretese, e da subito è stato chiaro che la scelta della \texttt{JVM}
come piattaforma di destinazione avrebbe limitato le prestazioni dei programmi.


Per quanto il linguaggio permetta solamente di definire semplici funzioni usando numeri interi e costanti booleane,
le minime funzionalità di cui è dotato soddisfano i requisiti prestabiliti, inclusa la possibilità di ricorsione. 