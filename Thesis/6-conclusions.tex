\chapter{\localized{Conclusioni}{Conclusions}}
\label{chap:6-conclusions}

All'epilogo del lavoro di tesi, è possibile affermare che il progetto è stato portato a termine con un più che discreto successo:
il linguaggio funzionle \textbf{Funx} è molto semplice ma può essere esteso per aumentarne l'espressività,
mentre il compilatore offre spazio per miglioramenti e ottimizzazioni.


Contemporaneamente, come ogni software, il risultato è lontano dall'essere perfetto,
e durante le fasi finali di sviluppo e di scrittura di questo documento sono emerse criticità e \textit{bug}
nel traduttore che ne inficiano affidabilità ed efficienza in alcuni rari casi, non scoperti in precedenza.


In questo breve capitolo verranno confrontati i risultati con gli obiettivi prefissati,
suggerite possibili estensioni per il linguagio e accennato un uso didattico.

\section{Obiettivi}
\label{sec:6-1-objectives}

Relativamente all'obiettivo di sviluppo del compilatore, l'approccio costruttivo della stringa \texttt{Java}
costituisce senz'altro una traduzione molto semplice e non eccessivamente intricata:
si è stati in grado di utilizzare le \textit{feature} menzionate nel Capitolo~\ref{chap:4-java} in modo
appropriato e in accordo con la specifica del piccolo linguaggio funzionale \textbf{Funx}.


Le soluzioni ai problemi di \textit{parsing}, inferenza e generazione del codice si sono rivelate abbastanza
agili e corrette da favorire tempi di compilazione ed esecuzione accettabili, ma ovviamente non sufficienti
per giustificare la preferenza di \textbf{Funx} rispetto ad altri linguaggi maturi;
d'altro canto il progetto non ha mai avuto simili pretese, e da subito è stato chiaro che la scelta della \texttt{JVM}
come piattaforma di destinazione avrebbe limitato le prestazioni dei programmi.


Per quanto il linguaggio permetta solamente di definire semplici funzioni usando numeri interi e costanti booleane,
le minime funzionalità di cui è dotato soddisfano i requisiti prestabiliti, inclusa la possibilità di ricorsione. 

\newpage

\section{Estensioni del linguaggio}
\label{sec:6-2-language-extensions}

Nel caso in cui si volesse continuare a sviluppare \textbf{Funx} e il compilatore annesso,
quelle che seguono sono alcune estensioni utili ad incrementarne notevolemente l'espressività:
\begin{itemize}
    \item \textbf{tuple}: nuovo tipo analogo alle tuple di \texttt{Haskell}, da ideare anche all'interno di \texttt{Java}
          stesso data la mancanza di supporto;
    \item \textbf{liste}: implementazione della corrispondenza tra le liste di \texttt{Java} e un nuovo tipo in \textbf{Funx},
          con il vantaggio di poter utilizzare alcune funzioni native;
    \item \textbf{stringhe}: la gestione di dati testuali si rivela essere potenzialmente tra le più complicate a seconda
          della traduzione dei singoli caratteri (utilizzare le liste in \texttt{Java} si rivelerebbe inefficiente);
    \item \textbf{switch-case}: aggiunta di una struttura di controllo più versatile rispetto all'attuale \texttt{if-else},
          affine alle \textit{switch expression} di \texttt{Java} e i \textit{case-of} di \texttt{Haskell};
    \item \textbf{tipi custom}: l'introduzione di tipi definiti dall'utente è un'altra espansione difficile da trattare,
          poiché la \textit{feature} di \texttt{Java} più idonea è l'uso delle classi, da generare sempre in modo automatico;
    \item \textbf{suite di test}: con l'ampliamento del linguaggio, è indispensabile preparare una collezione di test
          per la verifica della correttezza del compilatore.
\end{itemize}

\section{Scopo educativo}
\label{sec:6-3-educational-purpose}

L'adozione di un linguaggio simile a \textbf{Funx} a scopo educativo offrirebbe opportunità significative
per arricchire l'insegnamento della programmazione, soprattutto dove \texttt{Java} e altri linguaggi
orientati agli oggetti dominano il curriculum. In ambito scolastico e accademico si tende spesso a concentrarsi
molto sulla programmazione imperativa anche nelle fasi più avanzate dei corsi,
lasciando agli studenti l'approfondimento di altri paradigmi introdotti con minore importanza.

Alcuni concetti fondamentali di ogni linguaggio sono tuttavia più facilmente compresi attraverso la programmazione funzionale:
l'uso di \textbf{Funx}, assieme alla traduzione diretta in un linguaggio più familiare, sarebbe forse un modo per addolcire
la transizione a un paradigma al quale la maggior parte degli studenti non è abituata.