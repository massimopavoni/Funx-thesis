\chapter{Funx}
\label{chap:2-funx}

Questo capitolo descrive brevemente i linguaggi funzionali e le scelte effettuate
durante l'ideazione del linguaggio usato per il progetto: \textbf{Funx}.

\noindent Il nome nasce dall'unione dei due termini anglosassoni \textit{functional} e \textit{expression};
viene quindi pronunciato \textipa{["f2nIk"s]} in inglese,
\textipa{[fan\textperiodcentered èks]} o \textipa{[fan\textperiodcentered ìks]} in italiano.

\section{Linguaggi funzionali}
\label{sec:2-1-functional-languages}

Nonostante i linguaggi di programmazione non si possano confinare all'interno di un solo paradigma,
parlando di linguaggi di programmazione si fa comunque spesso riferimento a due grandi categorie:
linguaggi imperativi e linguaggi dichiarativi.


I primi hanno caratteristiche direttamente legate al modello di calcolo di \textit{John Von Neumann},
a sua volta non dissimile dalla macchina di \textit{Alan Turing}.
Questi linguaggi sono usati per codice che segue una precisa sequenza di istruzioni,
la quale descrive più o meno esplicitamente i passi necessari per risolvere il problema affrontato.

\noindent Appartengono alla famiglia dei linguaggi di programmazione imperativi sia linguaggi procedurali come
\texttt{Fortran}, \texttt{Cobol} e \texttt{Zig}, sia i linguaggi orientati agli oggetti, tra cui \texttt{Kotlin}, \texttt{C\#} e \texttt{Ruby}.


I linguaggi dichiarativi, invece, sono fondamentali per lo scopo del progetto:
tali linguaggi sono generalmente di altissimo livello e permettono allo sviluppatore
di concentrarsi sull'obiettivo da raggiungere piuttosto che sui dettagli implementativi.

\noindent Fanno parte di questa categoria linguaggi di interrogazione come \texttt{SQL},
linguaggi logici come \texttt{Prolog} e soprattutto i linguaggi funzionali:
\texttt{Lisp}, \texttt{Clojure}, \texttt{Elixir}, \texttt{OCaml} e \texttt{Haskell} sono alcuni esempi.


Alla base dei linguaggi funzionali vi è il \textbf{lambda calcolo} {\cite{Church-1932-FoundationLogic,Church-1933-FoundationLogic}}:
un sistema formale definito dal matematico \textit{Alonzo Church} (supervisore di \textit{Alan Turing} durante il dottorato),
equivalente alla macchina di Turing, ma fondato sulle funzioni pure.

\newpage

\noindent La grammatica del lambda calcolo verrà presentata poco più avanti (sezione~\ref{sec:2-3-syntax}),
ma le regole che ne governano il funzionamento e il modo in cui queste vengano utilizzate per ridurre
le espressioni ad una forma normale esulano dai fini di questo documento.

\noindent Rimane comunque rilevante elencare le principali qualità che un linguaggio funzionale
usualmente matura grazie al lambda calcolo:
\begin{itemize}
      \item \textbf{funzioni come entità di prima classe}: le funzioni possono essere passate come argomenti
            e restituite come risultato di altre funzioni;
      \item \textbf{immutabilità}: le variabili utilizzate sono immutabili;
      \item \textbf{purezza}: le funzioni sono libere da effetti collaterali
            (non modificano lo stato del programma) e restituiscono sempre lo stesso output per input identici;
      \item \textbf{ricorsione}: la ricorsione è il meccanismo più idiomatico per esprimere
            l'iterazione su una struttura dati.
\end{itemize}

\subsection{ML, Haskell e Funx}
\label{sec:2-2-ml-haskell-funx}

Nonostante le funzioni pure tipiche di un linguaggio funzionale siano un concetto molto attraente
dal punto di vista della correttezza della computazione, i vincoli così imposti possono risultare
stringenti a tal punto da rendere difficile, se non impossibile, la scrittura di programmi che
interagiscano con il mondo reale.


Per questo motivo, molti linguaggi funzionali permettono invece di utilizzare particolari funzioni
impure o di effettuare almeno operazioni di input/output. Inoltre, molti linguaggi prevalentemente
imperativi adottano ormai da tempo alcune caratteristiche tipiche dei linguaggi funzionali
(e.g. \texttt{Rust}, il linguaggio più amato%
\footnote{Stack Overflow Developer Survey 2023 (\url{https://survey.stackoverflow.co/2023}), \\
    \textit{Rust is the most admired language}}
dagli sviluppatori secondo i sondaggi di \textit{Stack Overflow},
eredita molto dal linguaggio con cui era scritto il suo primo compilatore, \texttt{OCaml}, ed è dotato quindi di
funzioni di prima classe, immutabilità di default, strutture dati algebriche, ecc.).


\texttt{ML} è un linguaggio funzionale sviluppato negli anni '70 presso l'Università di Edimburgo,
costituente la base per moltissimi dei linguaggi sviluppati in seguito.
\texttt{ML} permette effettivamente l'uso di funzioni impure, ma fra i suoi discendenti
vi è \texttt{Haskell}, uno dei pochi linguaggi invece completamente puri.

\noindent \texttt{Haskell} si avvale di un pattern di programmazione chiamato \textit{monadi} {\cite{Moggi-1991-ComputationMonads}}
per gestire le operazioni di input/output e altre impurità, mantenendo le funzioni pure.


Nell'ideare \textbf{Funx} l'ispirazione viene proprio da \texttt{Haskell}, ma è presente la possibilità
di dichiarare un'unica funzione impura (il cosiddetto \textit{main}) per permettere di visualizzare a schermo un risultato.
Il linguaggio non è quindi allo stesso livello di purezza di \texttt{Haskell}, e naturalmente non supporta
molte delle funzionalità più avanzate di quest'ultimo (come le \textit{classi di tipi} e il \textit{pattern matching}),
ma ne mutua altre comunque interessanti, tra cui l'uso di alcuni operatori infissi e il \textit{polimorfismo parametrico}.

\subsection{ML, Haskell e Funx}
\label{sec:2-2-ml-haskell-funx}

Nonostante le funzioni pure tipiche di un linguaggio funzionale siano un concetto molto attraente
dal punto di vista della correttezza della computazione, i vincoli così imposti possono risultare
stringenti a tal punto da rendere difficile, se non impossibile, la scrittura di programmi che
interagiscano con il mondo reale.


Per questo motivo, molti linguaggi funzionali permettono invece di utilizzare particolari funzioni
impure o di effettuare almeno operazioni di input/output. Inoltre, molti linguaggi prevalentemente
imperativi adottano ormai da tempo alcune caratteristiche tipiche dei linguaggi funzionali
(e.g. \texttt{Rust}, il linguaggio più amato%
\footnote{Stack Overflow Developer Survey 2023 (\url{https://survey.stackoverflow.co/2023}), \\
    \textit{Rust is the most admired language}}
dagli sviluppatori secondo i sondaggi di \textit{Stack Overflow},
eredita molto dal linguaggio con cui era scritto il suo primo compilatore, \texttt{OCaml}, ed è dotato quindi di
funzioni di prima classe, immutabilità di default, strutture dati algebriche, ecc.).


\texttt{ML} è un linguaggio funzionale sviluppato negli anni '70 presso l'Università di Edimburgo,
costituente la base per moltissimi dei linguaggi sviluppati in seguito.
\texttt{ML} permette effettivamente l'uso di funzioni impure, ma fra i suoi discendenti
vi è \texttt{Haskell}, uno dei pochi linguaggi invece completamente puri.

\noindent \texttt{Haskell} si avvale di un pattern di programmazione chiamato \textit{monadi} {\cite{Moggi-1991-ComputationMonads}}
per gestire le operazioni di input/output e altre impurità, mantenendo le funzioni pure.


Nell'ideare \textbf{Funx} l'ispirazione viene proprio da \texttt{Haskell}, ma è presente la possibilità
di dichiarare un'unica funzione impura (il cosiddetto \textit{main}) per permettere di visualizzare a schermo un risultato.
Il linguaggio non è quindi allo stesso livello di purezza di \texttt{Haskell}, e naturalmente non supporta
molte delle funzionalità più avanzate di quest'ultimo (come le \textit{classi di tipi} e il \textit{pattern matching}),
ma ne mutua altre comunque interessanti, tra cui l'uso di alcuni operatori infissi e il \textit{polimorfismo parametrico}.

\newpage

\section{Sintassi}
\label{sec:2-3-syntax}

La sintassi di \textbf{Funx} risulta molto simile a quella di \texttt{Haskell}, con poche differenze dovute
a tre principali motivi:
\begin{itemize}
    \item libera scelta di nomi e simboli per le parole chiave;
    \item necessità di successiva traduzione in \texttt{Java};
    \item difficoltà e scarso valore all'interno del progetto dell'implementazione di un parser dipendente dall'indentazione.
\end{itemize}

\noindent A prescindere da ciò, il cuore del linguaggio è lo stesso di ogni altro linguaggio derivato dal lambda calcolo:
la sua definizione si può agilmente comprendere visualizzando la grammatica del lambda calcolo e confrontandola con
quella (leggermente semplificata) di \textbf{Funx}, facendo attenzione alle regole aggiuntive.

\begin{figure}
    \vspace{4mm}
    \begin{bnf}
        $E$ : \small{Espressione} ::=
        | $x$ : \small{variabile}
        | $E_l\ E_r$ : \small{applicazione}
        | $\lambda x\ .\ E$ : \small{astrazione}
    \end{bnf}
    \caption{Grammatica del lambda calcolo}
    \label{fig:2-3-lambda-syntax}
    \vspace{4mm}
\end{figure}

\noindent Le tre regole in Figura~\ref{fig:2-3-lambda-syntax} indicano le tre componenti indispensabili
del lambda calcolo:
\begin{itemize}
    \item \textbf{variabile}: simbolo rappresentante un parametro;
    \item \textbf{applicazione}: applicazione di funzione ad un argomento (entrambi espressioni);
    \item \textbf{astrazione}: definizione di una funzione anonima, con un solo input $x$ (variabile vincolata)
          e un solo output $E$ (espressione); per definire funzioni con più
          parametri si debbono usare molteplici astrazioni annidate (tecnica detta \textit{currying}).
\end{itemize}

\newpage

\begin{figure}
    \begin{bnf}
        $M$ : \small{Modulo} ::=
        | $nome\ \cdot\ L$
        ;;
        $D$ : \small{Dichiarazione} ::=
        | $?(schema\ di\ tipo)\ \cdot\ id = E$ : \small{funzione}
        ;;
        $E$ : \small{Espressione} ::=
        | $c$ : \small{costante}
        | $x$ : \small{variabile}
        | $E_l\ E_r$ : \small{applicazione}
        | $\lambda x\ .\ E$ : \small{astrazione}
        | $L$ : \small{let}
        | $\textbf{if}\ E_c\ \textbf{then}\ E_t\ \textbf{else}\ E_e$ : \small{if}
        ;;
        $L$ : \small{Let} ::=
        | $\textbf{let}\ \cdot\ D\ (\cdot\ D)^*\ \cdot\ \textbf{in}\ E$
    \end{bnf}
    \caption{Grammatica di \textbf{Funx}}
    \label{fig:2-3-funx-syntax}
    \vspace{4mm}
\end{figure}

\noindent È facile constatare la presenza delle ulteriori produzioni per la definizione del modulo corrente
(informazione inclusa a prescindere dal fatto che il linguaggio ad ora non supporti l'importazione di moduli esterni
che non siano la libreria standard) e di funzioni con nome: lo \textit{schema di tipo} è un'informazione opzionale
relativa al tipo della funzione e di cui si parlerà più approfonditamente nella sezione~\ref{sec:3-3-system-fc}.

\noindent Per quanto riguarda invece le espressioni, vengono introdotte tre nuove regole:
\begin{itemize}
    \item \textbf{costante}: rappresenta un valore letterale, come un numero o una stringa;
    \item \textbf{let}: permette di avere dichiarazioni locali utilizzabili all'interno di un'espressione;
    \item \textbf{if}: la più classica istruzione condizionale controllata da un'espressione booleana.
\end{itemize}

\subsection{Zucchero sintattico}
\label{sec:2-4-syntactic-sugar}

Con lo scopo di rendere il codice più leggibile, conciso e semplice, \textbf{Funx} introduce
dello zucchero sintattico (del tutto simile a quello di \texttt{Haskell}).
In Tabella~\ref{tab:2-4-sugar} sono riportati l'indispensabile per evitare il parsing dell'indentazione,
le semplificazioni comuni utili all'arricchimento del lambda calcolo, e infine tutti gli operatori simbolici
supportati al momento (assieme alla notazione per indicarne associatività e precedenza).

\newpage

\begin{table}[H]
    \begin{center}
        \begin{tabularx}{\textwidth}{|P{15em}|X|}
            \hline
            \textbf{Zucchero}                & \textbf{Sostituzione}                                            \\
            \hline
            \texttt{$\backslash$x -> e}      & \texttt{$\lambda$x $\mathord{.}$ e}                              \\
            \hline
            \texttt{$\backslash$x y -> e}    & \texttt{$\lambda$x $\mathord{.}$ $\lambda$y $\mathord{.}$ e}     \\
            \hline
            \texttt{f x y = e}               & \texttt{f = $\lambda$x $\mathord{.}$ $\lambda$y $\mathord{.}$ e} \\
            \hline
            \texttt{let}                     &                                                                  \\
            \texttt{f1 = e1}                 & \texttt{let f1 = e1 $\cdot$ f2 = e2 in e3}                       \\
            \texttt{f2 = e2}                 &                                                                  \\
            \texttt{in e3}                   &                                                                  \\
            \hline
            \texttt{f3 = e3}                 &                                                                  \\
            \texttt{with}                    &                                                                  \\
            \texttt{f1 = e1}                 & \texttt{f3 = let f1 = e1 $\cdot$ f2 = e2 in e3}                  \\
            \texttt{f2 = e2}                 &                                                                  \\
            \texttt{out}                     &                                                                  \\
            \hline
            \texttt{main = e3}               &                                                                  \\
            \texttt{f1 = e1}                 & \texttt{main = let f1 = e1 $\cdot$ f2 = e2 in e3}                \\
            \texttt{f2 = e2}                 &                                                                  \\
            \hline
            \texttt{if b then e1 else e2 fi} & \texttt{if b then e1 else e2}                                    \\
            \hline
        \end{tabularx}
        % divide et impera because inconsistent tabbing is a thing
        \begin{tabularx}{\textwidth}{|P{7em}@{\quad}P{7em}|X|}
            \texttt{e1 $\mathord{.}$ e2} & \texttt{infixr 9} & \texttt{compose e1 e2}            \\
            \texttt{e1 / e2}             & \texttt{infixl 7} & \texttt{divide e1 e2}             \\
            \texttt{e1 \% e2}            & \texttt{infixl 7} & \texttt{modulo e1 e2}             \\
            \texttt{e1 * e2}             & \texttt{infixl 7} & \texttt{multiply e1 e2}           \\
            \texttt{e1 + e2}             & \texttt{infixl 6} & \texttt{add e1 e2}                \\
            \texttt{e1 - e2}             & \texttt{infixl 6} & \texttt{subtract e1 e2}           \\
            \texttt{e1 > e2}             & \texttt{infix 4}  & \texttt{greaterThan e1 e2}        \\
            \texttt{e1 >= e2}            & \texttt{infix 4}  & \texttt{greaterThanEquals e1 e2}  \\
            \texttt{e1 < e2}             & \texttt{infix 4}  & \texttt{lessThan e1 e2}           \\
            \texttt{e1 <= e2}            & \texttt{infix 4}  & \texttt{lessThanEquals e1 e2}     \\
            \texttt{e1 == e2}            & \texttt{infix 4}  & \texttt{equalsEquals e1 e2}       \\
            \texttt{e1 != e2}            & \texttt{infix 4}  & \texttt{notEquals e1 e2}          \\
            \texttt{!!e}                 & \texttt{prefix 4} & \texttt{not e}                    \\
            \texttt{e1 \&\& e2}          & \texttt{infixr 3} & \texttt{if e1 then e2 else False} \\
            \texttt{e1 || e2}            & \texttt{infixr 2} & \texttt{if e1 then True else e2}  \\
            \texttt{e1 \$ e2}            & \texttt{infixr 0} & \texttt{apply e1 e2}              \\
            \hline
        \end{tabularx}
    \end{center}
    \caption{Zucchero sintattico}
    \label{tab:2-4-sugar}
\end{table}

\newpage

\noindent Come già accennato, il Capitolo~\ref{chap:5-compiler} illustrerà come l'albero sintattico astratto (\textbf{AST})
di un programma viene ottenuto, annotato e tradotto in \texttt{Java}; la sezione~\ref{sec:4-2-ternary-operator}
esporrà invece il motivo della traduzione degli operatori booleani binari in if.

\noindent Alcuni esempi di funzioni sono presentati nel Codice~\ref{lst:2-4-example-funx};
seppur superflua, l'indentazione è inclusa per maggiore chiarezza.

\vspace{4mm}
\begin{lstlisting}[caption={Esempio di programma}, style=funxCode, label={lst:2-4-example-funx}]
main = factorial 20

factorial : Int @-> Int
factorial n = if n == 0 then 1 else n * factorial (n - 1) fi

even : Int @-> Bool
even = let
        even1 : Int @-> Bool
        even1 n = if n == 0 then True else odd (n - 1) fi

        odd : Int @-> Bool
        odd n = if n == 0 then False else even1 (n - 1) fi
    in even1

gcd : Int @-> Int @-> Int
gcd a b = if b == 0 then a else gcd b (a % b) fi

xor : Bool @-> Bool @-> Bool
xor a b = (a || b) && !!(a && b)
\end{lstlisting}

\subsection{Zucchero sintattico}
\label{sec:2-4-syntactic-sugar}

Con lo scopo di rendere il codice più leggibile, conciso e semplice, \textbf{Funx} introduce
dello zucchero sintattico (del tutto simile a quello di \texttt{Haskell}).
In Tabella~\ref{tab:2-4-sugar} sono riportati l'indispensabile per evitare il parsing dell'indentazione,
le semplificazioni comuni utili all'arricchimento del lambda calcolo, e infine tutti gli operatori simbolici
supportati al momento (assieme alla notazione per indicarne associatività e precedenza).

\newpage

\begin{table}[H]
    \begin{center}
        \begin{tabularx}{\textwidth}{|P{15em}|X|}
            \hline
            \textbf{Zucchero}                & \textbf{Sostituzione}                                            \\
            \hline
            \texttt{$\backslash$x -> e}      & \texttt{$\lambda$x $\mathord{.}$ e}                              \\
            \hline
            \texttt{$\backslash$x y -> e}    & \texttt{$\lambda$x $\mathord{.}$ $\lambda$y $\mathord{.}$ e}     \\
            \hline
            \texttt{f x y = e}               & \texttt{f = $\lambda$x $\mathord{.}$ $\lambda$y $\mathord{.}$ e} \\
            \hline
            \texttt{let}                     &                                                                  \\
            \texttt{f1 = e1}                 & \texttt{let f1 = e1 $\cdot$ f2 = e2 in e3}                       \\
            \texttt{f2 = e2}                 &                                                                  \\
            \texttt{in e3}                   &                                                                  \\
            \hline
            \texttt{f3 = e3}                 &                                                                  \\
            \texttt{with}                    &                                                                  \\
            \texttt{f1 = e1}                 & \texttt{f3 = let f1 = e1 $\cdot$ f2 = e2 in e3}                  \\
            \texttt{f2 = e2}                 &                                                                  \\
            \texttt{out}                     &                                                                  \\
            \hline
            \texttt{main = e3}               &                                                                  \\
            \texttt{f1 = e1}                 & \texttt{main = let f1 = e1 $\cdot$ f2 = e2 in e3}                \\
            \texttt{f2 = e2}                 &                                                                  \\
            \hline
            \texttt{if b then e1 else e2 fi} & \texttt{if b then e1 else e2}                                    \\
            \hline
        \end{tabularx}
        % divide et impera because inconsistent tabbing is a thing
        \begin{tabularx}{\textwidth}{|P{7em}@{\quad}P{7em}|X|}
            \texttt{e1 $\mathord{.}$ e2} & \texttt{infixr 9} & \texttt{compose e1 e2}            \\
            \texttt{e1 / e2}             & \texttt{infixl 7} & \texttt{divide e1 e2}             \\
            \texttt{e1 \% e2}            & \texttt{infixl 7} & \texttt{modulo e1 e2}             \\
            \texttt{e1 * e2}             & \texttt{infixl 7} & \texttt{multiply e1 e2}           \\
            \texttt{e1 + e2}             & \texttt{infixl 6} & \texttt{add e1 e2}                \\
            \texttt{e1 - e2}             & \texttt{infixl 6} & \texttt{subtract e1 e2}           \\
            \texttt{e1 > e2}             & \texttt{infix 4}  & \texttt{greaterThan e1 e2}        \\
            \texttt{e1 >= e2}            & \texttt{infix 4}  & \texttt{greaterThanEquals e1 e2}  \\
            \texttt{e1 < e2}             & \texttt{infix 4}  & \texttt{lessThan e1 e2}           \\
            \texttt{e1 <= e2}            & \texttt{infix 4}  & \texttt{lessThanEquals e1 e2}     \\
            \texttt{e1 == e2}            & \texttt{infix 4}  & \texttt{equalsEquals e1 e2}       \\
            \texttt{e1 != e2}            & \texttt{infix 4}  & \texttt{notEquals e1 e2}          \\
            \texttt{!!e}                 & \texttt{prefix 4} & \texttt{not e}                    \\
            \texttt{e1 \&\& e2}          & \texttt{infixr 3} & \texttt{if e1 then e2 else False} \\
            \texttt{e1 || e2}            & \texttt{infixr 2} & \texttt{if e1 then True else e2}  \\
            \texttt{e1 \$ e2}            & \texttt{infixr 0} & \texttt{apply e1 e2}              \\
            \hline
        \end{tabularx}
    \end{center}
    \caption{Zucchero sintattico}
    \label{tab:2-4-sugar}
\end{table}

\newpage

\noindent Come già accennato, il Capitolo~\ref{chap:5-compiler} illustrerà come l'albero sintattico astratto (\textbf{AST})
di un programma viene ottenuto, annotato e tradotto in \texttt{Java}; la sezione~\ref{sec:4-2-ternary-operator}
esporrà invece il motivo della traduzione degli operatori booleani binari in if.

\noindent Alcuni esempi di funzioni sono presentati nel Codice~\ref{lst:2-4-example-funx};
seppur superflua, l'indentazione è inclusa per maggiore chiarezza.

\vspace{4mm}
\begin{lstlisting}[caption={Esempio di programma}, style=funxCode, label={lst:2-4-example-funx}]
main = factorial 20

factorial : Int @-> Int
factorial n = if n == 0 then 1 else n * factorial (n - 1) fi

even : Int @-> Bool
even = let
        even1 : Int @-> Bool
        even1 n = if n == 0 then True else odd (n - 1) fi

        odd : Int @-> Bool
        odd n = if n == 0 then False else even1 (n - 1) fi
    in even1

gcd : Int @-> Int @-> Int
gcd a b = if b == 0 then a else gcd b (a % b) fi

xor : Bool @-> Bool @-> Bool
xor a b = (a || b) && !!(a && b)
\end{lstlisting}