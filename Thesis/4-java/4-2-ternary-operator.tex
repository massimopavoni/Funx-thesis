\section{Operatore ternario}
\label{sec:4-2-ternary-operator}

Una delle peculiarità della traduzione da \textbf{Funx} a \texttt{Java} è l'uso dell'operatore ternario
(\texttt{condition ? thenBranch : elseBranch}) ogni qualvolta
siano presenti espressioni condizionali \texttt{if-then-else}.

I linguaggi funzionali sfruttano spesso una caratteristica (non menzionata nella sezione~\ref{sec:2-1-functional-languages})
che prende il nome di \textit{lazy evaluation} (valutazione pigra): fino a quando il risultato di un'espressione
non è richiesto per un successivo calcolo, questa non verrà completamente valutata.

\noindent Oltre ad offrire molteplici possibilità di ottimizzazione dal punto di vista del tempo di esecuzione,
tale comportamento è molto comodo nella scrittura di funzioni che per esempio potrebbero terminare
prima del previso o magari effettuare computazioni su strutture dati infinite.
I linguaggi che sono \textit{lazy evaluated} di default impegano nella maggior parte dei casi un \textit{garbage collector}
per liberare la memoria occupata dalle espressioni non valutate e non più rilevanti.

Il linguaggio \texttt{Java} non adotta la \textit{lazy evaluation} di default se non in casi particolari, tra cui
gli operatori booleani binari, il costrutto \texttt{if-then-else} (e corrispondente operatore ternario) e altre
funzionalità più avanzate tra cui gli \textit{stream} e le \textit{lambda espressioni} già viste.
Utilizzando quest'ultime si potrebbero ottenere risultati simili, in termini di valutazione pigra, a quelli di un linguaggio funzionale;
tuttavia, rendere \textbf{Funx} un linguaggio completamente pigro avrebbe comportato una traduzione indubbiamente ancora più complessa,
molteplici rischi di peggiorare le prestazioni dei programmi e un'implementazione del compilatore che va oltre lo scopo di questo progetto.

Nonostante ciò, la scelta di ridurre gli operatori booleani binari (\textit{and} e \textit{or}) ad espressioni con operatore ternario
è stata considerata quasi obbligatoria per conservarne la natura pigra: gli operatori ternari utilizzati a questo scopo
derivano direttamente dalla costruzione dell'\textbf{AST} (sezione~\ref{sec:5-?-ast}), motivo per cui non vengono riconvertiti
in operatori nativi di \texttt{Java} in fase di traduzione.

\vspace{4mm}
\begin{lstlisting}[caption={If e operatori booleani in \textbf{Funx}}, style=funxCode, label={lst:4-ternary-funx}]
power : Int @-> Int @-> Int
power b e = if e == 0 then 1 else b * power b (e - 1) fi

xor : Bool @-> Bool @-> Bool
xor a b = (a || b) && !!(a && b)
\end{lstlisting}
\vspace{4mm}
\begin{lstlisting}[caption={Corrispondenti operatori ternari in \texttt{Java}}, style=javaCode, label={lst:4-ternary-java}]
public static Function@<Long, Function@<Long, Long@>@> power;

public static Function@<Boolean, Function@<Boolean, Boolean@>@> xor;

static {
    power = (b @-> (e @-> ((JavaPrelude.@<Long@>equalsEquals().apply(e).apply(0L))
        @? (1L)
        @: (multiply.apply(b)
            .apply(power.apply(b).apply(subtract.apply(e).apply(1L)))))));

    xor = (a @-> (b @->
        ((((a) @? (true) @: (b))) @? (not.apply(((a) @? (b) @: (false)))) @: (false))));
}
\end{lstlisting}
