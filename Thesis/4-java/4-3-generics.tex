\section{Tipi generici}
\label{sec:4-3-generics}

Il sistema di tipo di \textbf{Funx} necessita la traduzione di funzioni polimorfe,
e la soluzione più semplice e idiomatica in \texttt{Java} è l'utilizzo dei \textit{generics}:
tramite i parametri di tipo generici è possibile definire classi e metodi che agiscono su molteplici tipi di dati,
implementando comportamenti che possono essere condivisi dai diversi elementi del dominio di tipi delle funzioni rappresentate.


Nel contesto del \textit{sistema HM} di \textbf{Funx}, le variabili quantificate universalmente nei politipi
hanno una diretta corrispondenza con i parametri di tipo che possono essere dichiarati tra i modificatori di visibilità
e il tipo di ritorno di un metodo, il quale a sua volta combacia con la parte interna dello schema di tipo.


Nei Codici~\ref{lst:4-3-generics-funx} e \ref{lst:4-3-generics-java} si può notare come \texttt{Java} non sia in grado
di inferire i tipi desiderati per le funzioni polimorfe: queste devono infatti essere istanziate esplicitamente
usando la classe di appartenenza e le parentesi angolari (questa limitazione richiederà alcuni espedienti in casi limite
illustrati nella sezione~\ref{sec:5-?-type-casting}).

\vspace{4mm}
\begin{lstlisting}[caption={Scrittura e utilizzo di funzioni polimorfe in \textbf{Funx}}, style=funxCode, label={lst:4-3-generics-funx}]
sumToN : Int @-> Int
sumToN = let
        ap : (a @-> b @-> c) -> (a @-> b) @-> a @-> c
        ap op f x = op x (f x)
    in (flip divide 2) . ap multiply (add 1)
\end{lstlisting}
\vspace{4mm}
\begin{lstlisting}[caption={Corrispondente traduzione in \texttt{Java}}, style=javaCode, label={lst:4-3-generics-java}]
public static Function@<Long, Long@> sumToN;

static {
    sumToN = (new Let@<@>() {
        private @<a, b, c@>
            Function@<
                Function@<a, Function@<b, c@>@>,
                Function@<Function@<a, b@>, Function@<a, c@>@>@>
                    ap() {
            return (op @-> (f @-> (x @-> op.apply(x).apply(f.apply(x)))));
        }

        @Override
        public Function@<Long, Long@> _eval() {
            return FunxPrelude.@<Long, Long, Long@>compose()
                .apply(FunxPrelude.@<Long, Long, Long@>flip().apply(divide).apply(2L))
                .apply(this.@<Long, Long, Long@>ap().apply(multiply).apply(add.apply(1L)));
        }
    })._eval();
}  
\end{lstlisting}
