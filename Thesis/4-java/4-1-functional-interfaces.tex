\section{Interfacce funzionali}
\label{sec:4-1-functional-interfaces}

Nel tradurre un linguaggio funzionale viene naturale pensare immediatamente alle \textit{interfacce funzionali} e \textit{lambda espressioni}
introdotte in \texttt{Java} 8%
\footnote{OpenJDK 8 (\url{https://openjdk.org/projects/jdk8})}
per rappresentare funzioni anonime: l'interfaccia generica \texttt{Function} è la più adatta a riprodurre
il comportamento dell'astrazione, come mostrato nei Codici~\ref{lst:4-1-function-funx} e \ref{lst:4-1-function-java}.

\vspace{4mm}
\begin{lstlisting}[caption={Semplice funzione in \textbf{Funx}}, style=funxCode, label={lst:4-1-function-funx}]
constant : a @-> b @-> a
constant x = \y @-> x
\end{lstlisting}
\vspace{4mm}
\begin{lstlisting}[caption={Corrispondente metodo in \texttt{Java}}, style=javaCode, label={lst:4-1-function-java}]
public static @<a, b@> Function@<a, Function@<b, a@>@> constant() {
    return (x @-> (y @-> x));
}
\end{lstlisting}
\vspace{4mm}

\noindent Dato il naturale \textit{currying} di oggetti \texttt{Function}, questo tipo di traduzione ha il vantaggio
di permettere l'applicazione parziale di funzioni (tramite una sequenza di \texttt{apply()} in numero minore rispetto al totale degli input),
ma il grande svantaggio della creazione di una nuova istanza della funzione per ogni chiamata.


Poiché \texttt{constant} ha tipo polimorfo (sezione~\ref{sec:4-3-generics}), il metodo utilizza parametri di tipo
e deve quindi necessariamente restituire un nuovo oggetto con ogni chiamata:
nonostante la performance delle traduzioni non sia un obiettivo primario del progetto, la versione attuale
del compilatore fa uso di alcune piccole ottimizzazioni nella trasposizione delle funzioni monomorfe,
approfondite nella sezione~\ref{sec:5-?-optimizations}.

\newpage

\noindent In aggiunta alla classe \texttt{Function}, nella parte nativa (codice \texttt{Java}) della libreria standard
di \textbf{Funx} è definita un'ulteriore interfaccia funzionale per creare espressioni \texttt{let}:
in questo caso vi è un grande utilizzo di classi anonime, potenzialmente annidate,
rappresentanti le dichiarazioni locali e l'espressione principale.

\vspace{4mm}
\begin{lstlisting}[caption={Interfaccia funzionale per espressioni \texttt{let}}, style=javaCode, label={lst:4-1-let-interface-java}]
@FunctionalInterface
public interface Let@<T@> {
    T _eval();
}
\end{lstlisting}
\vspace{4mm}
\begin{lstlisting}[caption={Espressione \texttt{let} in \textbf{Funx}}, style=funxCode, label={lst:4-1-let-funx}]
hundredsSum : Int @-> Int @-> Int
hundredsSum = let
        on : (a @-> a @-> b) @-> (c @-> a) @-> c @-> c @-> b
        on op f x y = op (f x) (f y)
    in on add (multiply 100)
\end{lstlisting}
\vspace{4mm}
\begin{lstlisting}[caption={Corrispondente classe anonima in \texttt{Java}}, style=javaCode, label={lst:4-1-let-java}]
public static Function@<Long, Function@<Long, Long@>@> hundredsSum;

static {
    hundredsSum = (new Let@<@>() {
        private @<a, b, c@>
            Function@<
                    Function@<a, Function@<a, b@>@>,
                    Function@<Function@<c, a@>, Function@<c, Function@<c, b@>@>@>@>
                on() {
            return (op @-> (f @-> (x @-> (y @-> op.apply(f.apply(x)).apply(f.apply(y))))));
        }

        @Override
        public Function@<Long, Function@<Long, Long@>@> _eval() {
            return this.@<Long, Long, Long@>on().apply(add).apply(multiply.apply(100L));
        }
    })._eval();
}
\end{lstlisting}
\vspace{4mm}

\noindent Nei Codici~\ref{lst:4-1-let-funx} e \ref{lst:4-1-let-java} si può vedere che la funzione \texttt{hundredsSum}
è implementata attraverso la chiamata al metodo principale dell'interfaccia funzionale \texttt{Let},
realizzato internamente alla classe anonima con il supporto del metodo polimorfo \texttt{on}.

\noindent Inoltre, è immediatamente evidente come le traduzioni in \texttt{Java} siano progressivamente più complesse e meno leggibili
con l'introduzione di nuove funzionalità: l'esempio più eclatante è dato proprio dalla \textit{signature} del metodo locale \texttt{on},
divenuta di difficile comprensione rispetto alla sintassi molto concisa del linguaggio funzionale.