\chapter{\localized{Compilatore}{Compiler}}
\label{chap:5-compiler}

Il software sviluppato per il progetto è stato sviluppato mantenendo il codice sul repository \texttt{GitHub} \textbf{Funx-jt}%
\footnote{massimopavoni/Funx-jt (\url{https://github.com/massimopavoni/Funx-jt})},
nel cui nome \textit{"jt"} è l'acronimo per \textit{"Java Transpiler"}.

\noindent La prima versione stabile è disponibile sul repository (\texttt{Funx-jt-0.1.0});
la versione di \texttt{Java} utilizzata è la più recente \textit{Long Term Support (LTS)},
versione 21 di OpenJDK%
\footnote{OpenJDK 21 (\url{https://openjdk.org/projects/jdk/21})}.

Nel corso di questo capitolo si discuteranno le fasi di compilazione e la struttura del software,
analizzando nel dettaglio le parti più importanti.

\section{ANTLR}
\label{sec:5-1-antlr}

Al fine di semplificare lo sviluppo di \textit{lexer} e \textit{parser} per il linguaggio funzionale ideato
è stato scelto il generatore di \textit{parser} chiamato \texttt{ANTLR}%
\footnote{ANother Tool for Language Recognition;
    \citetitle{Parr-1995-ANTLRGenerator} \cite{Parr-1995-ANTLRGenerator},
    \citetitle{Parr-2013-DefinitiveANTLR} \cite{Parr-2013-DefinitiveANTLR}
    and ANTLRv4 (\url{https://www.antlr.org})}.

\noindent Grazie a tale strumento il processo iterativo di creazione della grammatica di \textbf{Funx}
è stato notevolmente semplificato e accelerato, in quanto \texttt{ANTLR} mette a disposizione del programmatore
un linguaggio per definire uno o più file di specifica per lessico e sintassi
(directory \texttt{Funx-jt/src/main/antlr} nel repository): questi vengono poi processati
per generare il codice sorgente del \textit{lexer} e del \textit{parser}.

\subsection{Analisi lessicale}
\label{sec:5-2-lexical-analysis}

Data la probabile complessità delle regole della grammatica di \textbf{Funx}, fin dall'inizio la definizione dei \textit{token} (lessemi)
del linguaggio è stata separata dalla specifica del \textit{parser}.

\noindent Il file \texttt{FunxLexer.g4} descrive i lessemi dividendoli nelle seguenti cateogorie:
\begin{enumerate}
    \item \textit{whitespace}: caratteri di spaziatura e tabulazione;
    \item \textit{comments}: commenti di linea e blocco;
    \item \textit{keywords}: parole chiave del linguaggio;
    \item \textit{Java keywords}: parole chiave del linguaggio Java, da evitare;
    \item \textit{types}: tipi di dato (funzioni di tipo con arità 0);
    \item \textit{literals}: costanti booleane e numeriche;
    \item \textit{variables}: identificatori per variabili di tipo o nomi di funzioni;
    \item \textit{module}: identificatori il modulo;
    \item vari operatori simbolici per:
          \begin{itemize}
              \item \textit{bool}: valori booleani;
              \item \textit{comparison}: confronti tra numeri;
              \item \textit{arithmetic}: operazioni aritmetiche;
              \item \textit{other symbols}: simboli della sintassi (come \texttt{->}) e varie funzioni di libreria;
              \item \textit{delimiters}: parentesi tonde, quadre e graffe.
          \end{itemize}
\end{enumerate}

\noindent Le categorie 1 e 2 contengono token da scartare, tranne \texttt{NEWLINE}, mentre la categoria 4 è utile
qualora eventualmente si permetta allo sviluppatore di utilizzare tali parole chiave riservate,
effettuando una rinomina automatica; le categorie 7 e 8 devono necessariamente apparire dopo le categorie 3 e 4,
poiché tra \textit{keyword} e identificatori di ogni genere le prime devono avere la precedenza
(la posizione della categoria 5 tiene conto di una possibile futura estensione per consentire la creazione di nuovi tipi).

Oltre alle categorie illustrate, in testa al file sono presenti dei cosiddetti \textit{fragment} (frammenti)
che semplificano le espressioni regolari dei \textit{token} e complessivamente aumentano la leggibilità della specifica.

\newpage

\begin{lstlisting}[caption={Alcune \textit{token} del \textit{lexer}}, style=antlrCode, label={lst:5-lexer}]
lexer grammar FunxLexer;

// Fragments
fragment LALPHA: [a-z];
fragment UALPHA: [A-Z];
fragment ALPHA: LALPHA | UALPHA;
fragment ALPHA_: ALPHA | UnderScore;

fragment DIGIT: [0-9];
fragment DECIMAL: DIGIT+;

// Whitespace
NEWLINE: '\r'? '\n' | '\r';

TAB: [\t]+ @-> skip;
WS: [\u0020\u00a0\u1680\u2000\u200a\u202f\u205f\u3000]+ @-> skip;

// Comments
CloseMultiComment: '\./';
OpenMultiComment: '/\.';
SingleComment: '//';

COMMENT: SingleComment ~[\r\n]* @-> skip;
MULTICOMMENT: OpenMultiComment .*? CloseMultiComment @-> skip;

// Keywords
ELSE: 'else';
FI: 'fi';
IF: 'if';
IN: 'in';
LET: 'let';

// Java keywords
RESERVED_JAVA_KEYWORD: 'abstract' | 'assert' | 'boolean' | 'break' | 'byte' | [\.\.\.];

// Types
TYPE: BOOLTYPE | INTTYPE;
BOOLTYPE: 'Bool';

// Literals
INT: DECIMAL | OpenParen '\-' DECIMAL CloseParen;

// Variables
VARID: LALPHA (ALPHA_ | DIGIT)*;

// Module
MODULEID: UALPHA (ALPHA_ | DIGIT)*;

// Bool
And: '&&';
Not: '!!';

// Comparison
EqualsEquals: '\=\=';
NotEquals: '!\=';

// Arithmetic
Add: '\+';

// Other symbols
UnderScore: '_';
Arrow: '\-\>';

// Delimiters
OpenParen: '\(';
CloseParen: '\)';
\end{lstlisting}

\subsection{Analisi sintattica}
\label{sec:5-3-syntactic-analysis}

Il file \texttt{FunxParser.g4} contiene le regole concrete della grammatica di \textbf{Funx}:
nonostante la somiglianza con le grammatiche delle Figure~\ref{fig:2-3-funx-syntax}~e~\ref{fig:3-3-system-hm},
è evidente che queste non collimino esattamente a causa di zucchero sintattico e requisiti di \texttt{ANTLR}.

\noindent Lo strumento utilizzato, infatti, è un generatore di \textit{parser} di tipo \textit{top-down}
per grammatiche \textit{LL}, le quali in generale non supportano regole ricorsive a sinistra.

\noindent Essendo tali regole spesso comuni nella definizione di qualsiasi linguaggio di programmazione, \textbf{Funx} incluso,
\texttt{ANTLRv4} offre un diverso tipo di parsing, detto \textit{Adaptive LL(*)} \cite{Parr-2011-FoundationANTLR,Parr-2014-AdaptiveLL}:
quest'ultimo è in grado di riscrivere automaticamente le grammatiche, eliminando la ricorsione a sinistra diretta (e.g. linee 36 e 38-43),
così da non incorrere in regole ambigue che potrebbero causare \textit{backtracking} e conseguente \textit{overhead}.

\noindent Il Codice~\ref{lst:5-3-parser-antlr} riporta integralmente le regole concrete della sintassi di \textbf{Funx}, tra cui:
\begin{itemize}
    \item \textit{module}: nome del modulo, funzione \texttt{main} opzionale e dichiarazioni globali;
    \item \textit{main}: funzione \texttt{main}, diversa dalle dichiarazioni classiche per l'assenza di schema di tipo e parametri lambda;
    \item \textit{declaration}: funzione con nome, tipo e parametri (e opzionalmente \texttt{with} per funzioni locali);
    \item \textit{typeElems}: tipo di una funzione, definito ricorsivamente secondo la grammatica del sistema di tipo di \textbf{Funx};
    \item \textit{statement}: per evitare ricorsione a sinistra indiretta, la separazione tra \textit{statement} ed \textit{expression}
          forza l'uso di parentesi nei casi in cui lambda astrazioni, let e if siano usati all'interno di un'espressione;
    \item \textit{expression}: racchiude l'applicazione funzionale, tutte le regole relative agli operatori simbolici,
          specificandone la priorità implicita (Tabella~\ref{tab:2-4-sugar}), e le espressioni primarie
          (costanti, variabili e parentesi per controllare la precedenza);
    \item \textit{lambda, let, if}: corrispondenti alle produzioni per astrazione, let e if della grammatica formale.
\end{itemize}

\newpage

\begin{lstlisting}[caption={Grammatica per il \textit{parser}}, style=antlrCode, label={lst:5-3-parser-antlr}]
parser grammar FunxParser;
options { tokenVocab = FunxLexer; }

// Module
module: (MODULE MODULEID (Dot MODULEID)* NEWLINE+)?
    (main NEWLINE+)? declarations EOF;

declarations: declaration (NEWLINE declaration?)*;

main: id = MAIN Equals statement with?;

// Declaration
declaration: (declarationScheme NEWLINE)?
    id = VARID lambdaParams? Equals statement with?;

declarationScheme: id = VARID Colon typeElems;

with: NEWLINE WITH localDeclarations OUT;

localDeclarations: NEWLINE declarations NEWLINE;

// Type
typeElems: OpenParen typeElems CloseParen %# parenType%
    | VARID %# typeVar%
    | TYPE %# namedType%
    | <assoc = right> typeElems Arrow typeElems %# arrowType%;

// Statement
statement: expression %# expressionStatement%
    | lambda %# lambdaStatement%
    | let %# letStatement%
    | ifS %# ifStatement%;

// Expression
expression: primary %# primExpression%
    | expression expression %# appExpression%
    | <assoc = right> expression bop = Dot expression %# composeExpression%
    | expression bop = (Divide | Modulo | Multiply) expression %# divModMultExpression%
    | expression bop = (Add | Subtract) expression %# addSubExpression%
    | expression
        bop = (GreaterThan | GreaterThanEquals | LessThan | LessThanEquals)
        expression %# compExpression%
    | expression bop = (EqualsEquals | NotEquals) expression %# eqExpression%
    | uop = Not expression %# notExpression%
    | <assoc = right> expression bop = And expression %# andExpression%
    | <assoc = right> expression bop = Or expression %# orExpression%
    | <assoc = right> expression bop = Dollar expression %# rightAppExpression%;

primary: OpenParen statement CloseParen %# parenPrimary%
    | constant %# constPrimary% | VARID %# varPrimary%;

// Lambda
lambda: Backslash lambdaParams? Arrow statement;

lambdaParams: VARID+;

// Let
let: LET localDeclarations IN statement;

// If
ifS: IF statement THEN statement ELSE statement FI;

// Constant
constant: BOOL | numConstant;

numConstant: INT;    
\end{lstlisting}

\section{Albero sintattico astratto}
\label{sec:5-4-abstract-syntax-tree}

\subsection{Gerarchia delle classi}
\label{sec:5-5-class-hierarchy}

Il primo passo per la costruzione di un \textbf{AST} per la sintassi di \textbf{Funx} è la definizione
di una gerarchia di classi \texttt{Java} che rappresentano i nodi dell'albero
(package \texttt{com.github.massimopavoni.funx.jt.ast.node}).


La classe astratta \texttt{ASTNode} è la radice dell'ordinamento poiché sarà usata per gli oggetti creati
a partire dal \textit{CST}: contiene la proprietà \texttt{inputPosition} per facilitare la segnalazione di errori
e vincola le classi figlie all'implementazione del metodo astratto \texttt{accept} per visitare i nodi.

\noindent Un'altra classe astratta, derivata dalla precedente, è \texttt{Expression}, la quale identifica
un'espressione ed è dotata di alcuni campi e metodi utili all'inferenza di tipo.


Ogni altra sottoclasse (ad eccezione di \texttt{Declarations}, utilizzata per dichiarazioni globali e locali)
è una trascrizione in \texttt{Java} delle produzioni della grammatica formale:
\begin{itemize}
    \item \texttt{Module}: modulo del programma;
    \item \texttt{Declaration}: dichiarazione di funzione;
    \item \texttt{Constant}: termini costanti;
    \item \texttt{Variable}: simboli per variabili;
    \item \texttt{Application}: applicazione di funzione;
    \item \texttt{Lambda}: astrazione per le funzioni anonime;
    \item \texttt{Let}: contenitore di dichiarazioni locali;
    \item \texttt{If}: costrutto condizionale.
\end{itemize}

\newpage

\begin{figure}
    \begin{tikzpicture}
        \umlsimpleclass[type=abstract]{ASTNode}
        \umlsimpleclass[x=4,type=interface]{Inferable}
        \umlsimpleclass[x=-4,y=-2]{Module}
        \umlsimpleclass[y=-2]{Declarations}
        \umlsimpleclass[x=4,y=-2,type=abstract]{Expression}
        \umlsimpleclass[x=-2,y=-3.5]{Constant}
        \umlsimpleclass[x=-2,y=-5]{Variable}
        \umlsimpleclass[x=-2,y=-6.5]{Application}
        \umlsimpleclass[x=-2,y=-8]{Lambda}
        \umlsimpleclass[x=-2,y=-9.5]{Let}
        \umlsimpleclass[x=-2,y=-11]{If}
        \umlinherit[geometry=|-]{Module}{ASTNode}
        \umlinherit{Declarations}{ASTNode}
        \umlinherit[geometry=-|-]{Expression}{ASTNode}
        \umlimpl{Expression}{Inferable}
        \umlinherit[geometry=-|,anchor2=-160]{Constant}{Expression}
        \umlinherit[geometry=-|,anchor2=-149]{Variable}{Expression}
        \umlinherit[geometry=-|,anchor2=-117]{Application}{Expression}
        \umlinherit[geometry=-|,anchor2=-63]{Lambda}{Expression}
        \umlinherit[geometry=-|,anchor2=-31]{Let}{Expression}
        \umlinherit[geometry=-|,anchor2=-20]{If}{Expression}
    \end{tikzpicture}
    \caption{Diagramma semplificato delle classi dell'\textbf{AST}}
    \label{fig:5-ast-classes}
    \vspace{4mm}
\end{figure}

\begin{lstlisting}[caption={Esempio di classe della gerarchia}, style=javaCode, label={lst:5-class-example-java}]
public static final class Lambda extends Expression {
    public final String paramId;

    public final Expression expression;

    public Lambda(InputPosition inputPosition, String paramId, ASTNode expression) {
        super(inputPosition); // ASTNode constructor
        this.paramId = paramId;
        this.expression = (Expression) expression;
    }

    @Override // from Inferable interface
    public Utils.Tuple<Substitution, Type> infer(Context ctx) { ... }

    @Override // from Expression abstract class
    protected void propagateSubstitution(Substitution substitution) { ... }

    @Override // from ASTNode abstract class
    public @<T@> T accept(ASTVisitor<? extends T> visitor) {
        return visitor.visitLambda(this);
    }
}
\end{lstlisting}



\subsection{AST builder}
\label{sec:5-6-ast-builder}



\section{Motore inferenziale}
\label{sec:5-7-inference-engine}

I sistemi di tipo e l'inferenza descritti nel Capitolo~\ref{chap:3-inference} costituiscono una parte imprescindibile del software,
data l'incapacità di simulare il polimorfismo parametrico del \textit{sistema HM} lasciando al compilatore \texttt{Java}
il compito di inferire e controllare i tipi dei metodi (o classi) generici.
In tal caso, infatti, la compilazione fallirebbe a causa di tipi "troppo generici" e \textit{type casting} non permesso
in maniera implicita (e.g. non si è in grado di conciliare ad esempio un tipo generico con una lambda espressione).

\begin{figure}
    \vspace{4mm}
    \includegraphics[width=\textwidth]{5-7-impossible-java.png}
    \caption{Esempio di errore in \texttt{Java} dovuto a tipi generici}
    \label{fig:5-7-impossible-java}
    \vspace{4mm}
\end{figure}

\noindent La linea di codice citata in Figura~\ref{fig:5-7-impossible-java} non è intrinsecamente errata,
ma non può essere compilata in mancanza di informazioni riguardanti il parametro di tipo generico della funzione \texttt{id}
(con tipo di ritorno \texttt{Function<T, T>}): il problema da risolvere è quindi conoscere il tipo dell'espressione
in input alla prima chiamata al metodo \texttt{apply}.


Il motore inferenziale è la parte del compilatore che si occupa di stabilire i tipi delle espressioni seguendo le regole
di inferenza e i passi dell'\textit{algoritmo $\mathcal{W}$}%
\footnote{\citetitle{Grabmuller-2006-AlgorithmW} \cite{Grabmuller-2006-AlgorithmW}}
descritti nella sezione~\ref{sec:3-4-hm-type-inference}; la fase di inferenza avviene subito dopo il parsing
e la costruzione dell'\textbf{AST} e precede la generazione del codice \texttt{Java}.

\subsection{Sistema HM}
\label{sec:5-8-system-hm}


\newpage

\subsection{Inferenza su espressioni}
\label{sec:5-9-expression-inference}


\section{Traduzione in Java}
\label{sec:5-10-java-translation}

La sottoclasse più importante di \texttt{ASTVisitor} è \texttt{JavaTranspiler}, il \textit{visitor} che si occupa
dell'ultimo stadio di compilazione, la traduzione dell'\textbf{AST} in codice \texttt{Java}:
come per \texttt{GraphvizBuilder}, la classe compone una stringa
che rappresenta il programma \texttt{Java} corrispondente al codice \textbf{Funx} sorgente.


Avendo già illustrato alcuni esempi di traduzione nel Capitolo~\ref{chap:4-java}, in questa sezione si
discuteranno le scelte e i compromessi nel processo di traduzione, e il modo in cui le limitazioni
di \texttt{Java} possano essere talvolta aggirate "piegando" le regole.

\subsection{Membri statici}
\label{sec:5-11-static-members}

Il paradigma dichiarativo dei linguaggi funzionali è ben diverso dalla programmazione ad oggetti di molti altri linguaggi rinomati,
motivo per cui la scelta di tradurre ogni programma \textbf{Funx} in un'unica classe statica è vista come semplice soluzione
per evitare complicanze e \textit{overhead} per la creazione di oggetti in aggiunta alle \texttt{Function}.


Ogni funzione definita diviene perciò una proprietà statica della classe in caso di monotipi (sezione~\ref{sec:5-13-monomorphic-declarations})
o un metodo statico con parametri di tipo in caso di politipi (sezione~\ref{sec:5-14-polymorphic-functions-instantiation}).
Fanno eccezione le funzioni appartenenti a espressioni \texttt{let} annidate: essendo classi anonime, queste creano
un unico oggetto con proprietà e metodi privati (accessibili solamente al metodo pubblico \texttt{eval} della classe \texttt{Let}).

\noindent La traduzione inizia con \textit{import} statici per la libreria standard e un costruttore privato.

\vspace{4mm}
\begin{lstlisting}[caption={Prime aggiunte alla stringa \texttt{Java}}, style=javaCode, label={lst:5-11-first-append-java}]
// append package, imports, class declaration and constructor
builder.append(module.packageName.isEmpty()
        @? ""
        @: String.format("package %s;%n", module.packageName.toLowerCase()))
    .append("\n\nimport ").append(Function.class.getName())
    .append(";\n\nimport ").append(JavaPrelude.class.getName())
    .append(";\n\nimport ").append(FunxPrelude.class.getName())
    .append(";\n\nimport static ").append(JavaPrelude.class.getName())
    .append(".*;\nimport static ").append(FunxPrelude.class.getName())
    .append(".*;\n\npublic class ").append(module.name).append(" {\n")
    .append("private ").append(module.name)
    .append("() {\n// private constructor to prevent instantiation\n}\n\n");
\end{lstlisting}
\vspace{4mm}
\begin{lstlisting}[caption={Corrispondente codice \texttt{Java} generato}, style=javaCode, label={lst:5-11-class-start-java}]
import java.util.function.Function;

import com.github.massimopavoni.funx.lib.JavaPrelude;

import com.github.massimopavoni.funx.lib.FunxPrelude;

import static com.github.massimopavoni.funx.lib.JavaPrelude.*;
import static com.github.massimopavoni.funx.lib.FunxPrelude.*;

public class Chapter5Header {
    private Chapter5Header() {
        // private constructor to prevent instantiation
    }

    // ...
}
\end{lstlisting}

\newpage

\subsection{Lista dei contesti}
\label{sec:5-12-scope-stack}

\subsection{Dichiarazioni monomorfe}
\label{sec:5-13-monomorphic-declarations}

La possibilità in \textbf{Funx} di utilizzare funzioni ricorsive (e/o mutuamente ricorsive)
e la volontà di evitare la traduzione in metodi quando possibile sono in conflitto
a causa di \textit{Illegal Self Reference} e \textit{Illegal Forward Reference}:
tali errori si presentano durante la compilazione del codice \texttt{Java}
qualora i campi statici che identificano funzioni monomorfe vengano dichiarati e inizializzati
nello stesso \textit{statement} (stessa linea).


La dichiarazione delle proprietà deve avvenire prima dell'inizializzazione di altre variabili che ne fanno uso;
si potrebbe effettuare un'analisi iniziale dell'\textbf{AST} per identificare le dipendenze tra le funzioni
(approccio di ordinamento topologico estremamente utile anche per l'inferenza), ma la soluzione adottata
è di più semplice implementazione.


Come si può notare nel Codice~\ref{lst:5-13-monomorphic-java} e in alcuni esempi già presentati in precedenza,
si effettua la dichiarazione di ogni campo, pubblico e statico per le dichiarazioni globali, privato per quelle locali,
e solo successivamente si inizializzano rispettivamente con blocco statico e metodo \texttt{eval}.

\vspace{4mm}
\begin{lstlisting}[caption={Esempio di traduzione per funzioni monomorfe}, style=javaCode, label={lst:5-13-monomorphic-java}]
public class Chapter5Monomorphic {
    private Chapter5Monomorphic() {
        // private constructor to prevent instantiation
    }
    
    public static void main(String[] args) {
        System.out.println(add.apply(add.apply(fun1).apply(fun2)).apply(letFun));
    }
    
    public static Long fun1;    
    public static Long fun2;    
    public static Long letFun;
    
    static {
        fun1 = 1L;    
        fun2 = 2L;    
        letFun =
            (new Let@<@>() {
                private Long a;    
                private Long b;
    
                @Override
                public Long _eval() {
                    a = 3L;
                    b = 4L;
                    return add.apply(a).apply(b);
                }
            })._eval();
    }
}
\end{lstlisting}

\newpage

\noindent Poiché potrebbero essere presenti diversi \texttt{let} annidati, è necessario tenere traccia delle espressioni
corpo delle dichiarazioni monomorfe in modo da poterle inizializzare al momento corretto, dopo aver tradotto ulteriori classi interne.

\noindent La procedura di traduzione di dichiarazioni monomorfe si compone delle seguenti fasi:
\begin{itemize}
    \item definizione di uno \textit{stack} contenente mappe tra nomi delle dichiarazioni e nodi espressione corrispondenti;
    \item inserimento di una nuova mappa per il livello corrente di annidamento (modulo o espressione \texttt{let});
    \item dichiarazione delle variabili e aggiunta delle espressioni monomorfe
          alla mappa corrente (potrebbero essere aggiunti nuovi livelli prima di poterne "riempire" uno);
    \item creazione del blocco statico (o metodo \texttt{eval} per le espressioni \texttt{let}) con l'inizializzazione
          delle funzioni monomorfe: in questa fase finale torna utile la versatilità del \textit{visitor pattern} per posticipare
          la traduzione delle espressioni.
\end{itemize}

\vspace{4mm}
\begin{lstlisting}[caption={Traduzione di funzioni monomorfe in \texttt{let}}, style=javaCode, label={lst:5-13-monomorphic-translation-java}]
private final Deque<Map<String, Expression>>
    monomorphicDeclarationsQueue = new ArrayDeque<>();

@Override
public Void visitLet(Expression.Let let) {
    currentLevel++;
    // ...
    // use a new anonymous class for the let expression
    // and push a new monomorphic let declarations map
    builder.append("(new ")
            .append(JavaPrelude.Let.class.getSimpleName()).append("<>() {\n");
    monomorphicDeclarationsQueue.push(new LinkedHashMap<>());
    visit(let.localDeclarations);
    builder.append("""
                    @Override
                    public\s""")
            .append(typeStringOf(let.expression.type()))
            .append("\n_eval() {\n");
    // if there are any monomorphic declarations, initialize them in the _eval method,
    // then pop the map either way
    if (!monomorphicDeclarationsQueue.getFirst().isEmpty())
        monomorphicDeclarationsQueue.getFirst().forEach((id, expression) @-> {
            builder.append(id).append(" = ");
            visit(expression); // deferred expression visit
            appendSemiColon();
            appendNewline();
        });
    monomorphicDeclarationsQueue.pop();
    // ...
    currentLevel--;
    return null;
}
\end{lstlisting}

\newpage

\begin{lstlisting}[caption={Metodo \texttt{visit} per le dichiarazioni}, style=javaCode, label={lst:5-13-visit-declaration-java}]
@Override
public Void visitDeclaration(Declaration declaration) {
    // top level declarations should be static and public,
    // while let local declarations should be private to the anonymous class
    builder.append(currentLevel == 0 @? "public static " @: "private ");
    String scheme = schemeStringOf(declaration.scheme());
    if (declaration.scheme().variables.isEmpty()) {
        // defer initialization of monomorphic declarations
        builder.append(scheme).append(" ").append(declaration.id);
        appendSemiColon();
        monomorphicDeclarationsQueue
            .getFirst().put(declaration.id, declaration.expression);
    } else {
        // initialize polymorphic declarations immediately (as methods with generics)
        builder.append(scheme)
                .append(" ").append(declaration.id).append("() {\nreturn ");
        visit(declaration.expression);
        appendSemiColon();
        appendCloseBrace();
    }
    appendNewline();
    return null;
}
\end{lstlisting}

\subsection{Instanziazione di funzioni polimorfe}
\label{sec:5-14-polymorphic-functions-instantiation}

\subsection{Type casting "selvaggio"}
\label{sec:5-15-wild-type-casting}
