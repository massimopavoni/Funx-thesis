\section{ML, Haskell e Funx}
\label{sec:2-ml-haskell-funx}

Nonostante le funzioni pure tipiche di un linguaggio funzionale siano un concetto molto attraente
dal punto di vista della correttezza della computazione, i vincoli così imposti possono risultare
stringenti a tal punto da rendere difficile, se non impossibile, la scrittura di programmi che
interagiscano con il mondo reale.

Per questo motivo, molti linguaggi funzionali permettono invece di utilizzare particolari funzioni
impure o di effettuare almeno operazioni di input/output. Inoltre, molti linguaggi prevalentemente
imperativi adottano ormai da tempo alcune caratteristiche tipiche dei linguaggi funzionali
(e.g. \texttt{Rust}, il linguaggio più amato dagli sviluppatori secondo gli ultimi sondaggi di \textit{Stack Overflow},
eredita molto dal linguaggio con cui era scritto il suo primo compilatore, \texttt{OCaml}, ed è dotato quindi di
funzioni di prima classe, immutabilità di default, strutture dati algebriche, ecc.).


\texttt{ML} è un linguaggio funzionale sviluppato negli anni '70 presso l'Università di Edimburgo,
costituente la base per moltissimi dei linguaggi sviluppati in seguito.
\texttt{ML} permette effettivamente l'uso di funzioni impure, ma fra i suoi discendenti
vi è \texttt{Haskell}, uno dei pochi linguaggi completamente puri.

\noindent \texttt{Haskell} si avvale di un pattern di programmazione chiamato \textit{monadi}
(vedi \textit{Notions of computation and monads}, \cite{Monads}) per gestire
le operazioni di input/output e altre operazioni impure, garantendo comunque la purezza delle funzioni.


Nell'ideare \textbf{Funx} l'ispirazione viene proprio da \texttt{Haskell}, ma è presente la possibilità
di dichiarare un'unica funzione impura (il cosiddetto \textit{main}) per permettere di visualizzare a schermo un risultato.
Il linguaggio non è quindi allo stesso livello di purezza di \texttt{Haskell}, e naturalmente non supporta
molte delle funzionalità più avanzate di quest'ultimo (come le \textit{classi di tipi} e il \textit{pattern matching}),
ma ne mutua altre comunque interessanti, tra cui l'uso di alcuni operatori infissi e il \textit{polimorfismo parametrico}.