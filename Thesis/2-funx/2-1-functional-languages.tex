\section{Linguaggi funzionali}
\label{sec:2-1-functional-languages}

Nonostante i linguaggi di programmazione non si possano confinare all'interno di un solo paradigma,
parlando di linguaggi di programmazione si fa comunque spesso riferimento a due grandi categorie:
linguaggi imperativi e linguaggi dichiarativi.


I primi hanno caratteristiche direttamente legate al modello di calcolo di \textit{John Von Neumann},
a sua volta non dissimile dalla macchina di \textit{Alan Turing}.
Questi linguaggi sono usati per codice che segue una precisa sequenza di istruzioni,
la quale descrive più o meno esplicitamente i passi necessari per risolvere il problema affrontato.

\noindent Appartengono alla famiglia dei linguaggi di programmazione imperativi sia linguaggi procedurali come
\texttt{Fortran}, \texttt{Cobol} e \texttt{Zig}, sia i linguaggi orientati agli oggetti, tra cui \texttt{Kotlin}, \texttt{C\#} e \texttt{Ruby}.


I linguaggi dichiarativi, invece, sono fondamentali per lo scopo del progetto:
tali linguaggi sono generalmente di altissimo livello e permettono allo sviluppatore
di concentrarsi sull'obiettivo da raggiungere piuttosto che sui dettagli implementativi.

\noindent Fanno parte di questa categoria linguaggi di interrogazione come \texttt{SQL},
linguaggi logici come \texttt{Prolog} e soprattutto i linguaggi funzionali:
\texttt{Lisp}, \texttt{Clojure}, \texttt{Elixir}, \texttt{OCaml} e \texttt{Haskell} sono alcuni esempi.


Alla base dei linguaggi funzionali vi è il \textbf{lambda calcolo} {\cite{Church-1932-FoundationLogic,Church-1933-FoundationLogic}}:
un sistema formale definito dal matematico \textit{Alonzo Church} (supervisore di \textit{Alan Turing} durante il dottorato),
equivalente alla macchina di Turing, ma fondato sulle funzioni pure.

\newpage

\noindent La grammatica del lambda calcolo verrà presentata poco più avanti (sezione~\ref{sec:2-3-syntax}),
ma le regole che ne governano il funzionamento e il modo in cui queste vengano utilizzate per ridurre
le espressioni ad una forma normale esulano dai fini di questo documento.

\noindent Rimane comunque rilevante elencare le principali qualità che un linguaggio funzionale
usualmente matura grazie al lambda calcolo:
\begin{itemize}
      \item \textbf{funzioni come entità di prima classe}: le funzioni possono essere passate come argomenti
            e restituite come risultato di altre funzioni;
      \item \textbf{immutabilità}: le variabili utilizzate sono immutabili;
      \item \textbf{purezza}: le funzioni sono libere da effetti collaterali
            (non modificano lo stato del programma) e restituiscono sempre lo stesso output per input identici;
      \item \textbf{ricorsione}: la ricorsione è il meccanismo più idiomatico per esprimere
            l'iterazione su una struttura dati.
\end{itemize}

\subsection{ML, Haskell e Funx}
\label{sec:2-2-ml-haskell-funx}

Nonostante le funzioni pure tipiche di un linguaggio funzionale siano un concetto molto attraente
dal punto di vista della correttezza della computazione, i vincoli così imposti possono risultare
stringenti a tal punto da rendere difficile, se non impossibile, la scrittura di programmi che
interagiscano con il mondo reale.


Per questo motivo, molti linguaggi funzionali permettono invece di utilizzare particolari funzioni
impure o di effettuare almeno operazioni di input/output. Inoltre, molti linguaggi prevalentemente
imperativi adottano ormai da tempo alcune caratteristiche tipiche dei linguaggi funzionali
(e.g. \texttt{Rust}, il linguaggio più amato%
\footnote{Stack Overflow Developer Survey 2023 (\url{https://survey.stackoverflow.co/2023}), \\
    \textit{Rust is the most admired language}}
dagli sviluppatori secondo i sondaggi di \textit{Stack Overflow},
eredita molto dal linguaggio con cui era scritto il suo primo compilatore, \texttt{OCaml}, ed è dotato quindi di
funzioni di prima classe, immutabilità di default, strutture dati algebriche, ecc.).


\texttt{ML} è un linguaggio funzionale sviluppato negli anni '70 presso l'Università di Edimburgo,
costituente la base per moltissimi dei linguaggi sviluppati in seguito.
\texttt{ML} permette effettivamente l'uso di funzioni impure, ma fra i suoi discendenti
vi è \texttt{Haskell}, uno dei pochi linguaggi invece completamente puri.

\noindent \texttt{Haskell} si avvale di un pattern di programmazione chiamato \textit{monadi} {\cite{Moggi-1991-ComputationMonads}}
per gestire le operazioni di input/output e altre impurità, mantenendo le funzioni pure.


Nell'ideare \textbf{Funx} l'ispirazione viene proprio da \texttt{Haskell}, ma è presente la possibilità
di dichiarare un'unica funzione impura (il cosiddetto \textit{main}) per permettere di visualizzare a schermo un risultato.
Il linguaggio non è quindi allo stesso livello di purezza di \texttt{Haskell}, e naturalmente non supporta
molte delle funzionalità più avanzate di quest'ultimo (come le \textit{classi di tipi} e il \textit{pattern matching}),
ma ne mutua altre comunque interessanti, tra cui l'uso di alcuni operatori infissi e il \textit{polimorfismo parametrico}.