\section{Linguaggi funzionali}
\label{sec:2-1-functional-languages}

Nonostante molti linguaggi non si possano confinare all'interno di un solo paradigma,
parlando di linguaggi di programmazione si fa spesso riferimento a due grandi categorie:
linguaggi imperativi e linguaggi dichiarativi.


I primi hanno caratteristiche direttamente legate al modello di calcolo di \textit{John Von Neumann},
a sua volta non dissimile dalla macchina di \textit{Alan Turing}.
Questi linguaggi sono usati per scrivere codice che segue una precisa sequenza di istruzioni,
la quale descrive più o meno esplicitamente i passi necessari per risolvere il problema affrontato.

\noindent Appartengono alla famiglia dei linguaggi di programmazione imperativi sia linguaggi procedurali come
\texttt{Fortran}, \texttt{Cobol} e \texttt{Zig}, sia i linguaggi orientati agli oggetti, tra cui \texttt{Kotlin}, \texttt{C\#} e \texttt{Ruby}.


I linguaggi dichiarativi, invece, sono fondamentali per lo scopo del progetto:
tali linguaggi sono generalmente di altissimo livello e permettono allo sviluppatore
di concentrarsi sull'obiettivo da raggiungere piuttosto che sui dettagli implementativi.

\noindent Fanno parte di questa categoria linguaggi di interrogazione come \texttt{SQL},
linguaggi logici come \texttt{Prolog} e soprattutto i linguaggi funzionali:
\texttt{Lisp}, \texttt{Clojure}, \texttt{Elixir}, \texttt{OCaml} e \texttt{Haskell} sono alcuni esempi.


Alla base di ogni linguaggio funzionale vi è il \textbf{lambda calcolo}%
\footnote{\citetitle{Church-1932-FoundationLogic}, \cite{Church-1932-FoundationLogic}
      e \citetitle{Church-1933-FoundationLogic} (Second paper), \cite{Church-1933-FoundationLogic}}:
un sistema formale definito dal matematico \textit{Alonzo Church} (supervisore di \textit{Alan Turing} durante il dottorato),
equivalente alla macchina di Turing, ma fondato sulle funzioni pure.

\newpage

\noindent La grammatica del lambda calcolo verrà presentata poco più avanti (sezione~\ref{sec:2-3-syntax}),
ma le regole che ne governano il funzionamento e il modo in cui queste vengano utilizzate per ridurre
le espressioni ad una forma normale esulano dai fini di questo documento.

\noindent Rimane comunque rilevante elencare le principali qualità che un linguaggio funzionale
usualmente matura grazie al lambda calcolo:
\begin{itemize}
      \item \textbf{funzioni come entità di prima classe}: le funzioni possono essere passate come argomenti
            e restituite come risultato di altre funzioni;
      \item \textbf{immutabilità}: le variabili utilizzate sono immutabili;
      \item \textbf{purezza}: le funzioni sono libere da effetti collaterali
            (non modificano lo stato del programma) e restituiscono sempre lo stesso output per input identici;
      \item \textbf{ricorsione}: la ricorsione è il meccanismo più idiomatico per esprimere
            l'iterazione su una struttura dati.
\end{itemize}