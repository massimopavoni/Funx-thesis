\section{Sintassi}
\label{sec:2-3-syntax}

La sintassi di \textbf{Funx} risulta molto simile a quella di \texttt{Haskell}, con poche differenze dovute
a tre principali motivi:
\begin{itemize}
    \item libera scelta di nomi e simboli per le parole chiave;
    \item necessità di successiva traduzione in \texttt{Java};
    \item difficoltà e scarso valore all'interno del progetto dell'implementazione di un parser dipendente dall'indentazione.
\end{itemize}

\noindent A prescindere da ciò, il cuore del linguaggio è lo stesso di ogni altro linguaggio derivato dal lambda calcolo:
la sua definizione si può agilmente comprendere visualizzando la grammatica del lambda calcolo e confrontandola con
quella (leggermente semplificata) di \textbf{Funx}, facendo attenzione alle regole aggiuntive.

\begin{figure}
    \vspace{4mm}
    \begin{bnf}
        $E$ : \small{Espressione} ::=
        | $x$ : \small{variabile}
        | $E_l\ E_r$ : \small{applicazione}
        | $\lambda x\ .\ E$ : \small{astrazione}
    \end{bnf}
    \caption{Grammatica del lambda calcolo}
    \label{fig:2-3-lambda-syntax}
    \vspace{4mm}
\end{figure}

\noindent Le tre regole in Figura~\ref{fig:2-3-lambda-syntax} indicano le tre componenti indispensabili
del lambda calcolo:
\begin{itemize}
    \item \textbf{variabile}: simbolo rappresentante un parametro;
    \item \textbf{applicazione}: applicazione di funzione ad un argomento (entrambi espressioni);
    \item \textbf{astrazione}: definizione di una funzione anonima, con un solo input $x$ (variabile vincolata)
          e un solo output $E$ (espressione); per definire funzioni con più
          parametri si debbono usare molteplici astrazioni annidate (tecnica detta \textit{currying}).
\end{itemize}

\newpage

\begin{figure}
    \begin{bnf}
        $M$ : \small{Modulo} ::=
        | $nome\ \cdot\ L$
        ;;
        $D$ : \small{Dichiarazione} ::=
        | $?(schema\ di\ tipo)\ \cdot\ id = E$ : \small{funzione}
        ;;
        $E$ : \small{Espressione} ::=
        | $c$ : \small{costante}
        | $x$ : \small{variabile}
        | $E_l\ E_r$ : \small{applicazione}
        | $\lambda x\ .\ E$ : \small{astrazione}
        | $L$ : \small{let}
        | $\textbf{if}\ E_c\ \textbf{then}\ E_t\ \textbf{else}\ E_e$ : \small{if}
        ;;
        $L$ : \small{Let} ::=
        | $\textbf{let}\ \cdot\ D\ (\cdot\ D)^*\ \cdot\ \textbf{in}\ E$
    \end{bnf}
    \caption{Grammatica di Funx}
    \label{fig:2-3-funx-syntax}
    \vspace{4mm}
\end{figure}

\noindent È facile constatare la presenza delle ulteriori produzioni per la definizione del modulo corrente
(informazione inclusa a prescindere dal fatto che il linguaggio ad ora non supporti l'importazione di moduli esterni
che non siano la libreria standard) e di funzioni con nome: lo \textit{schema di tipo} è un'informazione opzionale
relativa al tipo della funzione e di cui si parlerà più approfonditamente nella sezione~\ref{sec:3-3-system-fc}.

\noindent Per quanto riguarda invece le espressioni, vengono introdotte tre nuove regole:
\begin{itemize}
    \item \textbf{costante}: rappresenta un valore letterale, come un numero o una stringa;
    \item \textbf{let}: permette di avere dichiarazioni locali utilizzabili all'interno di un'espressione;
    \item \textbf{if}: la più classica istruzione condizionale controllata da un'espressione booleana.
\end{itemize}

\subsection{Zucchero sintattico}
\label{sec:2-4-syntactic-sugar}

Con lo scopo di rendere il codice più leggibile, conciso e semplice, \textbf{Funx} introduce
dello zucchero sintattico (del tutto simile a quello di \texttt{Haskell}).
In Tabella~\ref{tab:2-4-sugar} sono riportati l'indispensabile per evitare il parsing dell'indentazione,
le semplificazioni comuni utili all'arricchimento del lambda calcolo, e infine tutti gli operatori simbolici
supportati al momento (assieme alla notazione per indicarne associatività e precedenza).

\newpage

\begin{table}[H]
    \begin{center}
        \begin{tabularx}{\textwidth}{|P{15em}|X|}
            \hline
            \textbf{Zucchero}                & \textbf{Sostituzione}                                            \\
            \hline
            \texttt{$\backslash$x -> e}      & \texttt{$\lambda$x $\mathord{.}$ e}                              \\
            \hline
            \texttt{$\backslash$x y -> e}    & \texttt{$\lambda$x $\mathord{.}$ $\lambda$y $\mathord{.}$ e}     \\
            \hline
            \texttt{f x y = e}               & \texttt{f = $\lambda$x $\mathord{.}$ $\lambda$y $\mathord{.}$ e} \\
            \hline
            \texttt{let}                     &                                                                  \\
            \texttt{f1 = e1}                 & \texttt{let f1 = e1 $\cdot$ f2 = e2 in e3}                       \\
            \texttt{f2 = e2}                 &                                                                  \\
            \texttt{in e3}                   &                                                                  \\
            \hline
            \texttt{f3 = e3}                 &                                                                  \\
            \texttt{with}                    &                                                                  \\
            \texttt{f1 = e1}                 & \texttt{f3 = let f1 = e1 $\cdot$ f2 = e2 in e3}                  \\
            \texttt{f2 = e2}                 &                                                                  \\
            \texttt{out}                     &                                                                  \\
            \hline
            \texttt{main = e3}               &                                                                  \\
            \texttt{f1 = e1}                 & \texttt{main = let f1 = e1 $\cdot$ f2 = e2 in e3}                \\
            \texttt{f2 = e2}                 &                                                                  \\
            \hline
            \texttt{if b then e1 else e2 fi} & \texttt{if b then e1 else e2}                                    \\
            \hline
        \end{tabularx}
        % divide et impera because inconsistent tabbing is a thing
        \begin{tabularx}{\textwidth}{|P{7em}@{\quad}P{7em}|X|}
            \texttt{e1 $\mathord{.}$ e2} & \texttt{infixr 9} & \texttt{compose e1 e2}            \\
            \texttt{e1 / e2}             & \texttt{infixl 7} & \texttt{divide e1 e2}             \\
            \texttt{e1 \% e2}            & \texttt{infixl 7} & \texttt{modulo e1 e2}             \\
            \texttt{e1 * e2}             & \texttt{infixl 7} & \texttt{multiply e1 e2}           \\
            \texttt{e1 + e2}             & \texttt{infixl 6} & \texttt{add e1 e2}                \\
            \texttt{e1 - e2}             & \texttt{infixl 6} & \texttt{subtract e1 e2}           \\
            \texttt{e1 > e2}             & \texttt{infix 4}  & \texttt{greaterThan e1 e2}        \\
            \texttt{e1 >= e2}            & \texttt{infix 4}  & \texttt{greaterThanEquals e1 e2}  \\
            \texttt{e1 < e2}             & \texttt{infix 4}  & \texttt{lessThan e1 e2}           \\
            \texttt{e1 <= e2}            & \texttt{infix 4}  & \texttt{lessThanEquals e1 e2}     \\
            \texttt{e1 == e2}            & \texttt{infix 4}  & \texttt{equalsEquals e1 e2}       \\
            \texttt{e1 != e2}            & \texttt{infix 4}  & \texttt{notEquals e1 e2}          \\
            \texttt{!!e}                 & \texttt{prefix 4} & \texttt{not e}                    \\
            \texttt{e1 \&\& e2}          & \texttt{infixr 3} & \texttt{if e1 then e2 else False} \\
            \texttt{e1 || e2}            & \texttt{infixr 2} & \texttt{if e1 then True else e2}  \\
            \texttt{e1 \$ e2}            & \texttt{infixr 0} & \texttt{apply e1 e2}              \\
            \hline
        \end{tabularx}
    \end{center}
    \caption{Zucchero sintattico}
    \label{tab:2-4-sugar}
\end{table}

\newpage

\noindent Come già accennato, il Capitolo~\ref{chap:5-compiler} illustrerà come l'albero sintattico astratto (\textbf{AST})
di un programma viene ottenuto, annotato e tradotto in \texttt{Java}; la sezione~\ref{sec:4-2-ternary-operator}
esporrà invece il motivo della traduzione degli operatori booleani binari in if.

\noindent Alcuni esempi di funzioni sono presentati nel Codice~\ref{lst:2-4-example-funx};
seppur superflua, l'indentazione è inclusa per maggiore chiarezza.

\vspace{4mm}
\begin{lstlisting}[caption={Esempio di programma}, style=funxCode, label={lst:2-4-example-funx}]
main = factorial 20

factorial : Int @-> Int
factorial n = if n == 0 then 1 else n * factorial (n - 1) fi

even : Int @-> Bool
even = let
        even1 : Int @-> Bool
        even1 n = if n == 0 then True else odd (n - 1) fi

        odd : Int @-> Bool
        odd n = if n == 0 then False else even1 (n - 1) fi
    in even1

gcd : Int @-> Int @-> Int
gcd a b = if b == 0 then a else gcd b (a % b) fi

xor : Bool @-> Bool @-> Bool
xor a b = (a || b) && !!(a && b)
\end{lstlisting}