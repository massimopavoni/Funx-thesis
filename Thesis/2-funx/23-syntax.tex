\section{Sintassi}
\label{sec:2-syntax}

La sintassi di \textbf{Funx} risulta molto simile a quella di \texttt{Haskell}, con poche differenze dovute
a tre principali motivi:
\begin{itemize}
    \item libera scelta di nomi e simboli per le parole chiave;
    \item necessità di successiva traduzione in \texttt{Java};
    \item difficoltà e scarso valore all'interno del progetto dell'implementazione di un parser dipendente dall'indentazione.
\end{itemize}

\noindent A prescindere da ciò, il cuore del linguaggio è lo stesso di ogni altro linguaggio derivato dal lambda calcolo:
la sua definizione si può agilmente comprendere visualizzando la grammatica del lambda calcolo e confrontandola con
quella (leggermente semplificata) di \textbf{Funx}, facendo attenzione alle regole aggiuntive.

\begin{figure}[H]
    \centering
    \vspace{8mm}
    \begin{spacedbnf}
        $E$ : \small{Espressione} ::=
        | $x$ : \small{variabile}
        | $E_l\ E_r$ : \small{applicazione}
        | $\lambda x\ .\ E$ : \small{astrazione}
    \end{spacedbnf}
    \caption{Grammatica del lambda calcolo}
    \label{fig:2-lambda-syntax}
    \vspace{8mm}
\end{figure}

\noindent Le tre regole presenti in Figura \ref{fig:2-lambda-syntax} indicano le tre componenti indispensabili
del lambda calcolo:
\begin{itemize}
    \item \textbf{variabile}: simbolo rappresentante un parametro;
    \item \textbf{applicazione}: applicazione di funzione ad un argomento (entrambi espressioni);
    \item \textbf{astrazione}: definizione di una funzione anonima, con un solo input $x$ (variabile vincolata)
          e un solo output $E$ (espressione, potenzialmente un'altra astrazione).
\end{itemize}

\newpage

\begin{figure}[H]
    \centering
    \begin{spacedbnf}
        $M$ : \small{Modulo} ::=
        | $nome\ \cdot\ L$ : \small{dichiarazione del modulo}
        ;;
        $D$ : \small{Dichiarazione} ::=
        | $?(schema\ di\ tipo)\ \cdot\ id = E$ : \small{dichiarazione di funzione}
        ;;
        $E$ : \small{Espressione} ::=
        | $c$ : \small{costante}
        | $x$ : \small{variabile}
        | $E_l\ E_r$ : \small{applicazione}
        | $\lambda x\ .\ E$ : \small{astrazione}
        | $L$ : \small{let}
        | $if\ E\ then\ E\ else\ E$ : \small{if}
        ;;
        $L$ : \small{Let} ::=
        | $let\ \cdot\ D\ D*\ \cdot\ in\ E$ : \small{let}
    \end{spacedbnf}
    \caption{Grammatica di Funx}
    \label{fig:2-funx-syntax}
    \vspace{8mm}
\end{figure}

\noindent È facile constatare la presenza delle ulteriori produzioni per la definizione del modulo corrente
(informazione inclusa a prescindere dal fatto che il linguaggio ad ora non supporti l'importazione di moduli esterni
che non siano la libreria standard) e di funzioni con nome: lo \textit{schema di tipo} è un'informazione opzionale
relativa al tipo della funzione e di cui si parlerà più approfonditamente nella sezione \ref{3-parametric-polymorphism}.

\noindent Per quanto riguarda invece le espressioni, vengono introdotte tre nuove regole:
\begin{itemize}
    \item \textbf{costante}: rappresenta un valore letterale, come un numero o una stringa;
    \item \textbf{let}: permette di avere dichiarazioni locali utilizzabili all'interno di un'espressione;
    \item \textbf{if}: non è altro che la più classica istruzione condizionale controllata da un'espressione booleana.
\end{itemize}