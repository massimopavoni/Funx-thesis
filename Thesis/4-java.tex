\chapter{Java}
\label{chap:4-java}

Parallelamente alle prime fasi di sviluppo è stata svolta un'analisi di \texttt{Java}%
\footnote{OpenJDK (\url{https://openjdk.org})}
per valutare quali fossero le caratteristiche del linguaggio utili alla traduzione di codice \textbf{Funx}.

\noindent Questo capitolo riporta pertanto una breve panoramica delle principali funzionalità impiegate,
accompagnate da esempi di traduzione che illustrano alcuni dei risultati ottenibili con il compilatore
(si ricorda che altri esempi sono esibiti durante il Capitolo~\ref{chap:5-compiler}).

Le funzioni non dichiarate all'interno degli esempi stessi provengono dalla piccola libreria standard
(suddivisa in \texttt{FunxPrelude} e \texttt{JavaPrelude}, di cui la seconda include funzioni necessariamente definite in \texttt{Java}),
parzialmente presentata nella Tabella~\ref{tab:2-4-sugar} con gli operatori simbolici.

\section{Interfacce funzionali}
\label{sec:4-1-functional-interfaces}

Nel tradurre un linguaggio funzionale viene naturale pensare immediatamente alle \textit{interfacce funzionali} e \textit{lambda espressioni}
introdotte in \texttt{Java} 8%
\footnote{OpenJDK 8 (\url{https://openjdk.org/projects/jdk8})}
per rappresentare funzioni anonime: l'interfaccia generica \texttt{Function} è la più adatta a riprodurre
il comportamento dell'astrazione, come mostrato nei Codici~\ref{lst:4-1-function-funx}~e~\ref{lst:4-1-function-java}.

\vspace{4mm}
\begin{lstlisting}[caption={Semplice funzione in \textbf{Funx}}, style=funxCode, label={lst:4-1-function-funx}]
constant : a @-> b @-> a
constant x = \y @-> x
\end{lstlisting}
\vspace{4mm}
\begin{lstlisting}[caption={Corrispondente metodo in \texttt{Java}}, style=javaCode, label={lst:4-1-function-java}]
public static @<a, b@> Function@<a, Function@<b, a@>@> constant() {
    return (x @-> (y @-> x));
}
\end{lstlisting}
\vspace{4mm}

\noindent Dato il naturale \textit{currying} di oggetti \texttt{Function}, questo tipo di traduzione ha il vantaggio
di permettere l'applicazione parziale di funzioni (tramite una sequenza di \texttt{apply()} in numero minore rispetto al totale degli input),
ma il grande svantaggio della creazione di una nuova istanza della funzione per ogni chiamata.


Poiché \texttt{constant} ha tipo polimorfo (sezione~\ref{sec:4-3-generics}), il metodo utilizza parametri di tipo
e deve quindi necessariamente restituire un nuovo oggetto con ogni chiamata:
nonostante la performance delle traduzioni non sia un obiettivo primario del progetto, la versione attuale
del compilatore fa uso di alcune piccole ottimizzazioni nella trasposizione delle funzioni monomorfe,
approfondite nella sezione~\ref{sec:5-13-monomorphic-declarations}.

\newpage

\noindent In aggiunta alla classe \texttt{Function}, nella parte nativa (codice \texttt{Java}) della libreria standard
di \textbf{Funx} è definita un'ulteriore interfaccia funzionale per creare espressioni \texttt{let}:
in questo caso vi è un grande utilizzo di classi anonime, potenzialmente annidate,
rappresentanti le dichiarazioni locali e l'espressione principale.

\vspace{4mm}
\begin{lstlisting}[caption={Interfaccia funzionale per espressioni \texttt{let}}, style=javaCode, label={lst:4-1-let-interface-java}]
@FunctionalInterface
public interface Let@<T@> {
    T _eval();
}
\end{lstlisting}
\vspace{4mm}
\begin{lstlisting}[caption={Espressione \texttt{let} in \textbf{Funx}}, style=funxCode, label={lst:4-1-let-funx}]
hundredsSum : Int @-> Int @-> Int
hundredsSum = let
        on : (a @-> a @-> b) @-> (c @-> a) @-> c @-> c @-> b
        on op f x y = op (f x) (f y)
    in on add (multiply 100)
\end{lstlisting}
\vspace{4mm}
\begin{lstlisting}[caption={Corrispondente classe anonima in \texttt{Java}}, style=javaCode, label={lst:4-1-let-java}]
public static Function@<Long, Function@<Long, Long@>@> hundredsSum;

static {
    hundredsSum = (new Let@<@>() {
        private @<a, b, c@>
            Function@<
                    Function@<a, Function@<a, b@>@>,
                    Function@<Function@<c, a@>, Function@<c, Function@<c, b@>@>@>@>
                on() {
            return (op @-> (f @-> (x @-> (y @-> op.apply(f.apply(x)).apply(f.apply(y))))));
        }

        @Override
        public Function@<Long, Function@<Long, Long@>@> _eval() {
            return this.@<Long, Long, Long@>on().apply(add).apply(multiply.apply(100L));
        }
    })._eval();
}
\end{lstlisting}
\vspace{4mm}

\noindent Nei Codici~\ref{lst:4-1-let-funx}~e~\ref{lst:4-1-let-java} si può vedere che la funzione \texttt{hundredsSum}
è implementata attraverso la chiamata al metodo principale dell'interfaccia funzionale \texttt{Let},
realizzato internamente alla classe anonima con il supporto del metodo polimorfo \texttt{on}.

\noindent Inoltre, è immediatamente evidente come le traduzioni in \texttt{Java} siano progressivamente più complesse e meno leggibili
con l'introduzione di nuove funzionalità: l'esempio più eclatante è dato proprio dalla \textit{signature} del metodo locale \texttt{on},
divenuta di difficile comprensione rispetto alla sintassi molto concisa del linguaggio funzionale.

\newpage

\section{Operatore ternario}
\label{sec:4-2-ternary-operator}

Una delle peculiarità della traduzione da \textbf{Funx} a \texttt{Java} è l'uso dell'operatore ternario
(\texttt{condition ? thenBranch : elseBranch}) ogni qualvolta
siano presenti espressioni condizionali \texttt{if-then-else}.

I linguaggi funzionali sfruttano spesso una caratteristica (non menzionata nella sezione~\ref{sec:2-1-functional-languages})
che prende il nome di \textit{lazy evaluation} (valutazione pigra): fino a quando il risultato di un'espressione
non è richiesto per un successivo calcolo, questa non verrà completamente valutata.

\noindent Oltre ad offrire molteplici possibilità di ottimizzazione dal punto di vista del tempo di esecuzione,
tale comportamento è molto comodo nella scrittura di funzioni che per esempio potrebbero terminare
prima del previso o magari effettuare computazioni su strutture dati infinite.
I linguaggi che sono \textit{lazy evaluated} di default impegano nella maggior parte dei casi un \textit{garbage collector}
per liberare la memoria occupata dalle espressioni non valutate e non più rilevanti.

Il linguaggio \texttt{Java} non adotta la \textit{lazy evaluation} di default se non in casi particolari, tra cui
gli operatori booleani binari, il costrutto \texttt{if-then-else} (e corrispondente operatore ternario) e altre
funzionalità più avanzate tra cui gli \textit{stream} e le \textit{lambda espressioni} già viste.
Utilizzando quest'ultime si potrebbero ottenere risultati simili, in termini di valutazione pigra, a quelli di un linguaggio funzionale;
tuttavia, rendere \textbf{Funx} un linguaggio completamente pigro avrebbe comportato una traduzione indubbiamente ancora più complessa,
molteplici rischi di peggiorare le prestazioni dei programmi e un'implementazione del compilatore che va oltre lo scopo di questo progetto.

Nonostante ciò, la scelta di ridurre gli operatori booleani binari (\textit{and} e \textit{or}) ad espressioni con operatore ternario
è stata considerata quasi obbligatoria per conservarne la natura pigra: gli operatori ternari utilizzati a questo scopo
derivano direttamente dalla costruzione dell'\textbf{AST} (sezione~\ref{sec:5-6-ast-builder}), motivo per cui non vengono riconvertiti
in operatori nativi di \texttt{Java} in fase di traduzione.

\vspace{4mm}
\begin{lstlisting}[caption={If e operatori booleani in \textbf{Funx}}, style=funxCode, label={lst:4-ternary-funx}]
power : Int @-> Int @-> Int
power b e = if e == 0 then 1 else b * power b (e - 1) fi

xor : Bool @-> Bool @-> Bool
xor a b = (a || b) && !!(a && b)
\end{lstlisting}
\vspace{4mm}
\begin{lstlisting}[caption={Corrispondenti operatori ternari in \texttt{Java}}, style=javaCode, label={lst:4-ternary-java}]
public static Function@<Long, Function@<Long, Long@>@> power;

public static Function@<Boolean, Function@<Boolean, Boolean@>@> xor;

static {
    power = (b @-> (e @-> ((JavaPrelude.@<Long@>equalsEquals().apply(e).apply(0L))
        @? (1L)
        @: (multiply.apply(b)
            .apply(power.apply(b).apply(subtract.apply(e).apply(1L)))))));

    xor = (a @-> (b @->
        ((((a) @? (true) @: (b))) @? (not.apply(((a) @? (b) @: (false)))) @: (false))));
}
\end{lstlisting}


\newpage

\section{Tipi generici}
\label{sec:4-3-generics}

Il sistema di tipo di \textbf{Funx} necessita la traduzione di funzioni polimorfe,
e la soluzione più semplice e idiomatica in \texttt{Java} è l'utilizzo dei \textit{generics}:
tramite i parametri di tipo generici è possibile definire classi e metodi che agiscono su molteplici tipi di dati,
implementando comportamenti che possono essere condivisi dai diversi elementi del dominio di tipi delle funzioni rappresentate.


Nel contesto del \textit{sistema HM} di \textbf{Funx}, le variabili quantificate universalmente nei politipi
hanno una diretta corrispondenza con i parametri di tipo che possono essere dichiarati tra i modificatori di visibilità
e il tipo di ritorno di un metodo, il quale a sua volta combacia con la parte interna dello schema di tipo.


Nei Codici~\ref{lst:4-3-generics-funx}~e~\ref{lst:4-3-generics-java} si può notare come \texttt{Java} non sempre sia in grado
d'inferire i tipi desiderati per le funzioni polimorfe: queste devono infatti essere istanziate esplicitamente
usando la classe di appartenenza e le parentesi angolari (questa limitazione richiederà alcuni espedienti in casi limite
illustrati nelle sezioni~\ref{sec:5-14-polymorphic-functions-instantiation}~e~\ref{sec:5-15-wild-type-casting}).

\vspace{4mm}
\begin{lstlisting}[caption={Scrittura e utilizzo di funzioni polimorfe in \textbf{Funx}}, style=funxCode, label={lst:4-3-generics-funx}]
sumToN : Int @-> Int
sumToN = let
        ap : (a @-> b @-> c) @-> (a @-> b) @-> a @-> c
        ap op f x = op x (f x)
    in (flip divide 2) . ap multiply (add 1)
\end{lstlisting}
\vspace{4mm}
\begin{lstlisting}[caption={Corrispondenti proprietà e metodi \texttt{Java}}, style=javaCode, label={lst:4-3-generics-java}]
public static Function@<Long, Long@> sumToN;

static {
    sumToN = (new Let@<@>() {
        private @<a, b, c@>
            Function@<
                Function@<a, Function@<b, c@>@>,
                Function@<Function@<a, b@>, Function@<a, c@>@>@>
                    ap() {
            return (op @-> (f @-> (x @-> op.apply(x).apply(f.apply(x)))));
        }

        @Override
        public Function@<Long, Long@> _eval() {
            return FunxPrelude.@<Long, Long, Long@>compose()
                .apply(FunxPrelude.@<Long, Long, Long@>flip().apply(divide).apply(2L))
                .apply(this.@<Long, Long, Long@>ap().apply(multiply).apply(add.apply(1L)));
        }
    })._eval();
}  
\end{lstlisting}
