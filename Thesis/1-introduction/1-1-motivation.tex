\section{Motivazione}
\label{sec:1-1-motivation}

I linguaggi di programmazione più diffusi non sempre possono essere classificati
solamente come procedurali, ad oggetti, logici o funzionali: soprattutto con la crescente adozione di linguaggi funzionali
nell'industria e il progresso della tecnologia per l'esecuzione di algoritmi paralleli, l'introduzione di nuove \textit{feature}
ispirate dalla programmazione funzionale è ben gradita dalla maggior parte degli sviluppatori.

\noindent Molti linguaggi moderni possono più opportunamente essere chiamati "ibridi" o "multi-paradigma",
in virtù della combinazione di diverse metodologie.


\texttt{Java} nello specifico, a partire dalla versione 8, è stato arricchito con nuovi strumenti e costrutti
che permettono di scrivere codice più conciso e dichiarativo: la domanda che ci si pone è quindi se sia possibile tradurre
un linguaggio funzionale in codice imperativo, sfruttando le nuove funzionalità così come la programmazione ad oggetti.